\subsection{Стандарты информационной безопасности и криптографической защиты информации}

\subsubsection*{1. Область применения}

\begin{itemize}
    \item \textbf{ГОСТ Р 59162-2020:}
    \begin{quote}
        В настоящем стандарте описаны угрозы, требования к информационной безопасности (ИБ), меры по контролю и проектированию систем безопасности, связанные с беспроводными сетями. Настоящий стандарт содержит рекомендации по выбору, реализации и мониторингу мер по обеспечению безопасности обмена информации через беспроводные сети \cite[стр.~1]{gost_r_59162}.
    \end{quote}
    Ориентирован на обеспечение безопасности беспроводных сетей.

    \item \textbf{Р 50.1.115-2016:}
    \begin{quote}
        Настоящие рекомендации предназначены для применения в информационных системах, использующих механизмы шифрования и защиты аутентичности данных... Протокол может применяться для обеспечения защиты каналов связи \cite[стр.~1]{r_50_1_115}.
    \end{quote}
    Ориентирован на криптографическую защиту информации и выработку общего ключа.
\end{itemize}

\subsubsection*{2. Методы защиты информации}

\begin{itemize}
    \item \textbf{ГОСТ Р 59162-2020:}
    \begin{quote}
        Предусматривается использование методов контроля шифрования, оценка целостности, аутентификация, авторизация и постоянный мониторинг беспроводных сетей \cite[стр.~12]{gost_r_59162}.
    \end{quote}

    \item \textbf{Р 50.1.115-2016:}
    \begin{quote}
        Протокол использует операции в группе точек эллиптической кривой, хэш-функции и алгоритмы проверки сообщений \cite[стр.~3]{r_50_1_115}.
    \end{quote}
\end{itemize}

\subsubsection*{3. Особенности реализации}

\begin{itemize}
    \item \textbf{ГОСТ Р 59162-2020:}
    \begin{quote}
        Стандарт направлен на определение мер по защите сетей, включая контроль доступа, устойчивость к атакам «отказ в обслуживании» и использование демилитаризированных зон \cite[стр.~16]{gost_r_59162}.
    \end{quote}

    \item \textbf{Р 50.1.115-2016:}
    \begin{quote}
        Протокол включает этапы генерации точек на эллиптической кривой, обмен сообщений для аутентификации и выработку общего ключа \cite[стр.~4]{r_50_1_115}.
    \end{quote}
\end{itemize}

\subsubsection*{4. Основные цели стандартов}

\begin{itemize}
    \item \textbf{ГОСТ Р 59162-2020:}
    \begin{quote}
        Обеспечение информационной безопасности при использовании беспроводных IP-сетей с учетом современных угроз и рисков \cite[стр.~IV]{gost_r_59162}.
    \end{quote}

    \item \textbf{Р 50.1.115-2016:}
    \begin{quote}
        Обеспечение безопасной выработки ключевой информации и взаимной аутентификации при наличии активного противника \cite[стр.~1]{r_50_1_115}.
    \end{quote}
\end{itemize}

\subsubsection*{Вывод}

\begin{itemize}
    \item \textbf{ГОСТ Р 59162-2020} фокусируется на обеспечении безопасности беспроводных сетей, включая контроль шифрования и устойчивость к атакам.
    \item \textbf{Р 50.1.115-2016} ориентирован на криптографическую защиту информации, предлагая протокол выработки общего ключа с аутентификацией.
\end{itemize}

\paragraph{Заключение:}
ГОСТ Р 59162-2020 лучше подходит для обеспечения комплексной безопасности беспроводных сетей, в то время как Р 50.1.115-2016 является специализированным стандартом для защиты информации в системах, требующих криптографической аутентификации.
Оба стандарта взаимно дополняют друг друга в процессе создания защищённых информационных систем.
