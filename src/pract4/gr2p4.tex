\subsection{Стандарты информационной безопасности и криптографической защиты информации}


\subsubsection*{ГОСТ Р 59162–2020}
\paragraph{Достоинства:}
\begin{itemize}
\item \textbf{Комплексный подход к безопасности беспроводных сетей:} Стандарт охватывает меры информационной безопасности, угрозы и рекомендации по защите беспроводных IP-сетей (раздел 1, область применения).
\item \textbf{Разделение на категории сетей:} Учитываются различия между WPAN, WLAN, WMAN и их специфические особенности безопасности (п. 6.1–6.3).
\item \textbf{Учет современных угроз:} Рассматриваются различные типы атак, включая отказ в обслуживании, анализ пакетов и атаки через Bluetooth (раздел 7).
\item \textbf{Рекомендации по проектированию систем:} Стандарт предоставляет методы проектирования, включая управление уязвимостями и использование безопасных конфигураций (раздел 9).
\end{itemize}

\paragraph{Недостатки:}
\begin{itemize}
\item \textbf{Фокус на беспроводные сети:} Ограниченная применимость к другим типам сетей, что делает стандарт менее универсальным для комплексных инфраструктур (раздел 1).
\item \textbf{Отсутствие подробных криптографических рекомендаций:} Не предоставляет детализированных алгоритмов, сосредотачиваясь на архитектурных аспектах (разделы 8, 9).
\end{itemize}

\subsubsection*{Р 50.1.115–2016}
\paragraph{Достоинства:}
\begin{itemize}
\item \textbf{Криптографическая основа:} Описывает протокол выработки общего ключа с использованием пароля и методами эллиптической криптографии (раздел 4).
\item \textbf{Универсальность:} Применим как для корпоративных, так и для общедоступных сетей, обеспечивая защиту каналов связи (раздел 1).
\item \textbf{Высокий уровень безопасности:} Использует современные криптографические подходы, такие как хэш-функции и электронная подпись (п. 4.1, 4.3).
\end{itemize}

\paragraph{Недостатки:}
\begin{itemize}
\item \textbf{Ограниченная область применения:} Сосредоточен на отдельных аспектах криптографии, не охватывая архитектурные и организационные аспекты безопасности (раздел 1).
\item \textbf{Сложность реализации:} Реализация алгоритмов требует высокой квалификации, что может усложнить их применение в малых организациях (п. 4.3, 5).
\end{itemize}

\subsubsection*{Вывод}
\begin{itemize}
\item \textbf{ГОСТ Р 59162–2020:} Лучше подходит для защиты беспроводных сетей, предоставляя рекомендации по архитектурным решениям и учету современных угроз.
\item \textbf{Р 50.1.115–2016:} Более уместен для обеспечения криптографической безопасности, особенно при необходимости выработки ключей и аутентификации.
\end{itemize}

Для приложения рекомендательной системы фильмов наиболее подходящим стандартом является \textbf{ГОСТ Р 50.1.115-2016}
