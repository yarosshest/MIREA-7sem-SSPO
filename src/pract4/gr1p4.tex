\subsection{Стандарты управления данными и взаимодействия информационных систем}

\subsubsection*{ГОСТ 34.321-96}

\paragraph{Достоинства:}
\begin{itemize}
    \item \textbf{Универсальность управления данными:} Устанавливает эталонную модель, определяющую общую терминологию
    и понятия для управления данными в информационных системах (раздел 1, область применения).
    \item \textbf{Независимость данных:} Обеспечивает независимость процессов от объектов данных, что позволяет изменять
    объекты данных без нарушения процессов (п. 4.4).
    \item \textbf{Управление доступом:} Предусматривает санкционированный доступ и предотвращение несанкционированного
    использования ресурсов (п. 4.6).
    \item \textbf{Реструктурирование данных:} Поддерживает логическое реструктурирование данных для повышения их
    эффективности в течение жизненного цикла (п. 4.7.10).
    \item \textbf{Работа с распределёнными системами:} Рассматриваются аспекты управления данными, включая фрагментацию,
    дублирование и восстановление в распределённых базах данных (п. 4.8).
\end{itemize}

\paragraph{Недостатки:}
\begin{itemize}
    \item \textbf{Ограниченное внимание к семантике:} Семантика данных рассматривается лишь косвенно, через правила
    структурирования и моделирования данных (раздел 5).
    \item \textbf{Меньший акцент на пользовательском взаимодействии:} Отсутствует глубокий фокус на восприятии данных
    конечным пользователем.
\end{itemize}

\subsubsection*{ГОСТ Р 43.0.11-2014}

\paragraph{Достоинства:}
\begin{itemize}
    \item \textbf{Семантическая организация данных:} Делает акцент на перцептивное и грамматическое представление данных для упрощения восприятия и принятия решений (п. 5.8).
    \item \textbf{Удобство для операторов:} Создаёт базы данных, предназначенные для специалистов, обеспечивая удобство осмысления и взаимодействия (п. 1, область применения).
    \item \textbf{Поддержка знаний:} Рассматривает базы данных как основу для создания баз знаний и упрощения их применения в обучении и практике (приложение А) .
\end{itemize}

\paragraph{Недостатки:}
\begin{itemize}
    \item \textbf{Ограниченный охват распределённых систем:} Не рассматривает работу с распределёнными базами данных.
    \item \textbf{Независимость данных:} Проблема независимости данных напрямую не поднимается, акцент сделан на их семантической организации (раздел 5).
    \item \textbf{Меньший фокус на безопасности:} Управление доступом связано с восприятием данных, а не с предотвращением несанкционированного доступа (п. 5.2).
\end{itemize}

\subsubsection*{Вывод}
\begin{itemize}
    \item \textbf{ГОСТ 34.321-96:} Подходит для систем, где важны универсальность, согласованность данных и работа в распределённых системах.
    \item \textbf{ГОСТ Р 43.0.11-2014:} Лучше для областей, где важны удобство восприятия данных и их использование в практической деятельности.
\end{itemize}
Для рекомендательной системы предпочтителен \textbf{ГОСТ 34.321-96}, ориентированный на устойчивость и согласованность данных.
