\section*{\LARGE Цель практической работы}
\addcontentsline{toc}{section}{Цель практической работы}

\textbf{Цель практической работы:}
Получить навыки по определению требуемых областей
стандартизации и по поиску страндартизирующих комитетов и
существующих стандартов, созданных данными комитетами. Получить
навыки анализа стандартов предметной области для выбора наиболее
подходящих из них на основании перечня, полученного при анализе
областей стандартизации.

\textbf{Задание на практическую работу:}

\begin{enumerate}
	\item Опираясь на подпроцессы выбранной области определить
		группы стандартов, которые могут быть применены
		для выбранного дипломного проекта.
		Минимально необходимо выбрать 5 групп стандартов.
	\item Определить, какие комитеты и подкомитеты занимаются
		стандартизацией областей.
	\item Опираясь на выбранные комитеты и подкомитеты,
		определить какие стандарты могут быть использованы
		для реализуемого проекта.
		Для каждой выделенной группы стандартов необходимо
		найти минимум 2 стандарта, пригодных для проекта.
	\item На основании выбранных стандартов провести исследование,
		какие из них наиболее уместно использовать в предлагаемой разработке.
		Обоснование должно заключаться в приведении конкретных достоинств
		и недостатков выбранных стандартов по сравнению с другими.
		При приведении конкретных недостатков или достоинств стандарта
		необходимо ссылаться на пункт,
		обуславливающий это достоинство или недостаток.
\end{enumerate}

\clearpage

\section*{\LARGE Выполнение практической работы}
\addcontentsline{toc}{section}{Выполнение практической работы}

\section{Группы стандартов}

Разрабатываемый программный продукт предполагает следующие группы стандартов:

\begin{itemize}
    \item Стандарты управления данными и взаимодействия информационных систем;
    \item Стандарты информационной безопасности и криптографической защиты информации;
    \item Стандарты производительности и качества ИТ продуктов и систем;
    \item Стандарты портативности прикладного программного обеспечения;
    \item Стандарты интероперабельности ИТ продуктов и систем;
    \item Стандарты унификации инструментов и средств разработки;
    \item Стандарты документирования ПО.
\end{itemize}

\subsection{Стандарты управления данными и взаимодействия информационных систем}

\section*{Общее описание}
Группа стандартов направлена на обеспечение эффективного управления данными и организацию корректного взаимодействия информационных систем. Стандарты регламентируют:
\begin{itemize}
    \item Основные подходы к проектированию и использованию систем управления данными;
    \item Принципы совместимости и интеграции различных информационных систем;
    \item Терминологическую основу для разработчиков и пользователей.
\end{itemize}

\section*{Основные положения}
\subsection*{Управление данными}
\begin{itemize}
    \item Организация хранения, обработки, поиска и восстановления данных.
    \item Определение структуры данных через схемы и модели, такие как реляционная и иерархическая.
    \item Обеспечение целостности, актуальности и единообразия данных.
\end{itemize}

\subsection*{Взаимодействие информационных систем}
\begin{itemize}
    \item Установление единых протоколов и интерфейсов для обмена данными.
    \item Реализация независимости от используемых технологий и платформ.
    \item Защита данных через управление доступом и идентификацию пользователей.
\end{itemize}
\subsection{Производительность и качество ИТ продуктов и систем}

Стандарты производительности и качества ИТ-продуктов и систем
направлены на обеспечение высоких показателей работы программного обеспечения
и оборудования.

Они включают:

\begin{itemize}
    \item Метрики производительности, такие как время отклика системы,
    пропускная способность сети, скорость обработки данных и т. д.
    \item Контроль качества и тестирование продуктов на этапах разработки
    и эксплуатации для выявления ошибок и уязвимостей.
    \item Системы управления качеством, такие как ISO/IEC 25010,
    которые определяют критерии качества программного обеспечения
    (функциональность, надежность, эффективность).
    \item Эти стандарты обеспечивают, чтобы продукты
    и системы отвечали ожиданиям пользователей по качеству и стабильности.
\end{itemize}




\subsection{Стандарты информационной безопасности и криптографической защиты информации}

\section*{Общее описание}
Группа стандартов направлена на обеспечение безопасности информационных систем и защиты данных. Стандарты регламентируют:
\begin{itemize}
    \item Основные подходы к проектированию и использованию систем безопасности;
    \item Принципы совместимости и интеграции различных информационных систем;
    \item Терминологическую основу для разработчиков и пользователей.
\end{itemize}

\section*{Основные положения}
\subsection*{Безопасность беспроводных сетей}
\begin{itemize}
    \item Определение угроз и требований к информационной безопасности.
    \item Предложение мер контроля и проектирования систем безопасности для беспроводных сетей.
\end{itemize}

\subsection*{Криптографическая защита информации}
\begin{itemize}
    \item Описание протоколов выработки ключей и взаимной аутентификации.
    \item Применение в корпоративных и общедоступных сетях.
\end{itemize}

\section*{Цель и задачи группы стандартов}
\begin{itemize}
    \item Обеспечение конфиденциальности, целостности и доступности информации в информационных системах.
    \item Разработка рекомендаций по проектированию и реализации мер безопасности в сетях.
    \item Использование современных криптографических методов для защиты данных.
\end{itemize}

\section*{Примечания}
Данная группа стандартов предназначена для специалистов, занимающихся информационной безопасностью, включая:
\begin{itemize}
    \item Разработчиков и проектировщиков систем безопасности.
    \item Сетевых администраторов и инженеров.
    \item Сотрудников, ответственных за соблюдение нормативных требований в области безопасности.
\end{itemize}

\subsection{Интероперабельность ИТ продуктов и систем}

Интероперабельность означает способность различных ИТ-систем
и продуктов взаимодействовать друг с другом и обмениваться данными.

Стандарты интероперабельности охватывают:

\begin{itemize}
    \item Сетевые протоколы (например, TCP/IP, HTTP)
    для обмена данными между различными устройствами и сетями.
    \item Форматы данных (XML, JSON),
    которые обеспечивают единый способ обмена информацией
    между приложениями.
    \item API и интерфейсы для интеграции различных программных систем
    и сервисов.
    \item Эти стандарты критически важны для построения взаимосвязанных систем,
    таких как интернет вещей (IoT), облачные сервисы и корпоративные сети.
\end{itemize}

\subsection{Документирование ПО}

Стандарты в области документирования разрабатывают требования
и рекомендации для создания, оформления
и управления документацией в самых разных сферах --- от производства
и проектирования до обслуживания и безопасности.
Такие стандарты призваны обеспечить единство,
качество и удобство использования документов,
а также повысить эффективность процессов обмена информацией
и минимизировать риски,
связанные с неправильным или недостаточным документированием.

Основные цели стандартов на документирование:

\begin{itemize}
    \item Унификация и стандартизация.
    Устанавливают единые правила для создания документов,
    что облегчает их понимание, проверку и сравнение.
    Это особенно важно в компаниях и отраслях,
    где несколько отделов или организаций взаимодействуют друг с другом.
    \item Повышение качества информации.
    Стандарты требуют точности, актуальности,
    достоверности и полноты данных.
    Это помогает избежать ошибок и недопонимания,
    которые могут привести к производственным сбоям,
    потерям и несоответствиям.
    \item Соответствие требованиям законодательства.
    В ряде отраслей (например, в фармацевтике, авиации, энергетике)
    требования к документированию строго регламентированы,
    и несоответствие стандартам может привести к штрафам
    или остановке деятельности.
\end{itemize}

\clearpage

\section{Комитеты и подкомитеты}

Технический комитет (ТК) по стандартизации --- это объединение специалистов,
являющихся полномочными представителями заинтересованных предприятий
(организаций) --- членов ТК,
создаваемое на добровольной основе для разработки национальных стандартов РФ,
проведения работ в области международной (региональной)
стандартизации по закрепленным за ТК объектам стандартизации
(областям деятельности).

\begin{longtable}{|p{2cm}|p{14cm}|}
    \caption{Комитеты} \label{table:tk} \\
    \hline
    \textbf{\No\ ТК}
    & \textbf{Наименование ТК} \\
    \hline
    \endfirsthead
    \conttable{table:tk} \\
    \hline
    \textbf{\No\ ТК}
    & \textbf{Наименование ТК} \\
    \hline
    \endhead
    Отсутс - твует & Комитет Украины по стандартизации, метрологии и сертификации\\ \hline
    \textbf{022} & Информационные технологии \\ \hline
    \textbf{026} & Криптографическая защита информации \\ \hline
    \textbf{191}
    & Научно-техническая информация, библиотечное и издательское дело \\ \hline
    \textbf{362}
    & Защита информации \\ \hline
    \textbf{379} & Информационное обеспечение техники и операторской деятельности \\ \hline
    \textbf{482}
    & Поддержка жизненного цикла продукции \\ \hline
    \textbf{700}
    & Математическое моделирование
    и высокопроизводительные вычислительные технологии \\ \hline
\end{longtable}

\subsection{Подкомитеты ТК 22 Информационные технологии}

\begin{longtable}{|p{2cm}|p{14cm}|}
    \caption{Подкомитеты ТК 22 Информационные технологии}
    \label{table:tk:22} \\
    \hline
    \textbf{\No\ ПК}
    & \textbf{Наименование ПК} \\
    \hline
    \endfirsthead
    \conttable{table:tk:22} \\
    \hline
    \textbf{\No\ ПК}
    & \textbf{Наименование ПК} \\
    \hline
    \endhead
    ПК107 (SC7) & Системная и программная инженерия \\ \hline
    ПК122 (SC22)
    & Языки программирования, их окружение
    и системы программных интерфейсов \\ \hline
    ПК125 (SC25)
    & Взаимосвязь оборудования для информационных технологий \\ \hline
    ПК127 (SC27) & Безопасность информационных технологий \\ \hline
    ПК128 (SC28) & Оборудование офисов \\ \hline
    ПК134 (SC34) & Описание документа и языки обработки \\ \hline
    ПК135 (SC35) & Пользовательские интерфейсы \\ \hline
    ПК138 (SC38) & Платформы и сервисы для распределенных приложений \\ \hline
    ПК140 (SC40)
    & Управление информационными технологиями и услугами ИТ \\ \hline
    ПК201 & Терминология в ИТ \\ \hline
    ПК206 & Интероперабельность \\ \hline
\end{longtable}


\begin{longtable}{|p{2cm}|p{14cm}|}
    \caption{Подкомитеты ТК 26 Криптографическая защита информации}
    \label{table:tk:26} \\
    \hline
    \textbf{\No\ ПК}
    & \textbf{Наименование ПК} \\
    \hline
    \endfirsthead
    \conttable{table:tk:26} \\
    \hline
    \textbf{\No\ ПК}
    & \textbf{Наименование ПК} \\
    \hline
    \endhead
    ПК1 & Криптографические алгоритмы и протоколы для применения в поставляемых для федеральных государственных
    нужд шифровальных (криптографических) средствах защиты информации, содержащей сведения, составляющие государственную тайну \\ \hline
    ПК2
    & Криптографические алгоритмы и протоколы для применения в поставляемых для федеральных государственных нужд
    шифровальных (криптографических) средствах защиты информации, содержащей сведения, относимые к охраняемой в
    соответствии с законодательством Российской Федерации информации ограниченного доступа \\ \hline
    ПК3
    & Криптографические алгоритмы и механизмы в национальной платежной системе Российской Федерации \\ \hline
    ПК4 & Российские шифровальные (криптографические) средства защиты информации, не содержащей сведений, составляющих
    государственную тайну, или относимых к охраняемой в соответствии с законодательством Российской Федерации к
    информации ограниченного доступа, а также зарубежные шифровальные (криптографические) средства защиты информации
    на территории Российской Федерации \\ \hline
\end{longtable}

\clearpage

\section{Выбранные стандарты}


\subsection{Стандарты управления данными и взаимодействия информационных систем}
\subsubsection{ГОСТ 34.321 -- 96}
\paragraph{ГОСТ 34.321 -- 96}
\emph{\href{https://internet-law.ru/gosts/gost/6808/}{ГОСТ 34.321 -- 96}
Информационные технологии.
Система стандартов по базам данных.
Эталонная модель управления данными
}
\par
Комитет: TK 022 Информационные технологии
\section*{Общее описание}
ГОСТ 34.321-96 устанавливает эталонную модель управления данными в информационных системах.
Стандарт определяет терминологию, основные концепты и требования, связанные с управлением базами данных.
Эталонная модель служит для описания услуг, предоставляемых системами управления базами данных (СУБД) и системами словарей данных.

\section*{Область применения}
Стандарт распространяется на процессы управления постоянными данными, включая их хранение, поиск, обновление, ввод, копирование, восстановление и передачу. Он ориентирован на:
\begin{itemize}
    \item Определение общих принципов управления данными;
    \item Обеспечение совместимости между различными системами;
    \item Создание единой терминологической основы для разработчиков и пользователей.
\end{itemize}

\section*{Термины и определения}
В рамках стандарта введены следующие ключевые термины:
\begin{itemize}
    \item \textbf{База данных (database):} Это совокупность взаимосвязанных данных, организованных в соответствии с определённой схемой, для обеспечения удобного доступа и обработки.
    \item \textbf{Схема базы данных (database schema):} Это формальное описание содержания, структуры и ограничений целостности базы данных.
    \item \textbf{Транзакция (transaction):} Совокупность операций, которая характеризуется такими свойствами, как атомарность, согласованность, изоляция и долговечность (ACID-свойства).
    \item \textbf{Управление доступом (access control):} Это процесс предотвращения несанкционированного доступа к данным путём использования различных механизмов идентификации и авторизации.
\end{itemize}

\section*{Основные компоненты модели}

\subsection*{База данных и схема}
Основным элементом модели является база данных, структура и содержание которой определяются её схемой. Схема включает описание элементов данных, их связей и правил целостности. Такое представление обеспечивает логическую организацию данных и их доступность для пользователей.

\subsection*{Средства моделирования данных}
Для создания схем базы данных применяются средства моделирования данных. Эти средства включают:
\begin{itemize}
    \item \textbf{Правила структурирования данных:} Определяют, как данные организуются и связываются друг с другом.
    \item \textbf{Правила манипулирования данными:} Определяют допустимые операции над данными.
\end{itemize}
Наиболее распространёнными моделями данных являются реляционная, сетевая и иерархическая модели.

\subsection*{Независимость данных}
Независимость данных обеспечивается следующими подходами:
\begin{itemize}
    \item Процессы взаимодействуют только с необходимыми частями схемы базы данных.
    \item Прикладные процессы отделены от физического представления данных.
    \item Ограничения целостности реализуются непосредственно в схеме базы данных.
\end{itemize}

\subsection*{Процессоры и интерфейсы}
Процессы управления данными выполняются с помощью процессоров, которые предоставляют интерфейсы для доступа к функционалу системы. Эти интерфейсы могут быть универсальными и независимыми от языков программирования.

\section*{Управление доступом}
Управление доступом в эталонной модели основывается на следующих принципах:
\begin{itemize}
    \item \textbf{Определение и модификация привилегий пользователей:} Позволяет устанавливать уровни доступа для различных категорий пользователей.
    \item \textbf{Реализация санкционированного доступа:} Осуществляется через процедуры идентификации и аутентификации пользователей.
    \item \textbf{Сохранение данных об управлении доступом:} Вся информация о привилегиях и ограничениях хранится непосредственно в базе данных.
\end{itemize}

\section*{Транзакции базы данных}
Транзакции базы данных представляют собой логические единицы работы, обеспечивающие надёжное выполнение операций. Основные требования к транзакциям включают:
\begin{itemize}
    \item \textbf{Атомарность:} Транзакция либо выполняется полностью, либо не выполняется вовсе.
    \item \textbf{Согласованность:} После выполнения транзакции данные остаются в непротиворечивом состоянии.
    \item \textbf{Изоляция:} Параллельно выполняемые транзакции не влияют на корректность данных.
    \item \textbf{Долговечность:} Изменения, произведённые транзакцией, сохраняются даже в случае сбоя системы.
\end{itemize}

\section*{Распределённые базы данных}
Распределённые базы данных организуют хранение данных на нескольких компьютерах, что обеспечивает масштабируемость и надёжность системы. Основные требования включают:
\begin{itemize}
    \item \textbf{Управление фрагментацией и дублированием данных:} Гарантирует оптимальное распределение данных между узлами.
    \item \textbf{Независимость распределения:} Пользователи не должны знать о физическом расположении данных.
    \item \textbf{Восстановление:} Обеспечивает целостность данных при сбоях.
\end{itemize}

\section*{Восстановление и журналирование}
Для обеспечения целостности и надёжности данных применяются следующие методы:
\begin{itemize}
    \item \textbf{Контрольные журналы:} Фиксируют изменения данных для возможности их последующего анализа и восстановления.
    \item \textbf{Резервное копирование:} Позволяет восстанавливать данные до состояния, предшествующего сбою.
    \item \textbf{Реорганизация физической памяти:} Выполняется для оптимизации хранения данных при изменениях структуры базы.
\end{itemize}

\subsubsection{ГОСТ Р 43.0.11 -- 2014}
\paragraph{ГОСТ Р 43.0.11 -- 2014}

\emph{\href{https://internet-law.ru/gosts/gost/57862/}{ГОСТ Р 43.0.11 -- 2014}
Информационное обеспечение техники и операторской деятельности.
Базы данных в технической деятельности
}
\par
ТК 379 «Информационное обеспечение
техники и операторской деятельности»
\section*{Общее описание}
ГОСТ Р 43.0.11-2014 устанавливает основные требования и принципы обеспечения совместимости информационных систем, определяя правила и процедуры для их взаимодействия. Стандарт ориентирован на повышение эффективности информационного обмена между различными системами.

\section*{Область применения}
Стандарт применяется к информационным системам, функционирующим в различных отраслях. Основные цели использования:
\begin{itemize}
    \item Определение правил совместимости компонентов информационных систем.
    \item Обеспечение корректного обмена данными между разнородными системами.
    \item Повышение надёжности и устойчивости систем к сбоям.
\end{itemize}

\section*{Термины и определения}
ГОСТ Р 43.0.11-2014 включает следующие ключевые термины:
\begin{itemize}
    \item \textbf{Информационная система (ИС):} Совокупность программных, аппаратных и организационных средств, предназначенных для обработки данных.
    \item \textbf{Совместимость:} Способность компонентов системы взаимодействовать для достижения заданной функциональности.
    \item \textbf{Интерфейс:} Средство взаимодействия между компонентами системы или между системами.
    \item \textbf{Протокол:} Набор правил для обмена данными между взаимодействующими компонентами.
\end{itemize}

\section*{Основные положения стандарта}

\subsection*{Совместимость систем}
Стандарт определяет совместимость как обязательное свойство информационных систем, достигаемое путём унификации процедур взаимодействия. Это включает:
\begin{itemize}
    \item Установление общих требований к интерфейсам.
    \item Определение единых форматов данных.
    \item Разработку протоколов обмена данными.
\end{itemize}

\subsection*{Модели взаимодействия}
ГОСТ Р 43.0.11-2014 предлагает модели взаимодействия систем, направленные на обеспечение их совместимости. Основные модели:
\begin{itemize}
    \item \textbf{Модель клиент-сервер:} Одна система (клиент) запрашивает данные или услуги у другой системы (сервер).
    \item \textbf{Модель однорангового взаимодействия:} Системы функционируют на равных правах, обмениваясь данными без центрального управляющего звена.
    \item \textbf{Модель шина данных:} Интеграция через централизованный канал, обеспечивающий передачу данных между системами.
\end{itemize}

\subsection*{Интерфейсы и протоколы}
Для обеспечения корректного взаимодействия между компонентами стандартизированы:
\begin{itemize}
    \item \textbf{Интерфейсы:} Должны быть открытыми и документированными, чтобы исключить проблемы интеграции.
    \item \textbf{Протоколы:} Определяют формат и последовательность действий для передачи данных.
\end{itemize}

\subsection*{Управление доступом и безопасностью}
Для защиты данных и обеспечения их конфиденциальности в стандарте предусматриваются следующие меры:
\begin{itemize}
    \item Использование механизмов идентификации и аутентификации пользователей.
    \item Контроль доступа к ресурсам системы на основе заданных политик.
    \item Шифрование данных при их передаче между системами.
\end{itemize}

\section*{Требования к данным}
ГОСТ Р 43.0.11-2014 предъявляет строгие требования к качеству и формату данных:
\begin{itemize}
    \item \textbf{Целостность данных:} Обеспечивается за счёт механизмов контроля ошибок и резервирования.
    \item \textbf{Актуальность:} Данные должны своевременно обновляться для обеспечения достоверности.
    \item \textbf{Единообразие:} Форматы данных и кодировки должны быть согласованы между системами.
\end{itemize}

\section*{Обеспечение надёжности}
Для обеспечения надёжной работы систем стандарт предусматривает:
\begin{itemize}
    \item \textbf{Дублирование ключевых компонентов:} Уменьшает риск отказа системы.
    \item \textbf{Резервное копирование данных:} Позволяет восстановить систему в случае сбоя.
    \item \textbf{Мониторинг и диагностика:} Обеспечивают контроль за состоянием системы и выявление неисправностей.
\end{itemize}

\section*{Применение стандарта}
ГОСТ Р 43.0.11-2014 рекомендуется использовать при разработке новых и модернизации существующих информационных систем. Его применение способствует:
\begin{itemize}
    \item Повышению уровня совместимости и интеграции между системами.
    \item Уменьшению затрат на разработку и сопровождение систем.
    \item Созданию надёжной и безопасной инфраструктуры для обработки данных.
\end{itemize}
\subsection{Стандарты информационной безопасности и криптографической защиты информации}

\subsubsection{ГОСТ Р 50.1.115 -- 2016}
\paragraph{ГОСТ Р 50.1.115 -- 2016}
\emph{\href{https://rst.gov.ru:8443/file-service/file/load/1699435168541}{Р 50.1.115 -- 2016}
Информационные технологии.
Криптографическая защита информации.
Протокол выработки общего ключа с аутентификацией на основе пароля}

\section*{Общее описание}
Настоящий ГОСТ описывает протокол выработки общего ключа с использованием пароля для аутентификации сторон. Протокол основан на эллиптических кривых и обеспечивает защиту от активного противника.

\section*{Область применения}
Стандарт применяется в системах с использованием криптографии для защиты данных, включая электронные цифровые подписи и хэширование. Основное назначение — установление защищенного соединения между сторонами.

\section*{Основные положения}
\begin{itemize}
    \item Использование эллиптических кривых для формирования ключей.
    \item Поддержка строгой аутентификации и защиты данных.
    \item Обеспечение защиты каналов связи от перехвата.
\end{itemize}

\subsubsection{ГОСТ Р 59162 -- 2020}
\paragraph{ГОСТ Р 59162 -- 2020}
\emph{\href{https://rst.gov.ru:8443/file-service/file/load/1699366818935}{ГОСТ Р 59162 -- 2020}
Информационные технологии.
Методы и средства обеспечения безопасности.
Безопасность сетей. Часть 6: Обеспечение информационной безопасности при использовании беспроводных IP-сетей.}

\section*{Общее описание}
Стандарт определяет методы и меры обеспечения информационной безопасности для беспроводных IP-сетей. Он основан на международных стандартах ISO/IEC 27033-6.

\section*{Область применения}
Применяется для проектирования, реализации и мониторинга систем безопасности беспроводных сетей в различных организациях.

\section*{Основные положения}
\begin{itemize}
    \item Определение угроз и рисков для беспроводных сетей.
    \item Реализация механизмов шифрования, аутентификации и контроля доступа.
    \item Поддержание устойчивости к атакам и обеспечение конфиденциальности данных.
\end{itemize}


\subsection{Производительность и качество ИТ продуктов и систем}

\subsubsection{ГОСТ Р 57700.26 -- 2020}

\emph{\href{https://docs.cntd.ru/document/573114591}{ГОСТ Р 57700.26 -- 2020}
Высокопроизводительные вычислительные системы.
Требования к приемочным испытаниям
}

Настоящий стандарт распространяется
на высокопроизводительные вычислительные системы (ВВС)
универсального назначения,
применяемые для решения широкого круга задач компьютерного моделирования
с использованием алгоритмов распараллеливания,
и устанавливает требования к видам испытаний и реализации их результатов,
порядку их проведения, а также к составу документов,
применяемых в процессе испытаний.

\subsubsection{ГОСТ Р ИСО/МЭК 9126 -- 93}

\emph{\href{https://docs.cntd.ru/document/1200009076}{ГОСТ Р ИСО/МЭК 9126 -- 93}
Информационная технология.
Оценка программной продукции.
Характеристики качества и руководства по их применению
}

Настоящий стандарт определяет шесть характеристик,
которые с минимальным дублированием описывают качество программного обеспечения.
Данные характеристики образуют основу для дальнейшего уточнения
и описания качества программного обеспечения.
Руководства описывают использование характеристик качества
для оценки качества программного обеспечения.

Настоящий стандарт не определяет подхарактеристики (комплексные показатели)
и показатели, а также методы измерения, ранжирования и оценки.
Данный стандарт придерживается определения качества по ИСО 8402.

Определения характеристик и соответствующая модель процесса оценки качества,
приведенные в настоящем стандарте, применимы тогда,
когда определены требования для программной продукции
и оценивается ее качество в процессе жизненного цикла.

Эти характеристики могут применяться к любому виду программного обеспечения,
включая программы ЭВМ и данные,
входящие в программно-технические средства (встроенные программы).

Настоящий стандарт предназначен для характеристик,
связанных с приобретением, разработкой, эксплуатацией, поддержкой,
сопровождением или проверкой программного обеспечения.

\subsubsection{ГОСТ Р ИСО/МЭК 25041 -- 2014}

\emph{\href{https://docs.cntd.ru/document/1200111122}
{ГОСТ Р ИСО/МЭК 25041 -- 2014}
Информационные технологии.
Разработка систем и программ.
Требования и оценивание качества систем и программ.
Руководство по оцениванию для разработчиков, покупателей и независимых оценщиков.
}

Настоящий стандарт содержит требования,
рекомендации и методические материалы по оценке качества продукции,
предназначенные для разработчиков, приобретателей и независимых оценщиков.
Однако он не ограничивается какой-либо конкретной прикладной областью
и может быть использован для оценки качества любого типа продукции.

Настоящий стандарт описывает процессы оценки качества продукции
и устанавливает специальные требования
для применения процесса оценки с точки зрения разработчиков,
приобретателей и независимых оценщиков.
Процесс оценки можно использовать для различных целей и подходов.
Процесс можно использовать
для оценки качества ранее разработанного программного обеспечения,
готового к использованию программного обеспечения
или разработанного по заказу как в процессе, так и по завершении разработки.

Настоящий стандарт предназначен для лиц,
ответственных за оценку программного продукта, однако,
может быть использован также разработчиками,
приобретателями и независимыми оценщиками продуктов.

Настоящий стандарт не предназначен
для оценки других аспектов программных продуктов,
таких как функциональные требования,
требования к процессу, бизнес-требования, и т.д.

\subsection{Интероперабельность ИТ продуктов и систем}

\subsubsection{ГОСТ Р 55062 -- 2021}

\emph{\href{https://docs.cntd.ru/document/1200181340}{ГОСТ Р 55062 -- 2021}
Информационные технологии. Интероперабельность. Основные положения
}

Настоящий стандарт определяет:

\begin{itemize}
	\item основные понятия, связанные с понятием <"интероперабельность">;
	\item эталонную модель интероперабельности;
	\item единый подход (методику) к обеспечению интероперабельности
		информационных систем широкого класса, представляющую метатехнологию;
	\item основные и вспомогательные этапы по достижению интероперабельности.
\end{itemize}

Настоящий стандарт предназначен для заказчиков, поставщиков, разработчиков,
потребителей, а также персонала, сопровождающего информационные системы
и осуществляющего программное обеспечение и услуги.

\subsubsection{ГОСТ Р 59797 -- 2021}

\emph{\href{https://docs.cntd.ru/document/1200181353}{ГОСТ Р 59797 -- 2021}
Информационные технологии. Сложные системы. Интероперабельность. Основные положения
}

Основная область применения настоящего стандарта
--- сложные системы различного назначения.
Настоящий стандарт предлагает набор общих правил по оценке
и обеспечению интероперабельности сложных систем.
Он может применяться на всех уровнях, включая национальный,
региональный и муниципальный,
включая государственные администрации, предприятия и организации.

Настоящий стандарт определяет:

\begin{itemize}
	\item основные термины и определения,
		связанные с понятием <<интероперабельность>> и <<сложная система>>;
	\item методику обеспечения
		и оценки интероперабельности сложных систем;
	\item описание содержания работ
		по достижению интероперабельности в ходе выполнения основных
		этапов создания (модернизации) сложных систем.
\end{itemize}

Настоящий стандарт предназначен для заказчиков,
разработчиков и потребителей,
а также персонала по сопровождению сложных систем.

\subsection{Документирование ПО}

\subsubsection{ГОСТ 34.201 -- 2020}

\emph{\href{https://docs.cntd.ru/document/1200181803}{ГОСТ 34.201 -- 2020}
Информационные технологии.
Комплекс стандартов на автоматизированные системы.
Виды, комплектность и обозначение документов
при создании автоматизированных систем.
}

Настоящий стандарт распространяется на автоматизированные системы (АС),
используемые в различных сферах деятельности
(управление, исследования, проектирование и т. п.), включая их сочетания,
и устанавливает требования к видам, наименованию,
комплектности и обозначению документов, разрабатываемых на стадиях создания АС.
В случае отсутствия выделения стадий (или деления на другие стадии)
при создании АС перечень разрабатываемой документации
и сроки ее представления определяются техническим заданием
или совместным решением заказчика и разработчика.

\subsubsection{ГОСТ Р 59988.02.1 -- 2022}

\emph{\href{https://docs.cntd.ru/document/1200192137}{ГОСТ Р 59988.02.1 -- 2022}
Системы автоматизированного проектирования электроники.
Информационное обеспечение.
Технические характеристики электронных компонентов.
Микросхемы интегральные.
Спецификации декларативных знаний по техническим характеристикам
}

Настоящий стандарт предназначен для применения при разработке
баз данных (БД), баз знаний (БЗ), технических заданий (ТЗ),
технических условий (ТУ) и прочего
и позволяет обеспечить семантическую однозначность данных
по техническим характеристикам (ТХ) электронной компонентной базы (ЭКБ).

Настоящий стандарт устанавливает правила и рекомендации
по применению в БД, БЗ и других информационных ресурсах:

\begin{itemize}
	\item предпочтительных наименований ТХ ЭКБ
		с перечнем применяемых на практике синонимов;
	\item определений ТХ ЭКБ;
	\item единиц измерения ТХ ЭКБ;
	\item квалификаторов измерения ТХ ЭКБ;
	\item типов данных ТХ ЭКБ.
\end{itemize}

Настоящий стандарт не распространяется на рассмотрение
всех проблем классификации и терминологии ТХ ЭКБ
и разработан в развитие требований государственных,
отраслевых стандартов и других руководящих документов по ЭКБ.

\subsubsection{ГОСТ Р 59988.03.1 -- 2022}

\emph{\href{https://docs.cntd.ru/document/1200195155}{ГОСТ Р 59988.03.1 -- 2022}
Системы автоматизированного проектирования электроники.
Информационное обеспечение.
Технические характеристики электронных компонентов.
Приборы и модули полупроводниковые.
Спецификации декларативных знаний по техническим характеристикам.
}

Настоящий стандарт предназначен для применения при разработке
баз данных (БД), баз знаний (БЗ), технических заданий (ТЗ),
технических условий (ТУ) и прочего,
и позволяет обеспечить семантическую однозначность данных
по техническим характеристикам (ТХ) электронной компонентной базы (ЭКБ).

Настоящий стандарт устанавливает правила и рекомендации
по применению в БД, БЗ и других информационных ресурсах:

\begin{itemize}
	\item предпочтительных наименований ТХ ЭКБ
		с перечнем применяемых на практике синонимов;
	\item определений ТХ ЭКБ;
	\item единиц измерения ТХ ЭКБ;
	\item квалификаторов измерения ТХ ЭКБ;
	\item типов данных ТХ ЭКБ.
\end{itemize}

Настоящий стандарт не распространяется на рассмотрение
всех проблем классификации и терминологии ТХ ЭКБ
и разработан в развитие требований государственных, отраслевых стандартов
и других руководящих документов по ЭКБ.

\clearpage

\section{Выбор подходящего стандарта для проекта}

Для выбора наиболее подходящих стандартов из каждой группы
для разрабатываемого проекта можно рассмотреть их достоинства и недостатки,
чтобы определить, какие из них лучше всего соответствуют специфике разработки.

\subsection{Стандарты управления данными и взаимодействия информационных систем}

\subsubsection*{ГОСТ 34.321-96}

\paragraph{Достоинства:}
\begin{itemize}
    \item \textbf{Универсальность управления данными:} Устанавливает эталонную модель, определяющую общую терминологию
    и понятия для управления данными в информационных системах (раздел 1, область применения).
    \item \textbf{Независимость данных:} Обеспечивает независимость процессов от объектов данных, что позволяет изменять
    объекты данных без нарушения процессов (п. 4.4).
    \item \textbf{Управление доступом:} Предусматривает санкционированный доступ и предотвращение несанкционированного
    использования ресурсов (п. 4.6).
    \item \textbf{Реструктурирование данных:} Поддерживает логическое реструктурирование данных для повышения их
    эффективности в течение жизненного цикла (п. 4.7.10).
    \item \textbf{Работа с распределёнными системами:} Рассматриваются аспекты управления данными, включая фрагментацию,
    дублирование и восстановление в распределённых базах данных (п. 4.8).
\end{itemize}

\paragraph{Недостатки:}
\begin{itemize}
    \item \textbf{Ограниченное внимание к семантике:} Семантика данных рассматривается лишь косвенно, через правила
    структурирования и моделирования данных (раздел 5).
    \item \textbf{Меньший акцент на пользовательском взаимодействии:} Отсутствует глубокий фокус на восприятии данных
    конечным пользователем.
\end{itemize}

\subsubsection*{ГОСТ Р 43.0.11-2014}

\paragraph{Достоинства:}
\begin{itemize}
    \item \textbf{Семантическая организация данных:} Делает акцент на перцептивное и грамматическое представление данных для упрощения восприятия и принятия решений (п. 5.8).
    \item \textbf{Удобство для операторов:} Создаёт базы данных, предназначенные для специалистов, обеспечивая удобство осмысления и взаимодействия (п. 1, область применения).
    \item \textbf{Поддержка знаний:} Рассматривает базы данных как основу для создания баз знаний и упрощения их применения в обучении и практике (приложение А) .
\end{itemize}

\paragraph{Недостатки:}
\begin{itemize}
    \item \textbf{Ограниченный охват распределённых систем:} Не рассматривает работу с распределёнными базами данных.
    \item \textbf{Независимость данных:} Проблема независимости данных напрямую не поднимается, акцент сделан на их семантической организации (раздел 5).
    \item \textbf{Меньший фокус на безопасности:} Управление доступом связано с восприятием данных, а не с предотвращением несанкционированного доступа (п. 5.2).
\end{itemize}

\subsubsection*{Вывод}
\begin{itemize}
    \item \textbf{ГОСТ 34.321-96:} Подходит для систем, где важны универсальность, согласованность данных и работа в распределённых системах.
    \item \textbf{ГОСТ Р 43.0.11-2014:} Лучше для областей, где важны удобство восприятия данных и их использование в практической деятельности.
\end{itemize}
Для рекомендательной системы предпочтителен \textbf{ГОСТ 34.321-96}, ориентированный на устойчивость и согласованность данных.

\subsection{Стандарты информационной безопасности и криптографической защиты информации}


\subsubsection*{ГОСТ Р 59162–2020}
\paragraph{Достоинства:}
\begin{itemize}
\item \textbf{Комплексный подход к безопасности беспроводных сетей:} Стандарт охватывает меры информационной безопасности, угрозы и рекомендации по защите беспроводных IP-сетей (раздел 1, область применения).
\item \textbf{Разделение на категории сетей:} Учитываются различия между WPAN, WLAN, WMAN и их специфические особенности безопасности (п. 6.1–6.3).
\item \textbf{Учет современных угроз:} Рассматриваются различные типы атак, включая отказ в обслуживании, анализ пакетов и атаки через Bluetooth (раздел 7).
\item \textbf{Рекомендации по проектированию систем:} Стандарт предоставляет методы проектирования, включая управление уязвимостями и использование безопасных конфигураций (раздел 9).
\end{itemize}

\paragraph{Недостатки:}
\begin{itemize}
\item \textbf{Фокус на беспроводные сети:} Ограниченная применимость к другим типам сетей, что делает стандарт менее универсальным для комплексных инфраструктур (раздел 1).
\item \textbf{Отсутствие подробных криптографических рекомендаций:} Не предоставляет детализированных алгоритмов, сосредотачиваясь на архитектурных аспектах (разделы 8, 9).
\end{itemize}

\subsubsection*{Р 50.1.115–2016}
\paragraph{Достоинства:}
\begin{itemize}
\item \textbf{Криптографическая основа:} Описывает протокол выработки общего ключа с использованием пароля и методами эллиптической криптографии (раздел 4).
\item \textbf{Универсальность:} Применим как для корпоративных, так и для общедоступных сетей, обеспечивая защиту каналов связи (раздел 1).
\item \textbf{Высокий уровень безопасности:} Использует современные криптографические подходы, такие как хэш-функции и электронная подпись (п. 4.1, 4.3).
\end{itemize}

\paragraph{Недостатки:}
\begin{itemize}
\item \textbf{Ограниченная область применения:} Сосредоточен на отдельных аспектах криптографии, не охватывая архитектурные и организационные аспекты безопасности (раздел 1).
\item \textbf{Сложность реализации:} Реализация алгоритмов требует высокой квалификации, что может усложнить их применение в малых организациях (п. 4.3, 5).
\end{itemize}

\subsubsection*{Вывод}
\begin{itemize}
\item \textbf{ГОСТ Р 59162–2020:} Лучше подходит для защиты беспроводных сетей, предоставляя рекомендации по архитектурным решениям и учету современных угроз.
\item \textbf{Р 50.1.115–2016:} Более уместен для обеспечения криптографической безопасности, особенно при необходимости выработки ключей и аутентификации.
\end{itemize}

Для приложения рекомендательной системы фильмов наиболее подходящим стандартом является \textbf{ГОСТ Р 50.1.115-2016}


\subsection{Производительность и качество ИТ-продуктов и систем}

\paragraph{ИСО/МЭК 9126-93}

Достоинства:

\begin{itemize}
	\item Стандарт предоставляет шесть основных характеристик
		качества программного обеспечения (функциональность, надежность,
		удобство использования, производительность,
		сопровождаемость, переносимость),
		что позволяет структурировать анализ качества.
	\item Каждая характеристика делится на подхарактеристики,
		что детализирует анализ (например,
		в функциональности --- корректность, интероперабельность, полнота).
	\item Предлагаются количественные и качественные показатели
		для измерения характеристик,
		что делает стандарт удобным для практического применения.
	\item Подходит для оценки качества различных типов программных продуктов.
\end{itemize}

Недостатки:

\begin{itemize}
	\item Стандарт больше сосредоточен на оценке готового программного продукта
		и не охватывает полный жизненный цикл разработки.
	\item Принят в 1993 году, поэтому многие принципы
		и методики не учитывают современных технологий
		и процессов разработки.
	\item Не предоставляет методологии оценки,
		как это сделано в ИСО/МЭК 25041-2014.
\end{itemize}

\paragraph{ИСО/МЭК 25041-2014}

Достоинства:

\begin{itemize}
	\item Описывает процесс оценки, начиная с определения целей
		и заканчивая анализом результатов, что делает его универсальным
		для всех этапов жизненного цикла программного обеспечения (пункт 5.1).
	\item Учитывает различия в подходах к оценке для разработчиков,
		приобретателей и независимых оценщиков,
		адаптируя процессы к их потребностям (пункт 5.3).
	\item Может использоваться для оценки как статических
		(исходный код, спецификации),
		так и динамических продуктов
		(тестируемый продукт, готовое ПО) (пункт 5.2).
	\item Поддерживает современную серию стандартов SQuaRE,
		включая ИСО/МЭК 25010, для оценки качества (пункт 5.1).
	\item Учитывает текущие технологии и методы оценки,
		такие как использование инструментов автоматического тестирования.
\end{itemize}

Недостатки:

\begin{itemize}
	\item Процесс оценки требует значительных ресурсов,
		особенно для крупных проектов,
		где необходимо документировать каждую стадию оценки.
	\item В меньшей степени ориентирован на характеристику качества ПО,
		больше фокусируется на процессах и методах оценки.
\end{itemize}

\paragraph{Итоги сравнения}

ИСО/МЭК 9126-93 подходит для оценки качества ПО в простых проектах,
где важна структурированная модель характеристик качества.
Он полезен для базового анализа готовых продуктов,
но требует адаптации для современных проектов.

ИСО/МЭК 25041-2014 более современный и ориентирован на процессы оценки,
что делает его предпочтительным для сложных проектов.
Он предоставляет детальную методологию,
адаптированную к разным ролям и этапам жизненного цикла.

Для систем,
где требуется комплексная оценка качества на всех этапах жизненного цикла,
предпочтительнее использовать ИСО/МЭК 25041-2014.

\subsection{Интероперабельность ИТ-продуктов и систем}

\paragraph{ГОСТ Р 55062 -- 2021}

Достоинства:

\begin{itemize}
	\item Представляет собой развитие семиуровневой эталонной модели ВОС
		для обеспечения интероперабельности.
		Это упрощает интеграцию различных
		систем благодаря проверенным решениям (пункт 5).
	\item Стандарт предлагает универсальный подход
		с четким описанием технического, семантического
		и организационного уровней взаимодействия (пункты 5.1-5.3),
		что помогает устранить барьеры на разных этапах интеграции.
	\item Указывает на использование признанных стандартов
		(например, TCP/IP для технического уровня
		и XML для семантического уровня),
		что обеспечивает совместимость
		и широкую применимость (пункты 5.1, 5.2).
	\item Включает как теоретические, так и экспериментальные этапы,
		позволяя адаптировать методику под конкретные проекты (пункт 6).
\end{itemize}

Недостатки:

\begin{itemize}
	\item Хотя стандарт применим к широкому классу систем,
		он не предоставляет детализированных рекомендаций
		для сложных или узкоспециализированных систем.
	\item Подходит для типовых систем,
		но не охватывает специфические характеристики многокомпонентных,
		взаимосвязанных систем.
\end{itemize}

\paragraph{ГОСТ Р 59797 -- 2021}

Достоинства:

\begin{itemize}
	\item Стандарт предназначен для оценки
		и обеспечения интероперабельности сложных систем,
		где каждая подсистема может
		быть самостоятельной системой (пункты 1.1 и 4.2).
		Это делает его полезным для крупномасштабных проектов.
	\item Использует принципы системной инженерии,
		что особенно важно при проектировании взаимосвязанных
		и многоуровневых систем (пункт 5.1).
	\item Предлагает трехмерную архитектурную модель,
		включающую функции, сервисы и множество подсистем,
		что помогает структурировать сложные проекты (пункт 5.3.2).
	\item Уделяет внимание взаимному влиянию характеристик подсистем,
		что повышает устойчивость и совместимость сложных систем (пункт 4.4).
\end{itemize}

Недостатки:

\begin{itemize}
	\item Реализация методики требует значительных ресурсов,
		что делает стандарт менее применимым для небольших проектов
		или систем с ограниченными задачами.
	\item Хотя стандарт охватывает технические и семантические аспекты,
		организационная интероперабельность
		не является его ключевым направлением.
\end{itemize}

\paragraph{Итоги сравнения}

ГОСТ Р 55062-2021 лучше подходит для типовых информационных систем
с универсальными требованиями к интероперабельности.
Он обеспечивает простую реализацию,
ориентированную на использование стандартных протоколов и методик,
что делает его предпочтительным
для проектов с ограниченным масштабом или ресурсами.

ГОСТ Р 59797-2021 предпочтителен для крупных,
многокомпонентных систем с разнородными подсистемами,
где важны детализированные подходы к обеспечению взаимодействия.
Стандарт лучше подходит для сложных систем,
таких как государственные или корпоративные платформы,
где важны интеграция и масштабируемость.

Для разработки системы конвертации DRC правил,
которая является частью экосистемы EDA (Electronic Design Automation),
более подходящим представляется ГОСТ Р 55062-2021,
так как он предлагает универсальный подход к интероперабельности,
достаточный для взаимодействия с различными инструментами проектирования.

\subsection{Документирование ПО}

\paragraph{ГОСТ 34.201 -- 2020}

Достоинства:

\begin{itemize}
	\item Применим к автоматизированным системам (АС)
		в различных сферах деятельности, включая управление,
		проектирование и исследования,
		что делает его широко применимым для большинства проектов ПО (пункт 1).
	\item Устанавливает четкие требования к видам,
		наименованию и комплектности документации
		на всех стадиях разработки (пункты 4.1-4.5).
		Это позволяет разработчику планировать полный объем документации
		для конкретного проекта.
	\item Допускает адаптацию структуры документов под специфику проекта,
		включая разбиение на части, расширение номенклатуры
		и использование групповых документов (пункт 3).
	\item Описывает организационно-распорядительные документы,
		такие как акты приемки, планы-графики и протоколы испытаний,
		что упрощает переход к эксплуатации (пункт 3).
\end{itemize}

Недостатки:

\begin{itemize}
	\item Не уделяет внимания семантической однозначности данных,
		что важно для сложных баз данных и баз знаний.
	\item Может быть избыточен для простых проектов ПО,
		не включающих элементы аппаратной части.
\end{itemize}

\paragraph{ГОСТ Р 59988.02.1 -- 2022}

Достоинства:

\begin{itemize}
	\item Предоставляет правила
		и рекомендации для обеспечения однозначности
		технических характеристик (ТХ) ЭКБ,
		что важно для баз данных и знаний (пункт 1.2).
	\item Регламентирует использование предпочтительных наименований,
		синонимов, единиц измерения и квалификаторов,
		что улучшает структурирование данных (пункт 4).
	\item Подробные указания по созданию спецификаций для ТХ ЭКБ,
		представленных в приложениях А и Б,
		помогают унифицировать данные (пункт 5.2).
\end{itemize}

Недостатки:

\begin{itemize}
	\item Фокусируется исключительно на характеристиках
		электронной компонентной базы,
		что ограничивает применение в других сферах.
	\item Меньший акцент на общие документы и этапы разработки ПО,
		такие как акты приемки или планы-графики.
\end{itemize}

\paragraph{ГОСТ Р 59988.03.1 -- 2022}

Достоинства:

\begin{itemize}
	\item Развивает ГОСТ Р 59988.02.1-2022,
		адаптируя его к "Приборам и модулям полупроводниковым"
		и предоставляя рекомендации для этого класса компонентов (пункт 4).
	\item Устанавливает стандарты классификации,
		измерений и типов данных для полупроводниковых приборов,
		что упрощает управление информацией (пункт 5.1).
	\item Содержит примеры спецификаций для практического применения,
		что облегчает внедрение (пункт 5.2).
\end{itemize}

Недостатки:

\begin{itemize}
	\item Применим только для систем,
		связанных с ЭКБ в области полупроводниковых приборов.
	\item Стандарт не подходит для проектов, не связанных с ЭКБ.
\end{itemize}

\paragraph{Итоги сравнения}

ГОСТ 34.201-2020 наиболее подходящий для общего документирования ПО.
Он универсален, гибок и охватывает весь жизненный цикл разработки.

ГОСТ Р 59988.02.1-2022 и ГОСТ Р 59988.03.1-2022 применимы только для систем,
связанных с разработкой баз данных и знаний по ЭКБ.
Эти стандарты лучше подходят для проектов,
связанных с техническими характеристиками компонентов,
а не для ПО общего назначения. 

ГОСТ 34.201-2020 более подходящий для использования
для разработки системы конвертации DRC правил.

\clearpage

\section*{\LARGE Вывод}
\addcontentsline{toc}{section}{Вывод}

В ходе выполнения задания были изучены различные группы стандартов,
связанные с темой проекта.
Было отмечено, что каждая группа стандартов включает
свои специфические аспекты,
такие как безопасность данных, точность конвертации правил,
взаимодействие инструментов проектирования,
разработка программного обеспечения и документирование.
Это позволило глубже понять требования и рекомендации,
которые необходимо учитывать при реализации подобных проектов.

