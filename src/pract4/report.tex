\section*{\LARGE Цель практической работы}
\addcontentsline{toc}{section}{Цель практической работы}

\textbf{Цель практической работы:}
Получить навыки по определению требуемых областей
стандартизации и по поиску страндартизирующих комитетов и
существующих стандартов, созданных данными комитетами. Получить
навыки анализа стандартов предметной области для выбора наиболее
подходящих из них на основании перечня, полученного при анализе
областей стандартизации.

\textbf{Задание на практическую работу:}

\begin{enumerate}
	\item Опираясь на подпроцессы выбранной области определить
		группы стандартов, которые могут быть применены
		для выбранного дипломного проекта.
		Минимально необходимо выбрать 5 групп стандартов.
	\item Определить, какие комитеты и подкомитеты занимаются
		стандартизацией областей.
	\item Опираясь на выбранные комитеты и подкомитеты,
		определить какие стандарты могут быть использованы
		для реализуемого проекта.
		Для каждой выделенной группы стандартов необходимо
		найти минимум 2 стандарта, пригодных для проекта.
	\item На основании выбранных стандартов провести исследование,
		какие из них наиболее уместно использовать в предлагаемой разработке.
		Обоснование должно заключаться в приведении конкретных достоинств
		и недостатков выбранных стандартов по сравнению с другими.
		При приведении конкретных недостатков или достоинств стандарта
		необходимо ссылаться на пункт,
		обуславливающий это достоинство или недостаток.
\end{enumerate}

\clearpage

\section*{\LARGE Выполнение практической работы}
\addcontentsline{toc}{section}{Выполнение практической работы}
\section{Группы стандартов}

Разрабатываемый программный продукт предполагает следующие группы стандартов:

\begin{itemize}
    \item Стандарты управления данными и взаимодействия информационных систем;
    \item Стандарты информационной безопасности и криптографической защиты информации;
    \item Стандарты проектирования пользовательских интерфейсов;
    \item Стандарты тестирования и обеспечения качества программного обеспечения;
    \item Стандарты интеграции и совместимости.
\end{itemize}

\subsection{Стандарты управления данными и взаимодействия информационных систем}

\section*{Общее описание}
Группа стандартов направлена на обеспечение эффективного управления данными и организацию корректного взаимодействия информационных систем. Стандарты регламентируют:
\begin{itemize}
    \item Основные подходы к проектированию и использованию систем управления данными;
    \item Принципы совместимости и интеграции различных информационных систем;
    \item Терминологическую основу для разработчиков и пользователей.
\end{itemize}

\section*{Основные положения}
\subsection*{Управление данными}
\begin{itemize}
    \item Организация хранения, обработки, поиска и восстановления данных.
    \item Определение структуры данных через схемы и модели, такие как реляционная и иерархическая.
    \item Обеспечение целостности, актуальности и единообразия данных.
\end{itemize}

\subsection*{Взаимодействие информационных систем}
\begin{itemize}
    \item Установление единых протоколов и интерфейсов для обмена данными.
    \item Реализация независимости от используемых технологий и платформ.
    \item Защита данных через управление доступом и идентификацию пользователей.
\end{itemize}

\subsection{Стандарты информационной безопасности и криптографической защиты информации}

\section*{Общее описание}
Группа стандартов направлена на обеспечение безопасности информационных систем и защиты данных. Стандарты регламентируют:
\begin{itemize}
    \item Основные подходы к проектированию и использованию систем безопасности;
    \item Принципы совместимости и интеграции различных информационных систем;
    \item Терминологическую основу для разработчиков и пользователей.
\end{itemize}

\section*{Основные положения}
\subsection*{Безопасность беспроводных сетей}
\begin{itemize}
    \item Определение угроз и требований к информационной безопасности.
    \item Предложение мер контроля и проектирования систем безопасности для беспроводных сетей.
\end{itemize}

\subsection*{Криптографическая защита информации}
\begin{itemize}
    \item Описание протоколов выработки ключей и взаимной аутентификации.
    \item Применение в корпоративных и общедоступных сетях.
\end{itemize}

\section*{Цель и задачи группы стандартов}
\begin{itemize}
    \item Обеспечение конфиденциальности, целостности и доступности информации в информационных системах.
    \item Разработка рекомендаций по проектированию и реализации мер безопасности в сетях.
    \item Использование современных криптографических методов для защиты данных.
\end{itemize}

\section*{Примечания}
Данная группа стандартов предназначена для специалистов, занимающихся информационной безопасностью, включая:
\begin{itemize}
    \item Разработчиков и проектировщиков систем безопасности.
    \item Сетевых администраторов и инженеров.
    \item Сотрудников, ответственных за соблюдение нормативных требований в области безопасности.
\end{itemize}

\subsection{Стандарты проектирования пользовательских интерфейсов}

Цель: Гарантировать, что взаимодействие пользователя с приложением будет удобным, интуитивным и эстетически привлекательным.
\section*{Основные аспекты}
\begin{itemize}
    \item Принципы эргономики: Интерфейс должен быть адаптирован под широкий спектр пользователей, включая людей с ограниченными возможностями.
    \item Интуитивная навигация: Логичная структура интерфейса, позволяющая пользователям легко находить нужные функции.
    \item Многоуровневый дизайн: Создание интерфейсов, одинаково удобных для новичков и опытных пользователей.
    \item Адаптивность: Поддержка различных устройств и размеров экранов (мобильные телефоны, планшеты, компьютеры).
    \item Доступность: Учет принципов инклюзии, таких как поддержка специальных экранных считывателей и оптимизированная контрастность.
\end{itemize}

\subsection{Стандарты интеграции и совместимости}

Цель: Обеспечить безошибочное взаимодействие между всеми компонентами системы и их совместимость с внешними сервисами.
\section*{Основные аспекты}
\begin{itemize}
    \item Функциональное тестирование: Проверка работы всех модулей приложения, включая правильность рекомендаций, взаимодействие с базой данных и API.
    \item Нагрузочное тестирование: Анализ производительности системы при увеличении числа пользователей, обработке большого объема запросов или данных.
    \item Тестирование безопасности: Защита пользовательских данных от утечек, защита от SQL-инъекций, защита от несанкционированного доступа.
    \item Автоматизация тестирования: Использование специализированных инструментов для регулярного и быстрого тестирования кода.
    \item Оценка пользовательского опыта: Проведение тестирования удобства использования интерфейса, времени отклика и общего восприятия системы пользователями.
\end{itemize}

\subsection{Стандарты тестирования и обеспечения качества программного обеспечения}

Цель: Обеспечить работоспособность системы в любых эксплуатационных условиях и предотвратить сбои, которые могут нарушить пользовательский опыт.
\section*{Основные аспекты}
Данная группа стандартов предназначена для специалистов, занимающихся информационной безопасностью, включая:
\begin{itemize}
    \item Интеграция компонентов: Мобильное приложение, API и база данных должны быть связаны через четко определённые интерфейсы.
    \item Поддержка стандартных форматов данных: Использование общепринятых форматов (например, JSON или XML) для обмена данными между компонентами и внешними системами.
    \item Масштабируемость: Обеспечение возможности добавления новых функций или подключения дополнительных сервисов без изменения существующей архитектуры.
    \item Совместимость с внешними API: Возможность подключения сторонних сервисов (например, системы рекомендаций, рейтингов или платформ потокового видео).
    \item Обеспечение отказоустойчивости: Минимизация рисков сбоев в интеграции, которые могут нарушить работу системы.
\end{itemize}
\clearpage

\section{Комитеты и подкомитеты}

Технический комитет (ТК) по стандартизации --- это объединение специалистов,
являющихся полномочными представителями заинтересованных предприятий
(организаций) --- членов ТК,
создаваемое на добровольной основе для разработки национальных стандартов РФ,
проведения работ в области международной (региональной)
стандартизации по закрепленным за ТК объектам стандартизации
(областям деятельности).

\begin{longtable}{|p{2cm}|p{14cm}|}
    \caption{Комитеты} \label{table:tk} \\
    \hline
    \textbf{\No\ ТК}
    & \textbf{Наименование ТК} \\
    \hline
    \endfirsthead
    \conttable{table:tk} \\
    \hline
    \textbf{\No\ ТК}
    & \textbf{Наименование ТК} \\
    \hline
    \endhead
    \textbf{022} & Информационные технологии \\ \hline
    \textbf{026} & Криптографическая защита информации \\ \hline
    \textbf{379} & Информационное обеспечение техники и операторской деятельности \\ \hline
    \textbf{201} & Эргономика, психология труда и инженерная психология \\ \hline

\end{longtable}

\subsection{Подкомитеты ТК 22 Информационные технологии}

\begin{longtable}{|p{2cm}|p{14cm}|}
    \caption{Подкомитеты ТК 22 Информационные технологии}
    \label{table:tk:22} \\
    \hline
    \textbf{\No\ ПК}
    & \textbf{Наименование ПК} \\
    \hline
    \endfirsthead
    \conttable{table:tk:22} \\
    \hline
    \textbf{\No\ ПК}
    & \textbf{Наименование ПК} \\
    \hline
    \endhead
    ПК107 (SC7) & Системная и программная инженерия \\ \hline
    ПК122 (SC22)
    & Языки программирования, их окружение
    и системы программных интерфейсов \\ \hline
    ПК125 (SC25)
    & Взаимосвязь оборудования для информационных технологий \\ \hline
    ПК127 (SC27) & Безопасность информационных технологий \\ \hline
    ПК128 (SC28) & Оборудование офисов \\ \hline
    ПК134 (SC34) & Описание документа и языки обработки \\ \hline
    ПК135 (SC35) & Пользовательские интерфейсы \\ \hline
    ПК138 (SC38) & Платформы и сервисы для распределенных приложений \\ \hline
    ПК140 (SC40)
    & Управление информационными технологиями и услугами ИТ \\ \hline
    ПК201 & Терминология в ИТ \\ \hline
    ПК206 & Интероперабельность \\ \hline
\end{longtable}


\begin{longtable}{|p{2cm}|p{14cm}|}
    \caption{Подкомитеты ТК 26 Криптографическая защита информации}
    \label{table:tk:26} \\
    \hline
    \textbf{\No\ ПК}
    & \textbf{Наименование ПК} \\
    \hline
    \endfirsthead
    \conttable{table:tk:26} \\
    \hline
    \textbf{\No\ ПК}
    & \textbf{Наименование ПК} \\
    \hline
    \endhead
    ПК1 & Криптографические алгоритмы и протоколы для применения в поставляемых для федеральных государственных
    нужд шифровальных (криптографических) средствах защиты информации, содержащей сведения, составляющие государственную тайну \\ \hline
    ПК2
    & Криптографические алгоритмы и протоколы для применения в поставляемых для федеральных государственных нужд
    шифровальных (криптографических) средствах защиты информации, содержащей сведения, относимые к охраняемой в
    соответствии с законодательством Российской Федерации информации ограниченного доступа \\ \hline
    ПК3
    & Криптографические алгоритмы и механизмы в национальной платежной системе Российской Федерации \\ \hline
    ПК4 & Российские шифровальные (криптографические) средства защиты информации, не содержащей сведений, составляющих
    государственную тайну, или относимых к охраняемой в соответствии с законодательством Российской Федерации к
    информации ограниченного доступа, а также зарубежные шифровальные (криптографические) средства защиты информации
    на территории Российской Федерации \\ \hline
\end{longtable}


\begin{longtable}{|p{2cm}|p{14cm}|}
    \caption{Подкомитеты ТК 100 Стратегический и инновационный менеджмент}
    \label{table:tk:100} \\
    \hline
    \textbf{\No\ ПК}
    & \textbf{Наименование ПК} \\
    \hline
    \endfirsthead
    \conttable{table:tk:26} \\
    \hline
    \textbf{\No\ ПК}
    & \textbf{Наименование ПК} \\
    \hline
    \endhead
    ПК1 & Инновационный менеджмент \\ \hline
    ПК2 & Архитектура и интеграция \\ \hline
    ПК3 & Менеджмент проектов \\ \hline
    ПК4 & Электронные библиотеки данных \\ \hline
    ПК5 & Менеджмент знаний, защита результатов интеллектуальной деятельности \\ \hline
\end{longtable}

\clearpage

\section{Выбранные стандарты}


\subsection{Стандарты управления данными и взаимодействия информационных систем}
\subsubsection{ГОСТ 34.321 -- 96}
\paragraph{ГОСТ 34.321 -- 96}
\emph{\href{https://internet-law.ru/gosts/gost/6808/}{ГОСТ 34.321 -- 96}
Информационные технологии.
Система стандартов по базам данных.
Эталонная модель управления данными
}
\par
Комитет: TK 022 Информационные технологии
\section*{Общее описание}
ГОСТ 34.321-96 устанавливает эталонную модель управления данными в информационных системах.
Стандарт определяет терминологию, основные концепты и требования, связанные с управлением базами данных.
Эталонная модель служит для описания услуг, предоставляемых системами управления базами данных (СУБД) и системами словарей данных.

\section*{Область применения}
Стандарт распространяется на процессы управления постоянными данными, включая их хранение, поиск, обновление, ввод, копирование, восстановление и передачу. Он ориентирован на:
\begin{itemize}
    \item Определение общих принципов управления данными;
    \item Обеспечение совместимости между различными системами;
    \item Создание единой терминологической основы для разработчиков и пользователей.
\end{itemize}

\section*{Термины и определения}
В рамках стандарта введены следующие ключевые термины:
\begin{itemize}
    \item \textbf{База данных (database):} Это совокупность взаимосвязанных данных, организованных в соответствии с определённой схемой, для обеспечения удобного доступа и обработки.
    \item \textbf{Схема базы данных (database schema):} Это формальное описание содержания, структуры и ограничений целостности базы данных.
    \item \textbf{Транзакция (transaction):} Совокупность операций, которая характеризуется такими свойствами, как атомарность, согласованность, изоляция и долговечность (ACID-свойства).
    \item \textbf{Управление доступом (access control):} Это процесс предотвращения несанкционированного доступа к данным путём использования различных механизмов идентификации и авторизации.
\end{itemize}

\section*{Основные компоненты модели}

\subsection*{База данных и схема}
Основным элементом модели является база данных, структура и содержание которой определяются её схемой. Схема включает описание элементов данных, их связей и правил целостности. Такое представление обеспечивает логическую организацию данных и их доступность для пользователей.

\subsection*{Средства моделирования данных}
Для создания схем базы данных применяются средства моделирования данных. Эти средства включают:
\begin{itemize}
    \item \textbf{Правила структурирования данных:} Определяют, как данные организуются и связываются друг с другом.
    \item \textbf{Правила манипулирования данными:} Определяют допустимые операции над данными.
\end{itemize}
Наиболее распространёнными моделями данных являются реляционная, сетевая и иерархическая модели.

\subsection*{Независимость данных}
Независимость данных обеспечивается следующими подходами:
\begin{itemize}
    \item Процессы взаимодействуют только с необходимыми частями схемы базы данных.
    \item Прикладные процессы отделены от физического представления данных.
    \item Ограничения целостности реализуются непосредственно в схеме базы данных.
\end{itemize}

\subsection*{Процессоры и интерфейсы}
Процессы управления данными выполняются с помощью процессоров, которые предоставляют интерфейсы для доступа к функционалу системы. Эти интерфейсы могут быть универсальными и независимыми от языков программирования.

\section*{Управление доступом}
Управление доступом в эталонной модели основывается на следующих принципах:
\begin{itemize}
    \item \textbf{Определение и модификация привилегий пользователей:} Позволяет устанавливать уровни доступа для различных категорий пользователей.
    \item \textbf{Реализация санкционированного доступа:} Осуществляется через процедуры идентификации и аутентификации пользователей.
    \item \textbf{Сохранение данных об управлении доступом:} Вся информация о привилегиях и ограничениях хранится непосредственно в базе данных.
\end{itemize}

\section*{Транзакции базы данных}
Транзакции базы данных представляют собой логические единицы работы, обеспечивающие надёжное выполнение операций. Основные требования к транзакциям включают:
\begin{itemize}
    \item \textbf{Атомарность:} Транзакция либо выполняется полностью, либо не выполняется вовсе.
    \item \textbf{Согласованность:} После выполнения транзакции данные остаются в непротиворечивом состоянии.
    \item \textbf{Изоляция:} Параллельно выполняемые транзакции не влияют на корректность данных.
    \item \textbf{Долговечность:} Изменения, произведённые транзакцией, сохраняются даже в случае сбоя системы.
\end{itemize}

\section*{Распределённые базы данных}
Распределённые базы данных организуют хранение данных на нескольких компьютерах, что обеспечивает масштабируемость и надёжность системы. Основные требования включают:
\begin{itemize}
    \item \textbf{Управление фрагментацией и дублированием данных:} Гарантирует оптимальное распределение данных между узлами.
    \item \textbf{Независимость распределения:} Пользователи не должны знать о физическом расположении данных.
    \item \textbf{Восстановление:} Обеспечивает целостность данных при сбоях.
\end{itemize}

\section*{Восстановление и журналирование}
Для обеспечения целостности и надёжности данных применяются следующие методы:
\begin{itemize}
    \item \textbf{Контрольные журналы:} Фиксируют изменения данных для возможности их последующего анализа и восстановления.
    \item \textbf{Резервное копирование:} Позволяет восстанавливать данные до состояния, предшествующего сбою.
    \item \textbf{Реорганизация физической памяти:} Выполняется для оптимизации хранения данных при изменениях структуры базы.
\end{itemize}

\subsubsection{ГОСТ Р 43.0.11 -- 2014}
\paragraph{ГОСТ Р 43.0.11 -- 2014}
\emph{\href{https://internet-law.ru/gosts/gost/57862/}{ГОСТ Р 43.0.11 -- 2014}
Информационное обеспечение техники и операторской деятельности.
Базы данных в технической деятельности
}
\par
ТК 379 «Информационное обеспечение
техники и операторской деятельности»
\section*{Общее описание}
ГОСТ Р 43.0.11-2014 устанавливает основные требования и принципы обеспечения совместимости информационных систем, определяя правила и процедуры для их взаимодействия. Стандарт ориентирован на повышение эффективности информационного обмена между различными системами.

\section*{Область применения}
Стандарт применяется к информационным системам, функционирующим в различных отраслях. Основные цели использования:
\begin{itemize}
    \item Определение правил совместимости компонентов информационных систем.
    \item Обеспечение корректного обмена данными между разнородными системами.
    \item Повышение надёжности и устойчивости систем к сбоям.
\end{itemize}

\section*{Термины и определения}
ГОСТ Р 43.0.11-2014 включает следующие ключевые термины:
\begin{itemize}
    \item \textbf{Информационная система (ИС):} Совокупность программных, аппаратных и организационных средств, предназначенных для обработки данных.
    \item \textbf{Совместимость:} Способность компонентов системы взаимодействовать для достижения заданной функциональности.
    \item \textbf{Интерфейс:} Средство взаимодействия между компонентами системы или между системами.
    \item \textbf{Протокол:} Набор правил для обмена данными между взаимодействующими компонентами.
\end{itemize}

\section*{Основные положения стандарта}

\subsection*{Совместимость систем}
Стандарт определяет совместимость как обязательное свойство информационных систем, достигаемое путём унификации процедур взаимодействия. Это включает:
\begin{itemize}
    \item Установление общих требований к интерфейсам.
    \item Определение единых форматов данных.
    \item Разработку протоколов обмена данными.
\end{itemize}

\subsection*{Модели взаимодействия}
ГОСТ Р 43.0.11-2014 предлагает модели взаимодействия систем, направленные на обеспечение их совместимости. Основные модели:
\begin{itemize}
    \item \textbf{Модель клиент-сервер:} Одна система (клиент) запрашивает данные или услуги у другой системы (сервер).
    \item \textbf{Модель однорангового взаимодействия:} Системы функционируют на равных правах, обмениваясь данными без центрального управляющего звена.
    \item \textbf{Модель шина данных:} Интеграция через централизованный канал, обеспечивающий передачу данных между системами.
\end{itemize}

\subsection*{Интерфейсы и протоколы}
Для обеспечения корректного взаимодействия между компонентами стандартизированы:
\begin{itemize}
    \item \textbf{Интерфейсы:} Должны быть открытыми и документированными, чтобы исключить проблемы интеграции.
    \item \textbf{Протоколы:} Определяют формат и последовательность действий для передачи данных.
\end{itemize}

\subsection*{Управление доступом и безопасностью}
Для защиты данных и обеспечения их конфиденциальности в стандарте предусматриваются следующие меры:
\begin{itemize}
    \item Использование механизмов идентификации и аутентификации пользователей.
    \item Контроль доступа к ресурсам системы на основе заданных политик.
    \item Шифрование данных при их передаче между системами.
\end{itemize}

\section*{Требования к данным}
ГОСТ Р 43.0.11-2014 предъявляет строгие требования к качеству и формату данных:
\begin{itemize}
    \item \textbf{Целостность данных:} Обеспечивается за счёт механизмов контроля ошибок и резервирования.
    \item \textbf{Актуальность:} Данные должны своевременно обновляться для обеспечения достоверности.
    \item \textbf{Единообразие:} Форматы данных и кодировки должны быть согласованы между системами.
\end{itemize}

\section*{Обеспечение надёжности}
Для обеспечения надёжной работы систем стандарт предусматривает:
\begin{itemize}
    \item \textbf{Дублирование ключевых компонентов:} Уменьшает риск отказа системы.
    \item \textbf{Резервное копирование данных:} Позволяет восстановить систему в случае сбоя.
    \item \textbf{Мониторинг и диагностика:} Обеспечивают контроль за состоянием системы и выявление неисправностей.
\end{itemize}

\section*{Применение стандарта}
ГОСТ Р 43.0.11-2014 рекомендуется использовать при разработке новых и модернизации существующих информационных систем. Его применение способствует:
\begin{itemize}
    \item Повышению уровня совместимости и интеграции между системами.
    \item Уменьшению затрат на разработку и сопровождение систем.
    \item Созданию надёжной и безопасной инфраструктуры для обработки данных.
\end{itemize}
\subsection{Стандарты информационной безопасности и криптографической защиты информации}

\subsubsection{ГОСТ Р 50.1.115 -- 2016}
\paragraph{ГОСТ Р 50.1.115 -- 2016}
\emph{\href{https://rst.gov.ru:8443/file-service/file/load/1699435168541}{Р 50.1.115 -- 2016}
Информационные технологии.
Криптографическая защита информации.
Протокол выработки общего ключа с аутентификацией на основе пароля}

\section*{Общее описание}
Настоящий ГОСТ описывает протокол выработки общего ключа с использованием пароля для аутентификации сторон. Протокол основан на эллиптических кривых и обеспечивает защиту от активного противника.

\section*{Область применения}
Стандарт применяется в системах с использованием криптографии для защиты данных, включая электронные цифровые подписи и хэширование. Основное назначение — установление защищенного соединения между сторонами.

\section*{Основные положения}
\begin{itemize}
    \item Использование эллиптических кривых для формирования ключей.
    \item Поддержка строгой аутентификации и защиты данных.
    \item Обеспечение защиты каналов связи от перехвата.
\end{itemize}

\subsubsection{ГОСТ Р 59162 -- 2020}
\paragraph{ГОСТ Р 59162 -- 2020}
\emph{\href{https://rst.gov.ru:8443/file-service/file/load/1699366818935}{ГОСТ Р 59162 -- 2020}
Информационные технологии.
Методы и средства обеспечения безопасности.
Безопасность сетей. Часть 6: Обеспечение информационной безопасности при использовании беспроводных IP-сетей.}

\section*{Общее описание}
Стандарт определяет методы и меры обеспечения информационной безопасности для беспроводных IP-сетей. Он основан на международных стандартах ISO/IEC 27033-6.

\section*{Область применения}
Применяется для проектирования, реализации и мониторинга систем безопасности беспроводных сетей в различных организациях.

\section*{Основные положения}
\begin{itemize}
    \item Определение угроз и рисков для беспроводных сетей.
    \item Реализация механизмов шифрования, аутентификации и контроля доступа.
    \item Поддержание устойчивости к атакам и обеспечение конфиденциальности данных.
\end{itemize}

\subsection{Стандарты проектирования пользовательских интерфейсов}

\subsubsection{ГОСТ Р ИСО 9241-210 -- 2012}

ГОСТ Р ИСО 9241-210 -- 2012
\emph{\href{https://meganorm.ru/Data/534/53476.pdf}{ГОСТ Р ИСО 9241-210 -- 2012}
ЭРГОНОМИКА ВЗАИМОДЕЙСТВИЯ
ЧЕЛОВЕК—СИСТЕМА
Часть 210
Человеко-ориентированное проектирование
интерактивных систем
}
\par
Коммитет:  ТК 201 «Эргономика, психология труда и инженерная психология»


\subsection*{Область применения}
Настоящий стандарт содержит рекомендации для специалистов, осуществляющих разработку интерактивных систем, включая программные и аппаратные
Рассматриваются системы различных масштабов: от потребительских продуктов (например, мобильных приложений) до крупных автоматизированных систем (раздел 1,).

\subsection*{Логическое обоснование человеко-ориентированного подхода}
Применение человеко-ориентированного подхода способствует увеличению производительности пользователей, снижению расходов
на обучение, повышению доступности систем для различных категорий пользователей, а также снижению риска неблагоприятного влияния на здоровье.
Примеры выгод от применения подхода приведены в таблице 1 стандарта.

\subsection*{Принципы человеко-ориентированного проектирования}
Стандарт определяет шесть ключевых принципов (п4.1):
\begin{enumerate}
    \item Точное определение пользователей, задач и среды их работы.
    \item Вовлечение пользователей в проектирование.
    \item Оценка проекта с участием пользователей.
    \item Итеративное совершенствование проекта.
    \item Учет опыта пользователей.
    \item Междисциплинарный подход к проектированию.
\end{enumerate}

\subsection*{Планирование человеко-ориентированного проектирования}
Человеко-ориентированный подход должен быть интегрирован во все этапы жизненного цикла продукта:
от концепции до вывода из эксплуатации.
Планирование включает распределение ресурсов, определение методов, привлечение специалистов и организацию итераций (п5).

\subsection*{Процедура выполнения человеко-ориентированного проекта}
Проектирование включает четыре основные этапа:
\begin{itemize}
    \item Анализ условий использования (6.2.1).
    \item Определение требований пользователей.
    \item Разработка проектных решений.
    \item Оценка и совершенствование решений.
\end{itemize}
Каждый этап сопровождается описанием методов и рекомендаций для обеспечения соответствия требованиям пользователей.

\subsection*{Оценка и мониторинг проекта}
Стандарт предписывает использование оценки проекта на всех этапах его реализации.
Это может включать испытания с участием пользователей, проверку выполнения требований к удобству использования
и долгосрочный мониторинг эксплуатации системы (6.5).

\subsection*{Устойчивое развитие}
Человеко-ориентированное проектирование способствует решению задач устойчивого развития за счет улучшения экономических,
социальных и экологических характеристик систем (п7).

\section*{Заключение}
ГОСТ Р ИСО 9241-210–2012 предоставляет универсальные рекомендации, применимые к широкому спектру систем.
Принципы человеко-ориентированного проектирования способствуют созданию систем, удовлетворяющих требованиям
пользователей, улучшают взаимодействие и повышают качество жизни пользователей.



\subsubsection{ГОСТ Р ИСО 9241-151 -- 2014}

\emph{\href{https://meganorm.ru/Data2/1/4293768/4293768927.pdf}{ГОСТ Р ИСО 9241-151 -- 2014}
ЭРГОНОМИКА ВЗАИМОДЕЙСТВИЯ
ЧЕЛОВЕК — СИСТЕМА Часть 151
Руководство по проектированию пользовательских
интерфейсов сети Интернет
}
Коммитет:  ТК 201 «Эргономика, психология труда и инженерная психология»
\subsection*{Область применения}
Стандарт охватывает:
\begin{itemize}
    \item Проектирование архитектуры и стратегии (раздел 6);
    \item Разработка контента (раздел 7);
    \item Навигацию и поиск (раздел 8);
    \item Представление информационного наполнения (раздел 9).
\end{itemize}

\subsection*{Основные рекомендации}
\begin{itemize}
    \item Проектирование навигационной структуры (пункт 8.1);
    \item Адаптация контента для различных устройств (пункт 7.2.2);
    \item Учет потребностей пользователей с ограниченными возможностями (раздел 7.2.9).
\end{itemize}

\subsection*{Выводы}
ГОСТ Р ИСО 9241-151-2014 полезен для разработки пользовательских веб-интерфейсов, особенно в контексте обеспечения доступности и удобства навигации.

\subsection{Стандарты тестирования и обеспечения качества программного обеспечения}

\subsubsection{ГОСТ Р 56920 -- 2024}

\emph{\href{https://allgosts.ru/35/080/gost_r_56920-2024.pdf}{ГОСТ Р 56920 -- 2024}
Системная и программная инженерия. Тестирование программного обеспечения. Общие положения
}

\par
Комитет: TK 022 Информационные технологии

\subsection*{Область применения}
Настоящий стандарт содержит рекомендации для специалистов, осуществляющих разработку интерактивных систем, включая программные и аппаратные
Рассматриваются системы различных масштабов: от потребительских продуктов (например, мобильных приложений) до крупных автоматизированных систем (раздел 1,).

\subsection*{Логическое обоснование человеко-ориентированного подхода}
Применение человеко-ориентированного подхода способствует увеличению производительности пользователей, снижению расходов
на обучение, повышению доступности систем для различных категорий пользователей, а также снижению риска неблагоприятного влияния на здоровье.
Примеры выгод от применения подхода приведены в таблице 1 стандарта.

\subsection*{Принципы человеко-ориентированного проектирования}
Стандарт определяет шесть ключевых принципов (п4.1):
\begin{enumerate}
    \item Точное определение пользователей, задач и среды их работы.
    \item Вовлечение пользователей в проектирование.
    \item Оценка проекта с участием пользователей.
    \item Итеративное совершенствование проекта.
    \item Учет опыта пользователей.
    \item Междисциплинарный подход к проектированию.
\end{enumerate}

\subsection*{Планирование человеко-ориентированного проектирования}
Человеко-ориентированный подход должен быть интегрирован во все этапы жизненного цикла продукта:
от концепции до вывода из эксплуатации.
Планирование включает распределение ресурсов, определение методов, привлечение специалистов и организацию итераций (п5).

\subsection*{Процедура выполнения человеко-ориентированного проекта}
Проектирование включает четыре основные этапа:
\begin{itemize}
    \item Анализ условий использования (6.2.1).
    \item Определение требований пользователей.
    \item Разработка проектных решений.
    \item Оценка и совершенствование решений.
\end{itemize}
Каждый этап сопровождается описанием методов и рекомендаций для обеспечения соответствия требованиям пользователей.

\subsection*{Оценка и мониторинг проекта}
Стандарт предписывает использование оценки проекта на всех этапах его реализации.
Это может включать испытания с участием пользователей, проверку выполнения требований к удобству использования
и долгосрочный мониторинг эксплуатации системы (6.5).

\subsection*{Устойчивое развитие}
Человеко-ориентированное проектирование способствует решению задач устойчивого развития за счет улучшения экономических,
социальных и экологических характеристик систем (п7).

\section*{Заключение}
ГОСТ Р ИСО 9241-210–2012 предоставляет универсальные рекомендации, применимые к широкому спектру систем.
Принципы человеко-ориентированного проектирования способствуют созданию систем, удовлетворяющих требованиям
пользователей, улучшают взаимодействие и повышают качество жизни пользователей.



\subsubsection{ГОСТ Р ИСОМЭК 25051 -- 2017}

\emph{\href{https://meganorm.ru/Data/645/64532.pdf}{ГОСТ Р ИСОМЭК 25051 -- 2017}}
\par
Комитет: TK 022 Информационные технологии

устанавливает требования к качеству готового к использованию программного продукта (RUSP) и
инструкции по тестированию.
Стандарт направлен на обеспечение пользователей и разработчиков уверенности в соответствии программного обеспечения их
потребностям.
В нём представлены требования к описанию, документации и тестированию RUSP, а также рекомендации для оценки соответствия.

\subsection*{Область применения}
Стандарт применяется к программным продуктам, которые доступны для конечного пользователя без дополнительной разработки.
Он охватывает различные категории программного обеспечения, включая текстовые редакторы,
базы данных и приложения для смартфонов (\textit{раздел 1}).

\subsection*{Термины и определения}
Включены ключевые термины, такие как:
\begin{itemize}
    \item \textbf{RUSP} — готовый к использованию программный продукт, доступный пользователю без дополнительных доработок (\textit{п. 4.1.6});
    \item \textbf{Документация пользователя} — материалы, необходимые для правильного использования программного продукта (\textit{п. 4.1.26});
    \item \textbf{Оценка соответствия} — систематический анализ соответствия продукта требованиям (\textit{п. 4.1.4}).
\end{itemize}

\subsection*{Основные требования к RUSP}
\subsubsection*{Требования к описанию продукта}
Описание должно быть доступным для пользователей и включать:
\begin{itemize}
    \item Информацию о качественных характеристиках программного обеспечения (\textit{п. 5.1.2.1});
    \item Уникальный идентификатор продукта и данные о поставщике (\textit{п. 5.1.3.1-5.1.3.3});
    \item Сообщения о функциональной пригодности, надежности, безопасности и других аспектах качества (\textit{п. 5.1.5.1-5.1.10.1}).
\end{itemize}

\subsubsection*{Требования к документации пользователя}
Документация должна быть:
\begin{itemize}
    \item Доступной для пользователей (\textit{п. 5.2.1.1});
    \item Полной, включая описание всех функций и инструкций по их использованию (\textit{п. 5.2.4.1-5.2.4.4});
    \item Понятной и последовательной (\textit{п. 5.2.7.1-5.2.7.2}).
\end{itemize}

\subsubsection*{Требования к тестированию}
Включают:
\begin{itemize}
    \item План тестирования, который описывает подход, контрольные примеры и критерии прохождения (\textit{п. 6.2.1});
    \item Процедуры тестирования, включающие подготовку, выполнение и анализ результатов (\textit{п. 6.3.2});
    \item Результаты тестирования с указанием всех выявленных отклонений и предложений по их устранению (\textit{п. 6.4.1}).
\end{itemize}

\subsection*{Рекомендации по оценке соответствия}
Раздел 7 и приложение А предоставляют инструкции и рекомендации для проверки соответствия программного продукта установленным требованиям.
Они охватывают подготовку, процесс оценки и оформление результатов (\textit{п. 7.1-7.6}).

\section*{Вывод}
ГОСТ Р ИСО/МЭК 25051-2017 является ключевым стандартом для оценки качества программных продуктов.
Его применение позволяет поставщикам обеспечивать соответствие программного обеспечения требованиям пользователей,
а пользователям — получать уверенность в качестве приобретаемых решений.

\subsection{Стандарты тестирования и обеспечения качества программного обеспечения}

\subsubsection{ГОСТ Р 57136 -- 2016}

\emph{\href{https://meganorm.ru/Data2/1/4293751/4293751436.pdf}{ГОСТ Р 57136 -- 2016}
Системы промышленной автоматизации
и интеграция
ПОДХОД К ИНТЕГРАЦИИ ПРИЛОЖЕНИЙ
С ИСПОЛЬЗОВАНИЕМ МОДЕЛИРОВАНИЯ
ТРЕБОВАНИЙ К ОБМЕНУ ИНФОРМАЦИЕЙ
И ПРОФИЛИРОВАНИЯ ФУНКЦИОНАЛЬНЫХ
ВОЗМОЖНОСТЕЙ ПРОГРАММНОГО
ОБЕСПЕЧЕНИЯ
}
\par
Комитет: ТК 100 «Стратегический и инновационный менеджмент»

ГОСТ Р 57136-2016, идентичный международному стандарту ISO/TR 18161:2013, описывает подход к интеграции приложений
с использованием моделирования требований к обмену информацией и профилирования функциональных возможностей программного обеспечения.


\subsection*{Область применения}
ГОСТ применим для систем промышленной автоматизации и интеграции, где требуется стандартизированный подход к обмену информацией между приложениями. Он включает:
\begin{itemize}
    \item общие принципы интеграции (пункт 5.1);
    \item требования к функциональным возможностям программного обеспечения для интеграции (раздел 5);
    \item примеры и рекомендации для реализации профилей интеграции (раздел 6).
\end{itemize}

\subsection*{Основные элементы стандарта}
\subsubsection*{Моделирование требований к обмену информацией}
ГОСТ описывает, как определить и структурировать требования к обмену данными, чтобы обеспечить точность и надежность передачи информации между приложениями (пункт 5.2).

\subsubsection*{Профилирование функциональных возможностей}
Функциональные профили позволяют стандартизировать наборы требований к системам и упрощают процесс интеграции,
обеспечивая совместимость на уровне программного обеспечения (пункт 5.3).

\subsubsection*{Практические рекомендации}
В приложениях стандарта приведены примеры моделирования требований и создания функциональных профилей для конкретных типов интеграционных задач (раздел 6).

\subsection*{Вывод}
ГОСТ Р 57136-2016 является основополагающим документом для разработки интеграционных решений,
где важны стандартизированный обмен информацией и совместимость систем.
Его применение позволяет повысить эффективность разработки и интеграции сложных систем в промышленной автоматизации.



\subsubsection{ГОСТ Р 58538 -- 2019}

\emph{\href{https://meganorm.ru/Data/718/71869.pdf}{ГОСТ Р 58538 -- 2019}}
\par
Комитет: ТК 100 «Стратегический и инновационный менеджмент»


\subsection*{Общее описание}
ГОСТ Р 58538-2019 устанавливает спецификацию требований к организации информационного взаимодействия в системах промышленной
автоматизации и интеграции.
Основной целью стандарта является достижение функциональной совместимости устройств и приложений на основе единой методологии и этапов интеграции.

\subsection*{Область применения (раздел 1)}
Настоящий стандарт применим для обеспечения взаимодействия между системами и устройствами различного происхождения.
Он охватывает требования к функциональной совместимости на различных уровнях и описывает этапы обнаружения, конфигурирования, эксплуатации и управления устройствами.

\subsection*{Ключевые термины и определения (раздел 3)}
В стандарте определены такие ключевые термины, как:
\begin{itemize}
    \item \textbf{Функциональная совместимость (п. 3.3.2):} Способность систем обмениваться информацией и использовать её для совместной работы.
    \item \textbf{Обнаружение (п. 3.2.5):} Процесс, позволяющий системам находить новые элементы и определять их функциональность.
    \item \textbf{Конфигурирование (п. 3.2.4):} Установление связей между объектами для их взаимодействия.
\end{itemize}

\subsection*{Принципы функциональной совместимости (раздел 4)}
Стандарт определяет четыре этапа взаимодействия:
\begin{enumerate}
    \item \textbf{Обнаружение (п. 4.1.2):} Устройства распознают друг друга и получают доступ к необходимой информации.
    \item \textbf{Конфигурирование (п. 4.1.3):} Установление связей между объектами.
    \item \textbf{Эксплуатация (п. 4.1.4):} Реализация функций приложений в соответствии с заданной целью.
    \item \textbf{Управление (п. 4.1.5):} Мониторинг состояния системы, удалённая диагностика и настройка.
\end{enumerate}

\subsection*{Условия соответствия (раздел 5)}
Для обеспечения функциональной совместимости устройства должны соответствовать следующим требованиям:
\begin{itemize}
    \item \textbf{Идентификатор объекта (п. 5.1.2):} Каждый объект должен иметь уникальный идентификатор.
    \item \textbf{Описание объекта (п. 5.1.3):} Указание функциональности, безопасности и состояния объекта.
    \item \textbf{Процессы обнаружения и конфигурирования (п. 5.2.3 и 5.2.4):} Стандартизированные механизмы определения и настройки объектов.
\end{itemize}

\subsection*{Заключение}
ГОСТ Р 58538-2019 предоставляет чёткую методологию для обеспечения взаимодействия между устройствами и приложениями в системах промышленной автоматизации.
Он особенно полезен для сложных многокомпонентных систем, требующих высокого уровня совместимости и интеграции.

\clearpage

\section{Выбор подходящего стандарта для проекта}

Для выбора наиболее подходящих стандартов из каждой группы
для разрабатываемого проекта можно рассмотреть их достоинства и недостатки,
чтобы определить, какие из них лучше всего соответствуют специфике разработки.

\subsection{Стандарты управления данными и взаимодействия информационных систем}

\subsubsection*{ГОСТ 34.321-96}

\paragraph{Достоинства:}
\begin{itemize}
    \item \textbf{Универсальность управления данными:} Устанавливает эталонную модель, определяющую общую терминологию
    и понятия для управления данными в информационных системах (раздел 1, область применения).
    \item \textbf{Независимость данных:} Обеспечивает независимость процессов от объектов данных, что позволяет изменять
    объекты данных без нарушения процессов (п. 4.4).
    \item \textbf{Управление доступом:} Предусматривает санкционированный доступ и предотвращение несанкционированного
    использования ресурсов (п. 4.6).
    \item \textbf{Реструктурирование данных:} Поддерживает логическое реструктурирование данных для повышения их
    эффективности в течение жизненного цикла (п. 4.7.10).
    \item \textbf{Работа с распределёнными системами:} Рассматриваются аспекты управления данными, включая фрагментацию,
    дублирование и восстановление в распределённых базах данных (п. 4.8).
\end{itemize}

\paragraph{Недостатки:}
\begin{itemize}
    \item \textbf{Ограниченное внимание к семантике:} Семантика данных рассматривается лишь косвенно, через правила
    структурирования и моделирования данных (раздел 5).
    \item \textbf{Меньший акцент на пользовательском взаимодействии:} Отсутствует глубокий фокус на восприятии данных
    конечным пользователем.
\end{itemize}

\subsubsection*{ГОСТ Р 43.0.11-2014}

\paragraph{Достоинства:}
\begin{itemize}
    \item \textbf{Семантическая организация данных:} Делает акцент на перцептивное и грамматическое представление данных для упрощения восприятия и принятия решений (п. 5.8).
    \item \textbf{Удобство для операторов:} Создаёт базы данных, предназначенные для специалистов, обеспечивая удобство осмысления и взаимодействия (п. 1, область применения).
    \item \textbf{Поддержка знаний:} Рассматривает базы данных как основу для создания баз знаний и упрощения их применения в обучении и практике (приложение А) .
\end{itemize}

\paragraph{Недостатки:}
\begin{itemize}
    \item \textbf{Ограниченный охват распределённых систем:} Не рассматривает работу с распределёнными базами данных.
    \item \textbf{Независимость данных:} Проблема независимости данных напрямую не поднимается, акцент сделан на их семантической организации (раздел 5).
    \item \textbf{Меньший фокус на безопасности:} Управление доступом связано с восприятием данных, а не с предотвращением несанкционированного доступа (п. 5.2).
\end{itemize}

\subsubsection*{Вывод}
\begin{itemize}
    \item \textbf{ГОСТ 34.321-96:} Подходит для систем, где важны универсальность, согласованность данных и работа в распределённых системах.
    \item \textbf{ГОСТ Р 43.0.11-2014:} Лучше для областей, где важны удобство восприятия данных и их использование в практической деятельности.
\end{itemize}
Для рекомендательной системы предпочтителен \textbf{ГОСТ 34.321-96}, ориентированный на устойчивость и согласованность данных.

\subsection{Стандарты информационной безопасности и криптографической защиты информации}


\subsubsection*{ГОСТ Р 59162–2020}
\paragraph{Достоинства:}
\begin{itemize}
\item \textbf{Комплексный подход к безопасности беспроводных сетей:} Стандарт охватывает меры информационной безопасности, угрозы и рекомендации по защите беспроводных IP-сетей (раздел 1, область применения).
\item \textbf{Разделение на категории сетей:} Учитываются различия между WPAN, WLAN, WMAN и их специфические особенности безопасности (п. 6.1–6.3).
\item \textbf{Учет современных угроз:} Рассматриваются различные типы атак, включая отказ в обслуживании, анализ пакетов и атаки через Bluetooth (раздел 7).
\item \textbf{Рекомендации по проектированию систем:} Стандарт предоставляет методы проектирования, включая управление уязвимостями и использование безопасных конфигураций (раздел 9).
\end{itemize}

\paragraph{Недостатки:}
\begin{itemize}
\item \textbf{Фокус на беспроводные сети:} Ограниченная применимость к другим типам сетей, что делает стандарт менее универсальным для комплексных инфраструктур (раздел 1).
\item \textbf{Отсутствие подробных криптографических рекомендаций:} Не предоставляет детализированных алгоритмов, сосредотачиваясь на архитектурных аспектах (разделы 8, 9).
\end{itemize}

\subsubsection*{Р 50.1.115–2016}
\paragraph{Достоинства:}
\begin{itemize}
\item \textbf{Криптографическая основа:} Описывает протокол выработки общего ключа с использованием пароля и методами эллиптической криптографии (раздел 4).
\item \textbf{Универсальность:} Применим как для корпоративных, так и для общедоступных сетей, обеспечивая защиту каналов связи (раздел 1).
\item \textbf{Высокий уровень безопасности:} Использует современные криптографические подходы, такие как хэш-функции и электронная подпись (п. 4.1, 4.3).
\end{itemize}

\paragraph{Недостатки:}
\begin{itemize}
\item \textbf{Ограниченная область применения:} Сосредоточен на отдельных аспектах криптографии, не охватывая архитектурные и организационные аспекты безопасности (раздел 1).
\item \textbf{Сложность реализации:} Реализация алгоритмов требует высокой квалификации, что может усложнить их применение в малых организациях (п. 4.3, 5).
\end{itemize}

\subsubsection*{Вывод}
\begin{itemize}
\item \textbf{ГОСТ Р 59162–2020:} Лучше подходит для защиты беспроводных сетей, предоставляя рекомендации по архитектурным решениям и учету современных угроз.
\item \textbf{Р 50.1.115–2016:} Более уместен для обеспечения криптографической безопасности, особенно при необходимости выработки ключей и аутентификации.
\end{itemize}

Для приложения рекомендательной системы фильмов наиболее подходящим стандартом является \textbf{ГОСТ Р 50.1.115-2016}

\subsection{Стандарты проектирования пользовательских интерфейсов}

\subsubsection*{ГОСТ Р ИСО 9241-151-2014}

\paragraph{Достоинства:}
\begin{itemize}
    \item \textbf{Навигация и доступность:} Определяет требования для удобной навигации и доступности, включая элементы
    интерфейса, что улучшает взаимодействие пользователей с системой (\cite[п. 8]).
    \item \textbf{Совместимость с международными практиками:} Учитывает современные принципы проектирования для создания
    удобных и интуитивно понятных интерфейсов (раздел 5).
\end{itemize}

\paragraph{Недостатки:}
\begin{itemize}
    \item \textbf{Ограничение на веб-контекст:} Фокус стандарта больше на веб-приложениях, что может быть недостаточным
    для полной реализации мобильного приложения (п 4.1).
    \item \textbf{Меньший акцент на взаимодействии человека с системой:} Основной упор сделан на дизайн веб-интерфейсов,
    а не на общую концепцию взаимодействия (п. 6.1).
\end{itemize}

\subsubsection*{ГОСТ Р ИСО 9241-210-2012}

\paragraph{Достоинства:}
\begin{itemize}
    \item \textbf{Человеко-центричный подход:} Рассматривает проектирование интерфейсов с учётом потребностей и
    особенностей пользователей, что критично для рекомендательных систем (п. 4.2)
    \item \textbf{Поддержка интерактивных систем:} Ориентирован на создание удобных и интуитивных интерфейсов для
    сложных интерактивных систем (п. 5.1).
    \item \textbf{Обеспечение эффективности работы:} Способствует повышению продуктивности пользователей за счёт
    учёта их реальных потребностей (\cite[п. 4.1]).
\end{itemize}

\paragraph{Недостатки:}
\begin{itemize}
    \item \textbf{Менее специфичен для веб-контекста:} Меньше внимания уделено техническим аспектам дизайна
    веб-приложений, чем в ГОСТ Р ИСО 9241-151-2014 (п. 3).
    \item \textbf{Не описывает конкретные элементы интерфейсов:} Концентрация на концепциях взаимодействия, но не на
    деталях проектирования (п. 6.1).
\end{itemize}

\subsubsection*{Вывод}
\begin{itemize}
    \item \textbf{ГОСТ Р ИСО 9241-151-2014:} Предпочтителен для задач, связанных с проектированием веб-интерфейсов и доступностью API.
    \item \textbf{ГОСТ Р ИСО 9241-210-2012:} Подходит для систем, ориентированных на удобство пользователей и сложные интерактивные взаимодействия.
\end{itemize}

\subsubsection*{Обоснование выбора для рекомендательной системы фильмов}

Для рекомендательной системы фильмов предпочтителен
\textbf{ГОСТ Р ИСО 9241-210-2012}, так как он ориентирован на человеко-центричный дизайн и удобство взаимодействия.
Этот стандарт позволяет учитывать реальные потребности пользователей и улучшать их опыт, что критично для успешного
функционирования рекомендательной системы.

\subsection{Стандарты тестирования и обеспечения качества программного обеспечения}


\subsubsection*{ГОСТ Р ИСО/МЭК 25051--2017}

\paragraph{Достоинства:}
\begin{itemize}
    \item \textbf{Требования к качеству:} Устанавливает чёткие требования к функциональной пригодности,
    производительности, удобству использования и совместимости программных продуктов (пункт 5.1.5--5.1.8).
    \item \textbf{Оценка соответствия:} Определяет процедуры оценки соответствия требованиям, включая тестирование и верификацию (раздел 7).
    \item \textbf{Поддержка готовых решений:} Рассматривает аспекты качества готовых к использованию пр
    одуктов, что актуально для API и мобильных приложений (п1).
\end{itemize}

\paragraph{Недостатки:}
\begin{itemize}
    \item \textbf{Ограниченность жизненного цикла:} Основное внимание уделяется готовым продуктам, но менее охватываются аспекты их жизненного цикла и разработка.
    \item \textbf{Меньший акцент на автоматизированном тестировании:} Не рассматриваются специфичные для CI/CD методы тестирования программного обеспечения.
\end{itemize}

\subsubsection*{ГОСТ Р 56920--2024}

\paragraph{Достоинства:}
\begin{itemize}
    \item \textbf{Комплексный подход к тестированию:} Рассматривает полный цикл тестирования, включая п
    ланирование, разработку, выполнение тестов, управление инцидентами и регрессионное тестирование (пункты 5.4--5.9).
    \item \textbf{Документирование:} Особое внимание уделено стандартам и форматам документации тестирования (пункты 5.10--5.12).
\end{itemize}

\paragraph{Недостатки:}
\begin{itemize}
    \item \textbf{Ограниченность требований к функциональности:} Меньший акцент на описании требований к функциональной пригодности и удобству использования.
    \item \textbf{Фокус на больших системах:} Более применим к сложным, крупным системам, чем к мобильным приложениям или API.
\end{itemize}

\subsubsection*{Вывод и выбор стандарта}

Для рекомендательной системы фильмов:
\begin{itemize}
    \item \textbf{ГОСТ Р ИСО/МЭК 25051--2017} предпочтителен благодаря акценту на требованиях к качеству
    готовых решений и поддержке оценки соответствия пользовательских приложений и API.
    \item ГОСТ Р 56920--2024 может быть полезен для управления процессами тестирования и документирования, особенно в случае масштабирования системы.
\end{itemize}
Таким образом, основным стандартом для следует выбрать \textbf{ГОСТ Р ИСО/МЭК 25051--2017},
с частичным применением практик из ГОСТ Р 56920--2024 для разработки и автоматизации тестирования.

\subsection{Стандарты интеграции и совместимости}


\subsubsection*{ГОСТ Р 57136-2016}

\paragraph{Достоинства:}
\begin{itemize}
    \item \textbf{Гибкость интеграции:} Предлагает подход к интеграции приложений с использованием моделирования
    требований к обмену информацией и профилирования функциональных возможностей программного обеспечения, что ва
    жно для сложных систем (раздел 5.1) .
    \item \textbf{Устранение неоднозначностей:} Определяет открытые технические словари для стандартизации термин
    ов и упрощения коммуникации между системами (раздел 6.2).
\end{itemize}

\paragraph{Недостатки:}
\begin{itemize}
    \item \textbf{Фокус на промышленности:} Основное применение связано с промышленной автоматизацией, что
    ограничивает его применимость для приложений в других доменах, таких как рекомендательные системы (раздел 1).
    \item \textbf{Сложность внедрения:} Высокий уровень детализации может затруднить внедрение для менее
    формализованных систем, таких как мобильные приложения (раздел 8).
\end{itemize}

\subsubsection*{ГОСТ Р 58538-2019}

\paragraph{Достоинства:}
\begin{itemize}
    \item \textbf{Функциональная совместимость:} Ориентирован на обеспечение интероперабельности систем за счет
    описания требований и этапов реализации совместимости (раздел 4.1).
    \item \textbf{Масштабируемость:} Предоставляет методы для работы с системами разного масштаба, что важно
    для API и баз данных в рекомендательной системе (раздел 5).
\end{itemize}

\paragraph{Недостатки:}
\begin{itemize}
    \item \textbf{Фокус на аппаратных решениях:} Основное внимание уделяется взаимодействию аппаратных систем, что
    может быть избыточным для приложений с акцентом на софтверные аспекты (раздел 4.2).
    \item \textbf{Ограниченная масштабируемость:}Хоть стандарт и поддерживает совместимость на разных уровнях,
    его рекомендации больше сосредоточены на работе существующих систем и не дают детальных решений для масштабирования
    в больших информационных системах (п. 4.1.4, Этап эксплуатации)
    \item Недостаточный акцент на прикладных аспектах:
    \item Стандарт не предоставляет достаточных рекомендаций для разработки прикладных решений, таких как взаимодействие
    пользовательских интерфейсов, что важно для приложений с высокой степенью вовлеченности конечных пользователей,
    включая рекомендательные системы (Введение).

\end{itemize}

\subsubsection*{Вывод}
\begin{itemize}
    \item \textbf{ГОСТ Р 57136-2016:} Подходит для приложений, требующих высокой формализации и профилирования
    функциональных возможностей.
    Может быть полезен для анализа интеграции между модулями рекомендательной системы, но избыточен для мобильного приложения.
    \item \textbf{ГОСТ Р 58538-2019:} Предоставляет более универсальные решения для обеспечения совместимости
    между компонентами, включая API и базы данных.
    Его гибкость и фокус на масштабируемости делают его предпочтительным
    выбором для мобильного приложения рекомендательной системы фильмов.
\end{itemize}

Предпочтителен \textbf{ГОСТ Р 58538-2019}, так как он лучше
поддерживает интероперабельность и масштабируемость в условиях взаимодействия программных компонентов.


\clearpage

\section*{\LARGE Вывод}
\addcontentsline{toc}{section}{Вывод}

В ходе выполнения задания были изучены различные группы стандартов,
связанные с темой проекта.
Было отмечено, что каждая группа стандартов включает
свои специфические аспекты,
такие как безопасность данных, точность конвертации правил,
взаимодействие инструментов проектирования,
разработка программного обеспечения и документирование.
Это позволило глубже понять требования и рекомендации,
которые необходимо учитывать при реализации подобных проектов.

