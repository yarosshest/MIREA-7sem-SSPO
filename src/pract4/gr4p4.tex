\subsection{Стандарты тестирования и обеспечения качества программного обеспечения}


\subsubsection*{ГОСТ Р ИСО/МЭК 25051--2017}

\paragraph{Достоинства:}
\begin{itemize}
    \item \textbf{Требования к качеству:} Устанавливает чёткие требования к функциональной пригодности,
    производительности, удобству использования и совместимости программных продуктов (пункт 5.1.5--5.1.8).
    \item \textbf{Оценка соответствия:} Определяет процедуры оценки соответствия требованиям, включая тестирование и верификацию (раздел 7).
    \item \textbf{Поддержка готовых решений:} Рассматривает аспекты качества готовых к использованию пр
    одуктов, что актуально для API и мобильных приложений (п1).
\end{itemize}

\paragraph{Недостатки:}
\begin{itemize}
    \item \textbf{Ограниченность жизненного цикла:} Основное внимание уделяется готовым продуктам, но менее охватываются аспекты их жизненного цикла и разработка.
    \item \textbf{Меньший акцент на автоматизированном тестировании:} Не рассматриваются специфичные для CI/CD методы тестирования программного обеспечения.
\end{itemize}

\subsubsection*{ГОСТ Р 56920--2024}

\paragraph{Достоинства:}
\begin{itemize}
    \item \textbf{Комплексный подход к тестированию:} Рассматривает полный цикл тестирования, включая п
    ланирование, разработку, выполнение тестов, управление инцидентами и регрессионное тестирование (пункты 5.4--5.9).
    \item \textbf{Документирование:} Особое внимание уделено стандартам и форматам документации тестирования (пункты 5.10--5.12).
\end{itemize}

\paragraph{Недостатки:}
\begin{itemize}
    \item \textbf{Ограниченность требований к функциональности:} Меньший акцент на описании требований к функциональной пригодности и удобству использования.
    \item \textbf{Фокус на больших системах:} Более применим к сложным, крупным системам, чем к мобильным приложениям или API.
\end{itemize}

\subsubsection*{Вывод и выбор стандарта}

Для рекомендательной системы фильмов:
\begin{itemize}
    \item \textbf{ГОСТ Р ИСО/МЭК 25051--2017} предпочтителен благодаря акценту на требованиях к качеству
    готовых решений и поддержке оценки соответствия пользовательских приложений и API.
    \item ГОСТ Р 56920--2024 может быть полезен для управления процессами тестирования и документирования, особенно в случае масштабирования системы.
\end{itemize}
Таким образом, основным стандартом для следует выбрать \textbf{ГОСТ Р ИСО/МЭК 25051--2017},
с частичным применением практик из ГОСТ Р 56920--2024 для разработки и автоматизации тестирования.
