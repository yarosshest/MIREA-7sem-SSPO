\subsection{Стандарты тестирования и обеспечения качества программного обеспечения}

\subsubsection{ГОСТ Р 56920 -- 2024}

\emph{\href{https://allgosts.ru/35/080/gost_r_56920-2024.pdf}{ГОСТ Р 56920 -- 2024}
Системная и программная инженерия. Тестирование программного обеспечения. Общие положения
}

\par
Комитет: TK 022 Информационные технологии

\subsection*{Область применения}
Настоящий стандарт содержит рекомендации для специалистов, осуществляющих разработку интерактивных систем, включая программные и аппаратные
Рассматриваются системы различных масштабов: от потребительских продуктов (например, мобильных приложений) до крупных автоматизированных систем (раздел 1,).

\subsection*{Логическое обоснование человеко-ориентированного подхода}
Применение человеко-ориентированного подхода способствует увеличению производительности пользователей, снижению расходов
на обучение, повышению доступности систем для различных категорий пользователей, а также снижению риска неблагоприятного влияния на здоровье.
Примеры выгод от применения подхода приведены в таблице 1 стандарта.

\subsection*{Принципы человеко-ориентированного проектирования}
Стандарт определяет шесть ключевых принципов (п4.1):
\begin{enumerate}
    \item Точное определение пользователей, задач и среды их работы.
    \item Вовлечение пользователей в проектирование.
    \item Оценка проекта с участием пользователей.
    \item Итеративное совершенствование проекта.
    \item Учет опыта пользователей.
    \item Междисциплинарный подход к проектированию.
\end{enumerate}

\subsection*{Планирование человеко-ориентированного проектирования}
Человеко-ориентированный подход должен быть интегрирован во все этапы жизненного цикла продукта:
от концепции до вывода из эксплуатации.
Планирование включает распределение ресурсов, определение методов, привлечение специалистов и организацию итераций (п5).

\subsection*{Процедура выполнения человеко-ориентированного проекта}
Проектирование включает четыре основные этапа:
\begin{itemize}
    \item Анализ условий использования (6.2.1).
    \item Определение требований пользователей.
    \item Разработка проектных решений.
    \item Оценка и совершенствование решений.
\end{itemize}
Каждый этап сопровождается описанием методов и рекомендаций для обеспечения соответствия требованиям пользователей.

\subsection*{Оценка и мониторинг проекта}
Стандарт предписывает использование оценки проекта на всех этапах его реализации.
Это может включать испытания с участием пользователей, проверку выполнения требований к удобству использования
и долгосрочный мониторинг эксплуатации системы (6.5).

\subsection*{Устойчивое развитие}
Человеко-ориентированное проектирование способствует решению задач устойчивого развития за счет улучшения экономических,
социальных и экологических характеристик систем (п7).

\section*{Заключение}
ГОСТ Р ИСО 9241-210–2012 предоставляет универсальные рекомендации, применимые к широкому спектру систем.
Принципы человеко-ориентированного проектирования способствуют созданию систем, удовлетворяющих требованиям
пользователей, улучшают взаимодействие и повышают качество жизни пользователей.



\subsubsection{ГОСТ Р ИСОМЭК 25051 -- 2017}

\emph{\href{https://meganorm.ru/Data/645/64532.pdf}{ГОСТ Р ИСОМЭК 25051 -- 2017}}
\par
Комитет: TK 022 Информационные технологии

устанавливает требования к качеству готового к использованию программного продукта (RUSP) и
инструкции по тестированию.
Стандарт направлен на обеспечение пользователей и разработчиков уверенности в соответствии программного обеспечения их
потребностям.
В нём представлены требования к описанию, документации и тестированию RUSP, а также рекомендации для оценки соответствия.

\subsection*{Область применения}
Стандарт применяется к программным продуктам, которые доступны для конечного пользователя без дополнительной разработки.
Он охватывает различные категории программного обеспечения, включая текстовые редакторы,
базы данных и приложения для смартфонов (\textit{раздел 1}).

\subsection*{Термины и определения}
Включены ключевые термины, такие как:
\begin{itemize}
    \item \textbf{RUSP} — готовый к использованию программный продукт, доступный пользователю без дополнительных доработок (\textit{п. 4.1.6});
    \item \textbf{Документация пользователя} — материалы, необходимые для правильного использования программного продукта (\textit{п. 4.1.26});
    \item \textbf{Оценка соответствия} — систематический анализ соответствия продукта требованиям (\textit{п. 4.1.4}).
\end{itemize}

\subsection*{Основные требования к RUSP}
\subsubsection*{Требования к описанию продукта}
Описание должно быть доступным для пользователей и включать:
\begin{itemize}
    \item Информацию о качественных характеристиках программного обеспечения (\textit{п. 5.1.2.1});
    \item Уникальный идентификатор продукта и данные о поставщике (\textit{п. 5.1.3.1-5.1.3.3});
    \item Сообщения о функциональной пригодности, надежности, безопасности и других аспектах качества (\textit{п. 5.1.5.1-5.1.10.1}).
\end{itemize}

\subsubsection*{Требования к документации пользователя}
Документация должна быть:
\begin{itemize}
    \item Доступной для пользователей (\textit{п. 5.2.1.1});
    \item Полной, включая описание всех функций и инструкций по их использованию (\textit{п. 5.2.4.1-5.2.4.4});
    \item Понятной и последовательной (\textit{п. 5.2.7.1-5.2.7.2}).
\end{itemize}

\subsubsection*{Требования к тестированию}
Включают:
\begin{itemize}
    \item План тестирования, который описывает подход, контрольные примеры и критерии прохождения (\textit{п. 6.2.1});
    \item Процедуры тестирования, включающие подготовку, выполнение и анализ результатов (\textit{п. 6.3.2});
    \item Результаты тестирования с указанием всех выявленных отклонений и предложений по их устранению (\textit{п. 6.4.1}).
\end{itemize}

\subsection*{Рекомендации по оценке соответствия}
Раздел 7 и приложение А предоставляют инструкции и рекомендации для проверки соответствия программного продукта установленным требованиям.
Они охватывают подготовку, процесс оценки и оформление результатов (\textit{п. 7.1-7.6}).

\section*{Вывод}
ГОСТ Р ИСО/МЭК 25051-2017 является ключевым стандартом для оценки качества программных продуктов.
Его применение позволяет поставщикам обеспечивать соответствие программного обеспечения требованиям пользователей,
а пользователям — получать уверенность в качестве приобретаемых решений.
