\section{Группы стандартов}

Разрабатываемый программный продукт предполагает следующие группы стандартов:

\begin{itemize}
    \item Стандарты управления данными и взаимодействия информационных систем;
    \item Стандарты информационной безопасности и криптографической защиты информации;
    \item Стандарты проектирования пользовательских интерфейсов;
    \item Стандарты тестирования и обеспечения качества программного обеспечения;
    \item Стандарты интеграции и совместимости.
\end{itemize}

\subsection{Стандарты управления данными и взаимодействия информационных систем}

\section*{Общее описание}
Группа стандартов направлена на обеспечение эффективного управления данными и организацию корректного взаимодействия информационных систем. Стандарты регламентируют:
\begin{itemize}
    \item Основные подходы к проектированию и использованию систем управления данными;
    \item Принципы совместимости и интеграции различных информационных систем;
    \item Терминологическую основу для разработчиков и пользователей.
\end{itemize}

\section*{Основные положения}
\subsection*{Управление данными}
\begin{itemize}
    \item Организация хранения, обработки, поиска и восстановления данных.
    \item Определение структуры данных через схемы и модели, такие как реляционная и иерархическая.
    \item Обеспечение целостности, актуальности и единообразия данных.
\end{itemize}

\subsection*{Взаимодействие информационных систем}
\begin{itemize}
    \item Установление единых протоколов и интерфейсов для обмена данными.
    \item Реализация независимости от используемых технологий и платформ.
    \item Защита данных через управление доступом и идентификацию пользователей.
\end{itemize}

\subsection{Стандарты информационной безопасности и криптографической защиты информации}

\section*{Общее описание}
Группа стандартов направлена на обеспечение безопасности информационных систем и защиты данных. Стандарты регламентируют:
\begin{itemize}
    \item Основные подходы к проектированию и использованию систем безопасности;
    \item Принципы совместимости и интеграции различных информационных систем;
    \item Терминологическую основу для разработчиков и пользователей.
\end{itemize}

\section*{Основные положения}
\subsection*{Безопасность беспроводных сетей}
\begin{itemize}
    \item Определение угроз и требований к информационной безопасности.
    \item Предложение мер контроля и проектирования систем безопасности для беспроводных сетей.
\end{itemize}

\subsection*{Криптографическая защита информации}
\begin{itemize}
    \item Описание протоколов выработки ключей и взаимной аутентификации.
    \item Применение в корпоративных и общедоступных сетях.
\end{itemize}

\section*{Цель и задачи группы стандартов}
\begin{itemize}
    \item Обеспечение конфиденциальности, целостности и доступности информации в информационных системах.
    \item Разработка рекомендаций по проектированию и реализации мер безопасности в сетях.
    \item Использование современных криптографических методов для защиты данных.
\end{itemize}

\section*{Примечания}
Данная группа стандартов предназначена для специалистов, занимающихся информационной безопасностью, включая:
\begin{itemize}
    \item Разработчиков и проектировщиков систем безопасности.
    \item Сетевых администраторов и инженеров.
    \item Сотрудников, ответственных за соблюдение нормативных требований в области безопасности.
\end{itemize}

\subsection{Стандарты проектирования пользовательских интерфейсов}

Цель: Гарантировать, что взаимодействие пользователя с приложением будет удобным, интуитивным и эстетически привлекательным.
\section*{Основные аспекты}
\begin{itemize}
    \item Принципы эргономики: Интерфейс должен быть адаптирован под широкий спектр пользователей, включая людей с ограниченными возможностями.
    \item Интуитивная навигация: Логичная структура интерфейса, позволяющая пользователям легко находить нужные функции.
    \item Многоуровневый дизайн: Создание интерфейсов, одинаково удобных для новичков и опытных пользователей.
    \item Адаптивность: Поддержка различных устройств и размеров экранов (мобильные телефоны, планшеты, компьютеры).
    \item Доступность: Учет принципов инклюзии, таких как поддержка специальных экранных считывателей и оптимизированная контрастность.
\end{itemize}

\subsection{Стандарты интеграции и совместимости}

Цель: Обеспечить безошибочное взаимодействие между всеми компонентами системы и их совместимость с внешними сервисами.
\section*{Основные аспекты}
\begin{itemize}
    \item Функциональное тестирование: Проверка работы всех модулей приложения, включая правильность рекомендаций, взаимодействие с базой данных и API.
    \item Нагрузочное тестирование: Анализ производительности системы при увеличении числа пользователей, обработке большого объема запросов или данных.
    \item Тестирование безопасности: Защита пользовательских данных от утечек, защита от SQL-инъекций, защита от несанкционированного доступа.
    \item Автоматизация тестирования: Использование специализированных инструментов для регулярного и быстрого тестирования кода.
    \item Оценка пользовательского опыта: Проведение тестирования удобства использования интерфейса, времени отклика и общего восприятия системы пользователями.
\end{itemize}

\subsection{Стандарты тестирования и обеспечения качества программного обеспечения}

Цель: Обеспечить работоспособность системы в любых эксплуатационных условиях и предотвратить сбои, которые могут нарушить пользовательский опыт.
\section*{Основные аспекты}
Данная группа стандартов предназначена для специалистов, занимающихся информационной безопасностью, включая:
\begin{itemize}
    \item Интеграция компонентов: Мобильное приложение, API и база данных должны быть связаны через четко определённые интерфейсы.
    \item Поддержка стандартных форматов данных: Использование общепринятых форматов (например, JSON или XML) для обмена данными между компонентами и внешними системами.
    \item Масштабируемость: Обеспечение возможности добавления новых функций или подключения дополнительных сервисов без изменения существующей архитектуры.
    \item Совместимость с внешними API: Возможность подключения сторонних сервисов (например, системы рекомендаций, рейтингов или платформ потокового видео).
    \item Обеспечение отказоустойчивости: Минимизация рисков сбоев в интеграции, которые могут нарушить работу системы.
\end{itemize}