
\section{Группы стандартов}

Разрабатываемый программный продукт предполагает следующие группы стандартов:

\begin{itemize}
    \item Стандарты управления данными и взаимодействия информационных систем;
    \item Стандарты информационной безопасности и криптографической защиты информации;
    \item Стандарты производительности и качества ИТ продуктов и систем;
    \item Стандарты портативности прикладного программного обеспечения;
    \item Стандарты интероперабельности ИТ продуктов и систем;
    \item Стандарты унификации инструментов и средств разработки;
    \item Стандарты документирования ПО.
\end{itemize}

\subsection{Стандарты управления данными и взаимодействия информационных систем}

\section*{Общее описание}
Группа стандартов направлена на обеспечение эффективного управления данными и организацию корректного взаимодействия информационных систем. Стандарты регламентируют:
\begin{itemize}
    \item Основные подходы к проектированию и использованию систем управления данными;
    \item Принципы совместимости и интеграции различных информационных систем;
    \item Терминологическую основу для разработчиков и пользователей.
\end{itemize}

\section*{Основные положения}
\subsection*{Управление данными}
\begin{itemize}
    \item Организация хранения, обработки, поиска и восстановления данных.
    \item Определение структуры данных через схемы и модели, такие как реляционная и иерархическая.
    \item Обеспечение целостности, актуальности и единообразия данных.
\end{itemize}

\subsection*{Взаимодействие информационных систем}
\begin{itemize}
    \item Установление единых протоколов и интерфейсов для обмена данными.
    \item Реализация независимости от используемых технологий и платформ.
    \item Защита данных через управление доступом и идентификацию пользователей.
\end{itemize}
\subsection{Производительность и качество ИТ продуктов и систем}

Стандарты производительности и качества ИТ-продуктов и систем
направлены на обеспечение высоких показателей работы программного обеспечения
и оборудования.

Они включают:

\begin{itemize}
    \item Метрики производительности, такие как время отклика системы,
    пропускная способность сети, скорость обработки данных и т. д.
    \item Контроль качества и тестирование продуктов на этапах разработки
    и эксплуатации для выявления ошибок и уязвимостей.
    \item Системы управления качеством, такие как ISO/IEC 25010,
    которые определяют критерии качества программного обеспечения
    (функциональность, надежность, эффективность).
    \item Эти стандарты обеспечивают, чтобы продукты
    и системы отвечали ожиданиям пользователей по качеству и стабильности.
\end{itemize}




\subsection{Стандарты информационной безопасности и криптографической защиты информации}

\section*{Общее описание}
Группа стандартов направлена на обеспечение безопасности информационных систем и защиты данных. Стандарты регламентируют:
\begin{itemize}
    \item Основные подходы к проектированию и использованию систем безопасности;
    \item Принципы совместимости и интеграции различных информационных систем;
    \item Терминологическую основу для разработчиков и пользователей.
\end{itemize}

\section*{Основные положения}
\subsection*{Безопасность беспроводных сетей}
\begin{itemize}
    \item Определение угроз и требований к информационной безопасности.
    \item Предложение мер контроля и проектирования систем безопасности для беспроводных сетей.
\end{itemize}

\subsection*{Криптографическая защита информации}
\begin{itemize}
    \item Описание протоколов выработки ключей и взаимной аутентификации.
    \item Применение в корпоративных и общедоступных сетях.
\end{itemize}

\section*{Цель и задачи группы стандартов}
\begin{itemize}
    \item Обеспечение конфиденциальности, целостности и доступности информации в информационных системах.
    \item Разработка рекомендаций по проектированию и реализации мер безопасности в сетях.
    \item Использование современных криптографических методов для защиты данных.
\end{itemize}

\section*{Примечания}
Данная группа стандартов предназначена для специалистов, занимающихся информационной безопасностью, включая:
\begin{itemize}
    \item Разработчиков и проектировщиков систем безопасности.
    \item Сетевых администраторов и инженеров.
    \item Сотрудников, ответственных за соблюдение нормативных требований в области безопасности.
\end{itemize}

\subsection{Интероперабельность ИТ продуктов и систем}

Интероперабельность означает способность различных ИТ-систем
и продуктов взаимодействовать друг с другом и обмениваться данными.

Стандарты интероперабельности охватывают:

\begin{itemize}
    \item Сетевые протоколы (например, TCP/IP, HTTP)
    для обмена данными между различными устройствами и сетями.
    \item Форматы данных (XML, JSON),
    которые обеспечивают единый способ обмена информацией
    между приложениями.
    \item API и интерфейсы для интеграции различных программных систем
    и сервисов.
    \item Эти стандарты критически важны для построения взаимосвязанных систем,
    таких как интернет вещей (IoT), облачные сервисы и корпоративные сети.
\end{itemize}

\subsection{Документирование ПО}

Стандарты в области документирования разрабатывают требования
и рекомендации для создания, оформления
и управления документацией в самых разных сферах --- от производства
и проектирования до обслуживания и безопасности.
Такие стандарты призваны обеспечить единство,
качество и удобство использования документов,
а также повысить эффективность процессов обмена информацией
и минимизировать риски,
связанные с неправильным или недостаточным документированием.

Основные цели стандартов на документирование:

\begin{itemize}
    \item Унификация и стандартизация.
    Устанавливают единые правила для создания документов,
    что облегчает их понимание, проверку и сравнение.
    Это особенно важно в компаниях и отраслях,
    где несколько отделов или организаций взаимодействуют друг с другом.
    \item Повышение качества информации.
    Стандарты требуют точности, актуальности,
    достоверности и полноты данных.
    Это помогает избежать ошибок и недопонимания,
    которые могут привести к производственным сбоям,
    потерям и несоответствиям.
    \item Соответствие требованиям законодательства.
    В ряде отраслей (например, в фармацевтике, авиации, энергетике)
    требования к документированию строго регламентированы,
    и несоответствие стандартам может привести к штрафам
    или остановке деятельности.
\end{itemize}
