\subsection{Стандарты интеграции и совместимости}


\subsubsection*{ГОСТ Р 57136-2016}

\paragraph{Достоинства:}
\begin{itemize}
    \item \textbf{Гибкость интеграции:} Предлагает подход к интеграции приложений с использованием моделирования
    требований к обмену информацией и профилирования функциональных возможностей программного обеспечения, что ва
    жно для сложных систем (раздел 5.1) .
    \item \textbf{Устранение неоднозначностей:} Определяет открытые технические словари для стандартизации термин
    ов и упрощения коммуникации между системами (раздел 6.2).
\end{itemize}

\paragraph{Недостатки:}
\begin{itemize}
    \item \textbf{Фокус на промышленности:} Основное применение связано с промышленной автоматизацией, что
    ограничивает его применимость для приложений в других доменах, таких как рекомендательные системы (раздел 1).
    \item \textbf{Сложность внедрения:} Высокий уровень детализации может затруднить внедрение для менее
    формализованных систем, таких как мобильные приложения (раздел 8).
\end{itemize}

\subsubsection*{ГОСТ Р 58538-2019}

\paragraph{Достоинства:}
\begin{itemize}
    \item \textbf{Функциональная совместимость:} Ориентирован на обеспечение интероперабельности систем за счет
    описания требований и этапов реализации совместимости (раздел 4.1).
    \item \textbf{Масштабируемость:} Предоставляет методы для работы с системами разного масштаба, что важно
    для API и баз данных в рекомендательной системе (раздел 5).
\end{itemize}

\paragraph{Недостатки:}
\begin{itemize}
    \item \textbf{Фокус на аппаратных решениях:} Основное внимание уделяется взаимодействию аппаратных систем, что
    может быть избыточным для приложений с акцентом на софтверные аспекты (раздел 4.2).
    \item \textbf{Ограниченная масштабируемость:}Хоть стандарт и поддерживает совместимость на разных уровнях,
    его рекомендации больше сосредоточены на работе существующих систем и не дают детальных решений для масштабирования
    в больших информационных системах (п. 4.1.4, Этап эксплуатации)
    \item Недостаточный акцент на прикладных аспектах:
    \item Стандарт не предоставляет достаточных рекомендаций для разработки прикладных решений, таких как взаимодействие
    пользовательских интерфейсов, что важно для приложений с высокой степенью вовлеченности конечных пользователей,
    включая рекомендательные системы (Введение).

\end{itemize}

\subsubsection*{Вывод}
\begin{itemize}
    \item \textbf{ГОСТ Р 57136-2016:} Подходит для приложений, требующих высокой формализации и профилирования
    функциональных возможностей.
    Может быть полезен для анализа интеграции между модулями рекомендательной системы, но избыточен для мобильного приложения.
    \item \textbf{ГОСТ Р 58538-2019:} Предоставляет более универсальные решения для обеспечения совместимости
    между компонентами, включая API и базы данных.
    Его гибкость и фокус на масштабируемости делают его предпочтительным
    выбором для мобильного приложения рекомендательной системы фильмов.
\end{itemize}

Предпочтителен \textbf{ГОСТ Р 58538-2019}, так как он лучше
поддерживает интероперабельность и масштабируемость в условиях взаимодействия программных компонентов.
