\subsection{Стандарты проектирования пользовательских интерфейсов}

\subsubsection*{ГОСТ Р ИСО 9241-151-2014}

\paragraph{Достоинства:}
\begin{itemize}
    \item \textbf{Навигация и доступность:} Определяет требования для удобной навигации и доступности, включая элементы
    интерфейса, что улучшает взаимодействие пользователей с системой (\cite[п. 8]).
    \item \textbf{Совместимость с международными практиками:} Учитывает современные принципы проектирования для создания
    удобных и интуитивно понятных интерфейсов (раздел 5).
\end{itemize}

\paragraph{Недостатки:}
\begin{itemize}
    \item \textbf{Ограничение на веб-контекст:} Фокус стандарта больше на веб-приложениях, что может быть недостаточным
    для полной реализации мобильного приложения (п 4.1).
    \item \textbf{Меньший акцент на взаимодействии человека с системой:} Основной упор сделан на дизайн веб-интерфейсов,
    а не на общую концепцию взаимодействия (п. 6.1).
\end{itemize}

\subsubsection*{ГОСТ Р ИСО 9241-210-2012}

\paragraph{Достоинства:}
\begin{itemize}
    \item \textbf{Человеко-центричный подход:} Рассматривает проектирование интерфейсов с учётом потребностей и
    особенностей пользователей, что критично для рекомендательных систем (п. 4.2)
    \item \textbf{Поддержка интерактивных систем:} Ориентирован на создание удобных и интуитивных интерфейсов для
    сложных интерактивных систем (п. 5.1).
    \item \textbf{Обеспечение эффективности работы:} Способствует повышению продуктивности пользователей за счёт
    учёта их реальных потребностей (\cite[п. 4.1]).
\end{itemize}

\paragraph{Недостатки:}
\begin{itemize}
    \item \textbf{Менее специфичен для веб-контекста:} Меньше внимания уделено техническим аспектам дизайна
    веб-приложений, чем в ГОСТ Р ИСО 9241-151-2014 (п. 3).
    \item \textbf{Не описывает конкретные элементы интерфейсов:} Концентрация на концепциях взаимодействия, но не на
    деталях проектирования (п. 6.1).
\end{itemize}

\subsubsection*{Вывод}
\begin{itemize}
    \item \textbf{ГОСТ Р ИСО 9241-151-2014:} Предпочтителен для задач, связанных с проектированием веб-интерфейсов и доступностью API.
    \item \textbf{ГОСТ Р ИСО 9241-210-2012:} Подходит для систем, ориентированных на удобство пользователей и сложные интерактивные взаимодействия.
\end{itemize}

\subsubsection*{Обоснование выбора для рекомендательной системы фильмов}

Для рекомендательной системы фильмов предпочтителен
\textbf{ГОСТ Р ИСО 9241-210-2012}, так как он ориентирован на человеко-центричный дизайн и удобство взаимодействия.
Этот стандарт позволяет учитывать реальные потребности пользователей и улучшать их опыт, что критично для успешного
функционирования рекомендательной системы.
