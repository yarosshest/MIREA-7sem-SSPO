\subsection{Стандарты проектирования пользовательских интерфейсов}

\subsubsection{ГОСТ Р ИСО 9241-210 -- 2012}

ГОСТ Р ИСО 9241-210 -- 2012
\emph{\href{https://meganorm.ru/Data/534/53476.pdf}{ГОСТ Р ИСО 9241-210 -- 2012}
ЭРГОНОМИКА ВЗАИМОДЕЙСТВИЯ
ЧЕЛОВЕК—СИСТЕМА
Часть 210
Человеко-ориентированное проектирование
интерактивных систем
}
\par
Коммитет:  ТК 201 «Эргономика, психология труда и инженерная психология»


\subsection*{Область применения}
Настоящий стандарт содержит рекомендации для специалистов, осуществляющих разработку интерактивных систем, включая программные и аппаратные
Рассматриваются системы различных масштабов: от потребительских продуктов (например, мобильных приложений) до крупных автоматизированных систем (раздел 1,).

\subsection*{Логическое обоснование человеко-ориентированного подхода}
Применение человеко-ориентированного подхода способствует увеличению производительности пользователей, снижению расходов
на обучение, повышению доступности систем для различных категорий пользователей, а также снижению риска неблагоприятного влияния на здоровье.
Примеры выгод от применения подхода приведены в таблице 1 стандарта.

\subsection*{Принципы человеко-ориентированного проектирования}
Стандарт определяет шесть ключевых принципов (п4.1):
\begin{enumerate}
    \item Точное определение пользователей, задач и среды их работы.
    \item Вовлечение пользователей в проектирование.
    \item Оценка проекта с участием пользователей.
    \item Итеративное совершенствование проекта.
    \item Учет опыта пользователей.
    \item Междисциплинарный подход к проектированию.
\end{enumerate}

\subsection*{Планирование человеко-ориентированного проектирования}
Человеко-ориентированный подход должен быть интегрирован во все этапы жизненного цикла продукта:
от концепции до вывода из эксплуатации.
Планирование включает распределение ресурсов, определение методов, привлечение специалистов и организацию итераций (п5).

\subsection*{Процедура выполнения человеко-ориентированного проекта}
Проектирование включает четыре основные этапа:
\begin{itemize}
    \item Анализ условий использования (6.2.1).
    \item Определение требований пользователей.
    \item Разработка проектных решений.
    \item Оценка и совершенствование решений.
\end{itemize}
Каждый этап сопровождается описанием методов и рекомендаций для обеспечения соответствия требованиям пользователей.

\subsection*{Оценка и мониторинг проекта}
Стандарт предписывает использование оценки проекта на всех этапах его реализации.
Это может включать испытания с участием пользователей, проверку выполнения требований к удобству использования
и долгосрочный мониторинг эксплуатации системы (6.5).

\subsection*{Устойчивое развитие}
Человеко-ориентированное проектирование способствует решению задач устойчивого развития за счет улучшения экономических,
социальных и экологических характеристик систем (п7).

\section*{Заключение}
ГОСТ Р ИСО 9241-210–2012 предоставляет универсальные рекомендации, применимые к широкому спектру систем.
Принципы человеко-ориентированного проектирования способствуют созданию систем, удовлетворяющих требованиям
пользователей, улучшают взаимодействие и повышают качество жизни пользователей.



\subsubsection{ГОСТ Р ИСО 9241-151 -- 2014}

\emph{\href{https://meganorm.ru/Data2/1/4293768/4293768927.pdf}{ГОСТ Р ИСО 9241-151 -- 2014}
ЭРГОНОМИКА ВЗАИМОДЕЙСТВИЯ
ЧЕЛОВЕК — СИСТЕМА Часть 151
Руководство по проектированию пользовательских
интерфейсов сети Интернет
}
Коммитет:  ТК 201 «Эргономика, психология труда и инженерная психология»
\subsection*{Область применения}
Стандарт охватывает:
\begin{itemize}
    \item Проектирование архитектуры и стратегии (раздел 6);
    \item Разработка контента (раздел 7);
    \item Навигацию и поиск (раздел 8);
    \item Представление информационного наполнения (раздел 9).
\end{itemize}

\subsection*{Основные рекомендации}
\begin{itemize}
    \item Проектирование навигационной структуры (пункт 8.1);
    \item Адаптация контента для различных устройств (пункт 7.2.2);
    \item Учет потребностей пользователей с ограниченными возможностями (раздел 7.2.9).
\end{itemize}

\subsection*{Выводы}
ГОСТ Р ИСО 9241-151-2014 полезен для разработки пользовательских веб-интерфейсов, особенно в контексте обеспечения доступности и удобства навигации.
