\subsection{Стандарты тестирования и обеспечения качества программного обеспечения}

\subsubsection{ГОСТ Р 57136 -- 2016}

\emph{\href{https://meganorm.ru/Data2/1/4293751/4293751436.pdf}{ГОСТ Р 57136 -- 2016}
Системы промышленной автоматизации
и интеграция
ПОДХОД К ИНТЕГРАЦИИ ПРИЛОЖЕНИЙ
С ИСПОЛЬЗОВАНИЕМ МОДЕЛИРОВАНИЯ
ТРЕБОВАНИЙ К ОБМЕНУ ИНФОРМАЦИЕЙ
И ПРОФИЛИРОВАНИЯ ФУНКЦИОНАЛЬНЫХ
ВОЗМОЖНОСТЕЙ ПРОГРАММНОГО
ОБЕСПЕЧЕНИЯ
}
\par
Комитет: ТК 100 «Стратегический и инновационный менеджмент»

ГОСТ Р 57136-2016, идентичный международному стандарту ISO/TR 18161:2013, описывает подход к интеграции приложений
с использованием моделирования требований к обмену информацией и профилирования функциональных возможностей программного обеспечения.


\subsection*{Область применения}
ГОСТ применим для систем промышленной автоматизации и интеграции, где требуется стандартизированный подход к обмену информацией между приложениями. Он включает:
\begin{itemize}
    \item общие принципы интеграции (пункт 5.1);
    \item требования к функциональным возможностям программного обеспечения для интеграции (раздел 5);
    \item примеры и рекомендации для реализации профилей интеграции (раздел 6).
\end{itemize}

\subsection*{Основные элементы стандарта}
\subsubsection*{Моделирование требований к обмену информацией}
ГОСТ описывает, как определить и структурировать требования к обмену данными, чтобы обеспечить точность и надежность передачи информации между приложениями (пункт 5.2).

\subsubsection*{Профилирование функциональных возможностей}
Функциональные профили позволяют стандартизировать наборы требований к системам и упрощают процесс интеграции,
обеспечивая совместимость на уровне программного обеспечения (пункт 5.3).

\subsubsection*{Практические рекомендации}
В приложениях стандарта приведены примеры моделирования требований и создания функциональных профилей для конкретных типов интеграционных задач (раздел 6).

\subsection*{Вывод}
ГОСТ Р 57136-2016 является основополагающим документом для разработки интеграционных решений,
где важны стандартизированный обмен информацией и совместимость систем.
Его применение позволяет повысить эффективность разработки и интеграции сложных систем в промышленной автоматизации.



\subsubsection{ГОСТ Р 58538 -- 2019}

\emph{\href{https://meganorm.ru/Data/718/71869.pdf}{ГОСТ Р 58538 -- 2019}}
\par
Комитет: ТК 100 «Стратегический и инновационный менеджмент»


\subsection*{Общее описание}
ГОСТ Р 58538-2019 устанавливает спецификацию требований к организации информационного взаимодействия в системах промышленной
автоматизации и интеграции.
Основной целью стандарта является достижение функциональной совместимости устройств и приложений на основе единой методологии и этапов интеграции.

\subsection*{Область применения (раздел 1)}
Настоящий стандарт применим для обеспечения взаимодействия между системами и устройствами различного происхождения.
Он охватывает требования к функциональной совместимости на различных уровнях и описывает этапы обнаружения, конфигурирования, эксплуатации и управления устройствами.

\subsection*{Ключевые термины и определения (раздел 3)}
В стандарте определены такие ключевые термины, как:
\begin{itemize}
    \item \textbf{Функциональная совместимость (п. 3.3.2):} Способность систем обмениваться информацией и использовать её для совместной работы.
    \item \textbf{Обнаружение (п. 3.2.5):} Процесс, позволяющий системам находить новые элементы и определять их функциональность.
    \item \textbf{Конфигурирование (п. 3.2.4):} Установление связей между объектами для их взаимодействия.
\end{itemize}

\subsection*{Принципы функциональной совместимости (раздел 4)}
Стандарт определяет четыре этапа взаимодействия:
\begin{enumerate}
    \item \textbf{Обнаружение (п. 4.1.2):} Устройства распознают друг друга и получают доступ к необходимой информации.
    \item \textbf{Конфигурирование (п. 4.1.3):} Установление связей между объектами.
    \item \textbf{Эксплуатация (п. 4.1.4):} Реализация функций приложений в соответствии с заданной целью.
    \item \textbf{Управление (п. 4.1.5):} Мониторинг состояния системы, удалённая диагностика и настройка.
\end{enumerate}

\subsection*{Условия соответствия (раздел 5)}
Для обеспечения функциональной совместимости устройства должны соответствовать следующим требованиям:
\begin{itemize}
    \item \textbf{Идентификатор объекта (п. 5.1.2):} Каждый объект должен иметь уникальный идентификатор.
    \item \textbf{Описание объекта (п. 5.1.3):} Указание функциональности, безопасности и состояния объекта.
    \item \textbf{Процессы обнаружения и конфигурирования (п. 5.2.3 и 5.2.4):} Стандартизированные механизмы определения и настройки объектов.
\end{itemize}

\subsection*{Заключение}
ГОСТ Р 58538-2019 предоставляет чёткую методологию для обеспечения взаимодействия между устройствами и приложениями в системах промышленной автоматизации.
Он особенно полезен для сложных многокомпонентных систем, требующих высокого уровня совместимости и интеграции.
