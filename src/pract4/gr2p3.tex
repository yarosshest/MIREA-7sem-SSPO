\subsection{Стандарты информационной безопасности и криптографической защиты информации}

\subsubsection{ГОСТ Р 50.1.115 -- 2016}
\paragraph{ГОСТ Р 50.1.115 -- 2016}
\emph{\href{https://rst.gov.ru:8443/file-service/file/load/1699435168541}{Р 50.1.115 -- 2016}
Информационные технологии.
Криптографическая защита информации.
Протокол выработки общего ключа с аутентификацией на основе пароля}

\section*{Общее описание}
Настоящий ГОСТ описывает протокол выработки общего ключа с использованием пароля для аутентификации сторон. Протокол основан на эллиптических кривых и обеспечивает защиту от активного противника.

\section*{Область применения}
Стандарт применяется в системах с использованием криптографии для защиты данных, включая электронные цифровые подписи и хэширование. Основное назначение — установление защищенного соединения между сторонами.

\section*{Основные положения}
\begin{itemize}
    \item Использование эллиптических кривых для формирования ключей.
    \item Поддержка строгой аутентификации и защиты данных.
    \item Обеспечение защиты каналов связи от перехвата.
\end{itemize}

\subsubsection{ГОСТ Р 59162 -- 2020}
\paragraph{ГОСТ Р 59162 -- 2020}
\emph{\href{https://rst.gov.ru:8443/file-service/file/load/1699366818935}{ГОСТ Р 59162 -- 2020}
Информационные технологии.
Методы и средства обеспечения безопасности.
Безопасность сетей. Часть 6: Обеспечение информационной безопасности при использовании беспроводных IP-сетей.}

\section*{Общее описание}
Стандарт определяет методы и меры обеспечения информационной безопасности для беспроводных IP-сетей. Он основан на международных стандартах ISO/IEC 27033-6.

\section*{Область применения}
Применяется для проектирования, реализации и мониторинга систем безопасности беспроводных сетей в различных организациях.

\section*{Основные положения}
\begin{itemize}
    \item Определение угроз и рисков для беспроводных сетей.
    \item Реализация механизмов шифрования, аутентификации и контроля доступа.
    \item Поддержание устойчивости к атакам и обеспечение конфиденциальности данных.
\end{itemize}
