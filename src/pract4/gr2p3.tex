\subsection{Стандарты информационной безопасности и криптографической защиты информации}

\subsubsection{ГОСТ Р 50.1.115 -- 2016}
\paragraph{ГОСТ Р 50.1.115 -- 2016}
\emph{\href{https://rst.gov.ru:8443/file-service/file/load/1699435168541}{Р 50.1.115 -- 2016}
Информационные технологии.
Криптографическая защита информации.
Протокол выработки общего ключа с аутентификацией на основе пароля}
\par
ПК 1 Технического комитета по стандартизации ТК 26
\section*{Общее описание}

ГОСТ Р 50.1.115–2016 описывает протокол выработки общего криптографического ключа между двумя сторонами (A и B) с
взаимной аутентификацией на основе пароля .
Стандарт гарантирует, что обе стороны в конце выполнения протокола получают общий секретный ключ, не
раскрывая пароль посторонним лицам и противодействуя попыткам активных атакующих.

\section*{Назначение и область применения}
\label{sec:scope}
Протокол применим для систем, использующих криптографическую защиту информации, включая:
\begin{itemize}
    \item Выработку ключевых данных при наличии общего пароля;
    \item Применение в общедоступных и корпоративных сетях;
    \item Обеспечение аутентификации при установлении защищённого канала связи.
\end{itemize}
Стандарт рассчитан на ситуации, когда необходимо согласовать секретный ключ
для последующего применения, например, в протоколах шифрования или целостности данных.

\section*{Криптографические основы}
\label{sec:crypto}
Стойкость протокола основана на вычислительной сложности задачи Диффи-Хеллмана в группе точек эллиптической кривой.
Для хэширования используется функция ГОСТ Р 34.11–2012, а для кодов аутентификации — HMAC.
Применение PBKDF2 по ГОСТ Р 50.1.111–2016 обеспечивает преобразование пароля PW в криптографически стойкое значение.

\section*{Описание протокола}
\label{sec:prot_desc}
Протокол предполагает:
\begin{enumerate}
    \item Выбор эллиптической кривой и набора точек $\{Q_1, Q_2, \ldots, Q_N\}$ .
    \item Применение функции PBKDF2 к паролю PW и случайной соли (salt), чтобы получить точку $Q_{PW}$ на кривой.
    \item Обмен сообщениями между сторонами A и B, включающий:
    \begin{itemize}
        \item Вычисления над точками эллиптической кривой;
        \item Проверку аутентичности с помощью HMAC;
        \item Управление счётчиками попыток аутентификации (C1, C2, C3).
    \end{itemize}
    \item Формирование общего ключа $K$ после успешного завершения протокола.
\end{enumerate}

Все операции осуществляются в условиях потенциально активного противника, способного модифицировать или перехватывать
сообщения.
Благодаря взаимной аутентификации и использованию стойких криптопримитивов протокол снижает риск компрометации.

\section*{Генерация точек и параметры}
\label{sec:points}
Наборы точек $\{Q_1, \ldots, Q_N\}$ строятся доказуемо псевдослучайным образом с использованием хэш-функций.
Это гарантирует неизвестность их кратностей относительно порождающего элемента кривой, снижая возможности для криптоанализа.

В приложении стандарта даны примеры генерации точек для различных кривых, а также контрольные примеры для проверки корректности реализации.

\section*{Ключевой материал}
\label{sec:keymat}
По окончании протокола обе стороны получают общий ключ $K$.
Этот ключ может быть использован для установления защищённого соединения, например, в протоколах TLS или SSH, или для шифрования передаваемых данных.
Поскольку $K$ вырабатывается с учётом пароля PW, злоумышленник, не владеющий паролем, не способен эффективно вычислить данный ключ.

\section*{Учет попыток аутентификации}
\label{sec:counters}
Стандарт описывает ограничение числа неудачных попыток аутентификации, отслеживаемое счётчиками C1, C2, C3. При превышении допустимого числа попыток требуется смена пароля или применение дополнительных мер.
Это затрудняет перебор пароля и служит защитой от атак типа «отказ в обслуживании».

\section*{Приложения}
\label{sec:apps}
Стандарт содержит приложения:
\begin{itemize}
    \item Приложение А--- примеры точек на различных кривых, SEED-значений и координат.
    \item Приложение Б  --- контрольные примеры выполнения протокола для проверки корректности реализации.
\end{itemize}

Используя данные из приложений, разработчики могут проверить, что их реализация соответствует требованиям стандарта.

\section*{Заключение}
Протокол, описанный в ГОСТ Р 50.1.115–2016, обеспечивает безопасную выработку общего ключа при наличии общего пароля.
Он сочетает использование эллиптических кривых, стойких хэш-функций, HMAC, а также механизм PBKDF2 для повышения стойкости к перебору.
Управление попытками аутентификации и дополнительные параметры позволяют противостоять как пассивным, так и активным атакам.
Таким образом, стандарт даёт чёткие рекомендации по безопасному формированию ключевого материала для дальнейшего использования в криптосистемах.



\subsubsection{ГОСТ Р 59162 -- 2020}
\paragraph{ГОСТ Р 59162 -- 2020}
\emph{\href{https://rst.gov.ru:8443/file-service/file/load/1699366818935}{ГОСТ Р 59162 -- 2020}
Информационные технологии.
Методы и средства обеспечения безопасности.
Безопасность сетей. Часть 6: Обеспечение информационной безопасности при использовании беспроводных IP-сетей.}
Комитет: TK 022 Информационные технологии
\section*{Общее описание}
Стандарт описывает угрозы, требования к ИБ, меры контроля и методы проектирования для обеспечения безопасности беспроводных сетей.
Его положения предназначены для разработчиков и пользователей, ответственных за реализацию и эксплуатацию мер ИБ(Раздел 1).

\section*{Термины и определения}
В стандарте применяются термины из ГОСТ Р ИСО/МЭК 27000 и 27033-1, а также определяются следующие ключевые понятия:
\begin{itemize}
    \item \textbf{Точка беспроводного доступа (access point):} Устройство для подключения беспроводных устройств к проводной сети (п. 3.1).
    \item \textbf{Wi-Fi:} Технология беспроводных локальных сетей на основе стандартов IEEE 802.11 (п. 3.11).
    \item \textbf{Усиление защиты (hardening):} Процесс повышения защищённости системы.
\end{itemize}

\section*{Основные разделы}
ГОСТ Р 59162-2020 включает следующие основные разделы:
\begin{itemize}
    \item \textbf{Категории беспроводных IP-сетей:} Описываются WPAN, WLAN и WMAN, их технологии (Bluetooth, Wi-Fi, ZigBee и др.) и особенности (Раздел 6).
    \item \textbf{Угрозы безопасности:} Включают несанкционированный доступ, анализ пакетов, атаки типа "отказ в обслуживании" и др. (Раздел 7).
    \item \textbf{Требования к ИБ:} Конфиденциальность, целостность, доступность, аутентификация, авторизация и др. (Раздел 8).
    \item \textbf{Меры обеспечения ИБ:} Контроль шифрования, мониторинг, управление уязвимостями (Раздел 9).
    \item \textbf{Методы проектирования:} Особенности проектирования Wi-Fi, Bluetooth и других сетей (Раздел 10).
\end{itemize}

\section*{Угрозы безопасности}
В разделе 7 описаны типичные угрозы, включая:
\begin{itemize}
    \item \textbf{Несанкционированный доступ:} Получение доступа к сети без разрешения (п. 7.2).
    \item \textbf{Анализ пакетов:} Прослушивание трафика с использованием анализаторов (п. 7.3).
    \item \textbf{Фальшивая точка доступа:} Создание точек для перехвата данных (п. 7.4).
\end{itemize}
\section*{Заключение}
ГОСТ Р 59162-2020 представляет собой комплексный стандарт, направленный на обеспечение безопасности беспроводных IP-сетей.
Он охватывает широкий спектр вопросов от угроз и требований до методов реализации мер ИБ, что делает его незаменимым
инструментом для разработчиков и пользователей информационных систем.
