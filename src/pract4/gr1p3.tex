
\subsection{Стандарты управления данными и взаимодействия информационных систем}
\subsubsection{ГОСТ 34.321 -- 96}
\paragraph{ГОСТ 34.321 -- 96}
\emph{\href{https://internet-law.ru/gosts/gost/6808/}{ГОСТ 34.321 -- 96}
Информационные технологии.
Система стандартов по базам данных.
Эталонная модель управления данными
}
\par
Комитет: TK 022 Информационные технологии
\section*{Общее описание}
ГОСТ 34.321-96 устанавливает эталонную модель управления данными в информационных системах. Стандарт определяет терминологию, основные концепты и требования, связанные с управлением базами данных. Эталонная модель служит для описания услуг, предоставляемых системами управления базами данных (СУБД) и системами словарей данных.

\section*{Область применения}
Стандарт распространяется на процессы управления постоянными данными, включая их хранение, поиск, обновление, ввод, копирование, восстановление и передачу. Он ориентирован на:
\begin{itemize}
    \item Определение общих принципов управления данными;
    \item Обеспечение совместимости между различными системами;
    \item Создание единой терминологической основы для разработчиков и пользователей.
\end{itemize}

\section*{Термины и определения}
В рамках стандарта введены следующие ключевые термины:
\begin{itemize}
    \item \textbf{База данных (database):} Это совокупность взаимосвязанных данных, организованных в соответствии с определённой схемой, для обеспечения удобного доступа и обработки.
    \item \textbf{Схема базы данных (database schema):} Это формальное описание содержания, структуры и ограничений целостности базы данных.
    \item \textbf{Транзакция (transaction):} Совокупность операций, которая характеризуется такими свойствами, как атомарность, согласованность, изоляция и долговечность (ACID-свойства).
    \item \textbf{Управление доступом (access control):} Это процесс предотвращения несанкционированного доступа к данным путём использования различных механизмов идентификации и авторизации.
\end{itemize}

\section*{Основные компоненты модели}

\subsection*{База данных и схема}
Основным элементом модели является база данных, структура и содержание которой определяются её схемой. Схема включает описание элементов данных, их связей и правил целостности. Такое представление обеспечивает логическую организацию данных и их доступность для пользователей.

\subsection*{Средства моделирования данных}
Для создания схем базы данных применяются средства моделирования данных. Эти средства включают:
\begin{itemize}
    \item \textbf{Правила структурирования данных:} Определяют, как данные организуются и связываются друг с другом.
    \item \textbf{Правила манипулирования данными:} Определяют допустимые операции над данными.
\end{itemize}
Наиболее распространёнными моделями данных являются реляционная, сетевая и иерархическая модели.

\subsection*{Независимость данных}
Независимость данных обеспечивается следующими подходами:
\begin{itemize}
    \item Процессы взаимодействуют только с необходимыми частями схемы базы данных.
    \item Прикладные процессы отделены от физического представления данных.
    \item Ограничения целостности реализуются непосредственно в схеме базы данных.
\end{itemize}

\subsection*{Процессоры и интерфейсы}
Процессы управления данными выполняются с помощью процессоров, которые предоставляют интерфейсы для доступа к функционалу системы. Эти интерфейсы могут быть универсальными и независимыми от языков программирования.

\section*{Управление доступом}
Управление доступом в эталонной модели основывается на следующих принципах:
\begin{itemize}
    \item \textbf{Определение и модификация привилегий пользователей:} Позволяет устанавливать уровни доступа для различных категорий пользователей.
    \item \textbf{Реализация санкционированного доступа:} Осуществляется через процедуры идентификации и аутентификации пользователей.
    \item \textbf{Сохранение данных об управлении доступом:} Вся информация о привилегиях и ограничениях хранится непосредственно в базе данных.
\end{itemize}

\section*{Транзакции базы данных}
Транзакции базы данных представляют собой логические единицы работы, обеспечивающие надёжное выполнение операций. Основные требования к транзакциям включают:
\begin{itemize}
    \item \textbf{Атомарность:} Транзакция либо выполняется полностью, либо не выполняется вовсе.
    \item \textbf{Согласованность:} После выполнения транзакции данные остаются в непротиворечивом состоянии.
    \item \textbf{Изоляция:} Параллельно выполняемые транзакции не влияют на корректность данных.
    \item \textbf{Долговечность:} Изменения, произведённые транзакцией, сохраняются даже в случае сбоя системы.
\end{itemize}

\section*{Распределённые базы данных}
Распределённые базы данных организуют хранение данных на нескольких компьютерах, что обеспечивает масштабируемость и надёжность системы. Основные требования включают:
\begin{itemize}
    \item \textbf{Управление фрагментацией и дублированием данных:} Гарантирует оптимальное распределение данных между узлами.
    \item \textbf{Независимость распределения:} Пользователи не должны знать о физическом расположении данных.
    \item \textbf{Восстановление:} Обеспечивает целостность данных при сбоях.
\end{itemize}

\section*{Восстановление и журналирование}
Для обеспечения целостности и надёжности данных применяются следующие методы:
\begin{itemize}
    \item \textbf{Контрольные журналы:} Фиксируют изменения данных для возможности их последующего анализа и восстановления.
    \item \textbf{Резервное копирование:} Позволяет восстанавливать данные до состояния, предшествующего сбою.
    \item \textbf{Реорганизация физической памяти:} Выполняется для оптимизации хранения данных при изменениях структуры базы.
\end{itemize}

\subsubsection{ГОСТ Р 43.0.11 -- 2014}
\paragraph{ГОСТ Р 43.0.11 -- 2014}

\emph{\href{https://internet-law.ru/gosts/gost/57862/}{ГОСТ Р 43.0.11 -- 2014}
Информационное обеспечение техники и операторской деятельности.
Базы данных в технической деятельности
}
\par
ТК 379 «Информационное обеспечение
техники и операторской деятельности»
\section*{Общее описание}
ГОСТ Р 43.0.11-2014 устанавливает основные требования и принципы обеспечения совместимости информационных систем, определяя правила и процедуры для их взаимодействия. Стандарт ориентирован на повышение эффективности информационного обмена между различными системами.

\section*{Область применения}
Стандарт применяется к информационным системам, функционирующим в различных отраслях. Основные цели использования:
\begin{itemize}
    \item Определение правил совместимости компонентов информационных систем.
    \item Обеспечение корректного обмена данными между разнородными системами.
    \item Повышение надёжности и устойчивости систем к сбоям.
\end{itemize}

\section*{Термины и определения}
ГОСТ Р 43.0.11-2014 включает следующие ключевые термины:
\begin{itemize}
    \item \textbf{Информационная система (ИС):} Совокупность программных, аппаратных и организационных средств, предназначенных для обработки данных.
    \item \textbf{Совместимость:} Способность компонентов системы взаимодействовать для достижения заданной функциональности.
    \item \textbf{Интерфейс:} Средство взаимодействия между компонентами системы или между системами.
    \item \textbf{Протокол:} Набор правил для обмена данными между взаимодействующими компонентами.
\end{itemize}

\section*{Основные положения стандарта}

\subsection*{Совместимость систем}
Стандарт определяет совместимость как обязательное свойство информационных систем, достигаемое путём унификации процедур взаимодействия. Это включает:
\begin{itemize}
    \item Установление общих требований к интерфейсам.
    \item Определение единых форматов данных.
    \item Разработку протоколов обмена данными.
\end{itemize}

\subsection*{Модели взаимодействия}
ГОСТ Р 43.0.11-2014 предлагает модели взаимодействия систем, направленные на обеспечение их совместимости. Основные модели:
\begin{itemize}
    \item \textbf{Модель клиент-сервер:} Одна система (клиент) запрашивает данные или услуги у другой системы (сервер).
    \item \textbf{Модель однорангового взаимодействия:} Системы функционируют на равных правах, обмениваясь данными без центрального управляющего звена.
    \item \textbf{Модель шина данных:} Интеграция через централизованный канал, обеспечивающий передачу данных между системами.
\end{itemize}

\subsection*{Интерфейсы и протоколы}
Для обеспечения корректного взаимодействия между компонентами стандартизированы:
\begin{itemize}
    \item \textbf{Интерфейсы:} Должны быть открытыми и документированными, чтобы исключить проблемы интеграции.
    \item \textbf{Протоколы:} Определяют формат и последовательность действий для передачи данных.
\end{itemize}

\subsection*{Управление доступом и безопасностью}
Для защиты данных и обеспечения их конфиденциальности в стандарте предусматриваются следующие меры:
\begin{itemize}
    \item Использование механизмов идентификации и аутентификации пользователей.
    \item Контроль доступа к ресурсам системы на основе заданных политик.
    \item Шифрование данных при их передаче между системами.
\end{itemize}

\section*{Требования к данным}
ГОСТ Р 43.0.11-2014 предъявляет строгие требования к качеству и формату данных:
\begin{itemize}
    \item \textbf{Целостность данных:} Обеспечивается за счёт механизмов контроля ошибок и резервирования.
    \item \textbf{Актуальность:} Данные должны своевременно обновляться для обеспечения достоверности.
    \item \textbf{Единообразие:} Форматы данных и кодировки должны быть согласованы между системами.
\end{itemize}

\section*{Обеспечение надёжности}
Для обеспечения надёжной работы систем стандарт предусматривает:
\begin{itemize}
    \item \textbf{Дублирование ключевых компонентов:} Уменьшает риск отказа системы.
    \item \textbf{Резервное копирование данных:} Позволяет восстановить систему в случае сбоя.
    \item \textbf{Мониторинг и диагностика:} Обеспечивают контроль за состоянием системы и выявление неисправностей.
\end{itemize}

\section*{Применение стандарта}
ГОСТ Р 43.0.11-2014 рекомендуется использовать при разработке новых и модернизации существующих информационных систем. Его применение способствует:
\begin{itemize}
    \item Повышению уровня совместимости и интеграции между системами.
    \item Уменьшению затрат на разработку и сопровождение систем.
    \item Созданию надёжной и безопасной инфраструктуры для обработки данных.
\end{itemize}