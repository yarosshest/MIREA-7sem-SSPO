\section{Комитеты и подкомитеты}

Технический комитет (ТК) по стандартизации --- это объединение специалистов,
являющихся полномочными представителями заинтересованных предприятий
(организаций) --- членов ТК,
создаваемое на добровольной основе для разработки национальных стандартов РФ,
проведения работ в области международной (региональной)
стандартизации по закрепленным за ТК объектам стандартизации
(областям деятельности).

\begin{longtable}{|p{2cm}|p{14cm}|}
    \caption{Комитеты} \label{table:tk} \\
    \hline
    \textbf{\No\ ТК}
    & \textbf{Наименование ТК} \\
    \hline
    \endfirsthead
    \conttable{table:tk} \\
    \hline
    \textbf{\No\ ТК}
    & \textbf{Наименование ТК} \\
    \hline
    \endhead
    \textbf{022} & Информационные технологии \\ \hline
    \textbf{026} & Криптографическая защита информации \\ \hline
    \textbf{191}
    & Научно-техническая информация, библиотечное и издательское дело \\ \hline
    \textbf{362}
    & Защита информации \\ \hline
    \textbf{379} & Информационное обеспечение техники и операторской деятельности \\ \hline
    \textbf{482}
    & Поддержка жизненного цикла продукции \\ \hline
    \textbf{700}
    & Математическое моделирование
    и высокопроизводительные вычислительные технологии \\ \hline
\end{longtable}

\subsection{Подкомитеты ТК 22 Информационные технологии}

\begin{longtable}{|p{2cm}|p{14cm}|}
    \caption{Подкомитеты ТК 22 Информационные технологии}
    \label{table:tk:22} \\
    \hline
    \textbf{\No\ ПК}
    & \textbf{Наименование ПК} \\
    \hline
    \endfirsthead
    \conttable{table:tk:22} \\
    \hline
    \textbf{\No\ ПК}
    & \textbf{Наименование ПК} \\
    \hline
    \endhead
    ПК107 (SC7) & Системная и программная инженерия \\ \hline
    ПК122 (SC22)
    & Языки программирования, их окружение
    и системы программных интерфейсов \\ \hline
    ПК125 (SC25)
    & Взаимосвязь оборудования для информационных технологий \\ \hline
    ПК127 (SC27) & Безопасность информационных технологий \\ \hline
    ПК128 (SC28) & Оборудование офисов \\ \hline
    ПК134 (SC34) & Описание документа и языки обработки \\ \hline
    ПК135 (SC35) & Пользовательские интерфейсы \\ \hline
    ПК138 (SC38) & Платформы и сервисы для распределенных приложений \\ \hline
    ПК140 (SC40)
    & Управление информационными технологиями и услугами ИТ \\ \hline
    ПК201 & Терминология в ИТ \\ \hline
    ПК206 & Интероперабельность \\ \hline
\end{longtable}


\begin{longtable}{|p{2cm}|p{14cm}|}
    \caption{Подкомитеты ТК 26 Криптографическая защита информации}
    \label{table:tk:26} \\
    \hline
    \textbf{\No\ ПК}
    & \textbf{Наименование ПК} \\
    \hline
    \endfirsthead
    \conttable{table:tk:26} \\
    \hline
    \textbf{\No\ ПК}
    & \textbf{Наименование ПК} \\
    \hline
    \endhead
    ПК1 & Криптографические алгоритмы и протоколы для применения в поставляемых для федеральных государственных
    нужд шифровальных (криптографических) средствах защиты информации, содержащей сведения, составляющие государственную тайну \\ \hline
    ПК2
    & Криптографические алгоритмы и протоколы для применения в поставляемых для федеральных государственных нужд
    шифровальных (криптографических) средствах защиты информации, содержащей сведения, относимые к охраняемой в
    соответствии с законодательством Российской Федерации информации ограниченного доступа \\ \hline
    ПК3
    & Криптографические алгоритмы и механизмы в национальной платежной системе Российской Федерации \\ \hline
    ПК4 & Российские шифровальные (криптографические) средства защиты информации, не содержащей сведений, составляющих
    государственную тайну, или относимых к охраняемой в соответствии с законодательством Российской Федерации к
    информации ограниченного доступа, а также зарубежные шифровальные (криптографические) средства защиты информации
    на территории Российской Федерации \\ \hline
\end{longtable}