\section*{\LARGE Цель практической работы}
\addcontentsline{toc}{section}{Цель практической работы}

\textbf{Цель практической работы:}
Получить навыки по анализу основных характеристик и атрибутов
качества программных продуктов в соответствии с установленными
стандартами ГОСТ Р ИСО/МЭК 9126 и ГОСТ Р ИСО/МЭК 25010-2015.

\textbf{Задание на практическую работу:}

\begin{itemize}
    \item Выбрать программный продукт для анализа. В качестве программного
    продукта выступает тема дипломного проекта.
    \item Ознакомиться со стандартами
    ГОСТ Р ИСО/МЭК 9126 и ГОСТ Р ИСО/МЭК 25010-2015.
    \item Выбрав один из вышеперечисленных стандартов, оценить
    возможность соответствия программного продукта всем характеристикам
    качества согласно стандарту и сформулировать несколько требований к
    качеству программного продукта, которое должно быть удовлетворено в
    соответствии с данной характеристикой.
\end{itemize}

\clearpage

\section*{\LARGE Выполнение практической работы}
\addcontentsline{toc}{section}{Выполнение практической работы}

ISO/IEC 9126 --- международный стандарт, определяющий оценочные
характеристики качества программного обеспечения. Он состоит из 4-х частей
(модель качества, внешние метрики, внутренние метрики и применение
метрик). Российским аналогом является стандарт ГОСТ Р ИСО/МЭК 9126-93
<<Информационная технология. Оценка программной продукции.
Характеристики качества и руководства по их применению>>.
В 2011 году ISO/IEC 9126 был заменен стандартом ISO/IEC 25010:2011
(идентичный аналог ГОСТ Р ИСО/МЭК 25010-2015 <<Информационные
технологии. Системная и программная инженерия. Требования и оценка
качества систем и программного обеспечения (SQuaRE). Модели качества
систем и программных продуктов>>).

Требования к качеству программного продукта
<<Мобильное приложение рекомендации фильмов>> по стандарту ГОСТ Р ИСО/МЭК 9126-93
продемонстрированны в таблице~\ref{table}.

\begin{longtable}{|p{3cm}|p{3cm}|p{5cm}|p{5cm}|}
    \caption{Требования к качеству мобильного приложения для рекомендаций фильмов} \label{table} \ \hline \textbf{Характе-ристика} & \textbf{Атрибут} & \textbf{Требование 1} & \textbf{Требование 2} \ \hline \endfirsthead
    \conttable{table} \\
    \hline
    \textbf{Характе-ристика}
    & \textbf{Атрибут}
    & \textbf{Требование 1}
    & \textbf{Требование 2} \\
    \hline
    \endhead

    \textbf{Функцио-нальность}
    & Пригодность
    & Приложение должно поддерживать рекомендательные функции
    и выводить список фильмов на основе предпочтений пользователя (Python, FastAPI для API).
    & Приложение должно показывать персонализированные рекомендации
    на основе истории и оценок пользователя (используется PostgreSQL и ML-модель на основе библиотеки scikit-learn). \\ \hline

    & Правильность
    & Все рекомендации должны соответствовать выбору категорий
    и жанров пользователя, точность — не менее 85\% (ML-модель с использованием алгоритма Decision Tree).
    & Приложение должно корректно отображать информацию
    о каждом фильме, включая описание, рейтинг и трейлер (данные из API, такие как The Movie Database API). \\ \hline

    & Способность к взаимодействию
    & Приложение должно интегрироваться с системой кинопоиска для автоматического обновления данных о фильмах, рейтингах и рецензиях (через API, такие как API Кинопоиска).
    & Интерфейс должен быть адаптирован для взаимодействия с Rest API и синхронизироваться с базой данных (PostgreSQL, через библиотеку Dio для Flutter). \\ \hline

    & Согласова-нность
    & Данные о фильмах должны соответствовать спецификациям API и быть обновлены каждые 24 часа (обновление базы данных через Celery).
    & Интерфейс должен оставаться последовательным и интуитивно понятным на всех страницах (использование Material Design для UI). \\ \hline

    & Защищенность (безопасность)
    & Приложение должно хранить данные пользователя безопасно, используя методы шифрования (AES-256).
    & Личные данные пользователей не должны передаваться третьим лицам без разрешения (OAuth 2.0 для авторизации). \\ \hline

    \textbf{Надежность}
    & Стабильность
    & Приложение должно работать без сбоев при нормальной загрузке (до 500 активных пользователей в реальном времени).
    & При большем количестве пользователей (до 5000) приложение должно оставаться стабильным и не допускать аварийных завершений (использование Firebase для мониторинга и Crashlytics для отчетов об ошибках). \\ \hline

    & Устойчивость к ошибке
    & Приложение должно корректно обрабатывать ошибки сети, показывая сообщения для пользователя (использование плагина Connectivity для Flutter).
    & Все ошибки должны логироваться для дальнейшего анализа и устранения (логирование в Sentry или Firebase Analytics). \\ \hline

    & Восстанавли-ваемость
    & Приложение должно сохранять сессию пользователя при кратковременных отключениях (использование Shared Preferences для сохранения данных сессии).
    & В случае длительных сбоев приложение должно предоставлять опцию восстановления сессии при перезапуске. \\ \hline

    \textbf{Практич-ность (удобство использования)}
    & Понятность
    & Интерфейс приложения должен быть интуитивно понятным и включать простые элементы навигации (использование навигации Flutter Material PageRoute).
    & Основные функции должны быть доступны за 2-3 шага от главного экрана. \\ \hline

    & Обучаемость
    & Пользователи должны освоить основные функции приложения менее чем за 10 минут.
    & Приложение должно предлагать краткие подсказки для новых пользователей (добавление туториала на Flutter с помощью пакета CoachMark). \\ \hline

    & Простота использования
    & Запуск рекомендаций и фильтрация должны выполняться в 2-3 клика.
    & Приложение должно запоминать предпочтения для улучшения удобства последующих сессий (использование SQLite для хранения локальных данных). \\ \hline

    \textbf{Эффектив-ность (производительность)}
    & Характер изменения по времени
    & Загрузка рекомендаций не должна занимать более 2 секунд на запрос (кеширование запросов с помощью Redis).
    & Скорость отклика должна оставаться стабильной при увеличении числа пользователей до 1000. \\ \hline

    & Характер изменения ресурсов
    & Приложение должно потреблять не более 200 МБ ОЗУ при выполнении основных операций.
    & Использование процессора не должно превышать 15\% при выполнении базовых функций. \\ \hline

    \textbf{Сопровожда-емость}
    & Анализиру-емость
    & Исходный код должен быть задокументирован и соответствовать стандартам PEP 8 (для Python).
    & Приложение должно создавать отчеты о производительности для анализа (интеграция с Google Analytics для Firebase). \\ \hline

    & Изменяемость
    & Код должен быть модульным и легко изменяемым для добавления новых функций.
    & Изменения не должны влиять на текущий функционал (использование системы версионного контроля Git). \\ \hline

    & Устойчивость
    & Приложение должно оставаться стабильным после обновлений функционала.
    & Новые функции должны проходить тестирование на совместимость с текущими версиями (проведение тестирования с помощью Flutter Test и CI/CD). \\ \hline

    & Тестируемость
    & Приложение должно поддерживать возможность тестирования на Android и iOS.
    & Должны быть предусмотрены автоматические тесты для всех основных функций (использование Flutter и Firebase Test Lab). \\ \hline

    \textbf{Мобильность}
    & Адаптиру-емость
    & Приложение должно корректно работать на Android и iOS, поддержка разрешений экрана от 720x1280 до 1440x3040 пикселей (Flutter для кроссплатформенности).
    & Приложение должно быть легко адаптируемо к различным разрешениям экранов (использование пакета responsive\_framework для Flutter). \\ \hline

    & Простота внедрения
    & Процесс установки должен занимать не более 5 минут и включать пошаговые инструкции.
    & Обновления должны быть автоматизированы и включать проверку совместимости устройства (доставка обновлений через App Store и Google Play). \\ \hline

    & Соответствие
    & Приложение должно соответствовать требованиям Postgres Pro Standard при взаимодействии с бд.
    & Приложение должно соответствовать 3 версии Docker Compose. \\ \hline

    & Взаимозаме-няемость
    & Приложение должно поддерживать возможность замены API без изменения функциональности (использование интерфейсов и абстракций).
    & Компоненты интерфейса должны быть легко заменяемыми, позволяя обновлять внешний вид. \\ \hline

\end{longtable}

\clearpage

\section*{\LARGE Вывод}
\addcontentsline{toc}{section}{Вывод}

В ходе выполнения практической работы был проведен анализ
программного продукта по темой дипломной работы
с опорой на стандарт ГОСТ Р ИСО/МЭК 9126,
который описывает основные характеристики качества программного обеспечения.
Продукт был оценен по таким характеристикам, как
функциональность, надежность, удобство использования, производительность,
сопровождаемость и переносимость.
В результате анализа были сформулированы требования к качеству.
Это позволит повысить общее качество программного продукта
и обеспечить его соответствие международным стандартам.

