\section*{\LARGE Цель практической работы}
\addcontentsline{toc}{section}{Цель практической работы}

\textbf{Цель практической работы:}
Приобретение навыков в области документирования характеристик
программного обеспечения посредством разработки
паспорта программного продукта.

\textbf{Задание на практическую работу:}

\begin{itemize}
    \item Выбрать программный продукт. В качестве программного продукта
    выступает тема дипломного проекта.
    \item Разработать технический паспорт программного продукта,
    включающий ключевые разделы.
\end{itemize}

\clearpage

\section*{\LARGE Выполнение практической работы}
\addcontentsline{toc}{section}{Выполнение практической работы}

\section{Общие сведения о программном продукте}
\subsection{Наименование программного продукта}

\textbf{Полное наименование программного продукта}:
Рекомендательная система фильмов, основанная на личных оценках пользователя с мобильным клиентом.\par
\textbf{Краткое наименование программного продукта}:
Мобильное приложение рекомендации фильмов

\subsection{Назначение программного продукта,
    его основные функции и области применения}
Программный продукт, представляющий собой мобильное приложение рекомендации фильмов, предназначен для реализации
экспериментальной системы рекомендации.
Использующий индивидуальные предпочтения без использования дополнительных личных данных.
Основная цель --- обеспечить рекомендации, не имеющие аналогов у больших рекомендательных систем.

Основная функция заключается в предоставлении личных рекомендаций для пользователей.

Области применения:

\begin{itemize}
    \item Исследование возможностей языковых моделей в рекомендации по тексту;
    \item Исследование нестандартного алгоритма рекомендации в системе Пользователь -- Предмет;
    \item Улучшение пользовательского опыта стримминговых сервисов;
\end{itemize}
