
\section{Требования}\label{sec:treb}
\subsection{требования к структуре АС в целом}\label{subsec:treb:ob}
\subsubsection{Требования к способам и средствам обеспечения информационного взаимодействия компонентов}
Взаимодействие между базой данных и API должно происходить путем прямого подключения через драйвер базы данных.

Взаимодействие между модулем рекомендации может происходить как по следующим средствам: брокер сообщений, RESTapi,
внутри программное взаимодействие.

Взаимодействие между API и мобильным приложением должно происходить через RESTapi и веб-сокеты.

\subsubsection{Требования к характеристикам взаимосвязей создаваемой АС со смежными АС}
Смежными системами к данной являются:
\begin{itemize}
    \item PostgresSQL;
    \item Модели по векторизации текстов;
    \item Решающие деревья рекомендации.
\end{itemize}

Данные системы будут подключаемыми программными модулями для API и модуля рекомендации.
Они должны быть совместимы с целевым языком программирования данных систем: Python 3.9

\subsubsection{Требования к режимам функционирования}

У приложения должно быть 2 режима функционирования:
\begin{itemize}
    \item С рекомендательной системой;
    \item С отключенной рекомендательной системой.
\end{itemize}

В связи со специфичностью работы модуля рекомендации разрабатываемой системы.
Возможны случаи когда система не сможет обрабатывать новые запросы по рекомендации в связи с загруженностью системы.
В данном режиме пользователь должен получить оповещение о проблеме и о том что его рекомендации появиться позже.

\subsubsection{Перспективы развития, модернизации АС}
В будущем возможна создание интерфейса для подключения как к другим систем, так и общения других решений с
разрабатываемой системой.

\subsection{Требования к функциям}\label{subsec:func:treb}

\subsubsection{Время отклика приложения}
Ответ на каждое действие пользователя должно быть не более 30 секунд.
После истечения 30 секунд пользователь должен получить уведомление о проблемах на стороне сервиса.

\subsubsection{Регистрация}
Пользователь должен иметь возможность зарегистрироваться в приложении
Для регистрации используются такие данные:
\begin{itemize}
    \item Уникальный логин;
    \item Пароль длиною более 6 символов.
\end{itemize}
Если данный логин уже используется, то пользователь должен быть уведомлен об этом и мобильное приложение должно
попросить ввести новый уникальный логин.

\subsubsection{Авторизация}
Пользователь должен иметь возможность авторизоваться в приложении.
Для авторизации используются такие данные:
\begin{itemize}
    \item Уникальный логин;
    \item Пароль длиною более 6 символов.
\end{itemize}
Если после ввода данных для аутентификации прошло более 30 секунд и не пришел от сервера, пользователь должен быть
оповещен об ошибке на стороне сервера.

\subsubsection{Поиск фильмов}
Пользователь должен иметь возможность искать фильмы в приложении, по их названию.
Для поиска используется строка длинной не более 512 символов.
Если фильмов не нашлось, пользователь должен получить информацию о том, что фильмов по его запросу нет.

\subsubsection{Просмотр фильмов}
Пользователь должен иметь возможность просматривать информацию о фильме.
А именно:
\begin{itemize}
    \item Фото;
    \item Описание;
    \item Оценка пользователя.
\end{itemize}

\subsubsection{Получение рекомендаций}
Пользователь должен иметь возможность получать рекомендации.
Количество рекомендаций в день может быть ограниченно до 1 в день из-за нагрузки на сервер.
Пользователь должен был оценить как минимум 2 фильма положительно, 2 фильма отрицательно.

\subsubsection{Управление фильмами}
Система должна предоставить возможность для редактирования, добавления, удаления фильма и его параметров.

\subsubsection{Управление доступностью системы рекомендации}
Система должна предоставить возможность для отключения и включения системы рекомендации.

\subsubsection{Обновление библиотеки фильмов}
Система должна предоставить возможность для сбора фильмов из кинопоиска и пред обработки для записи в базу данных.

\subsection{Требования к видам обеспечения}
\subsubsection{Математическое обеспечение}
Система рекомендации должна быть разработанная по данному принципу:
Короткое описание фильма векториризируеться, данный вектор обогащается дополнительными данными о фильме.

Для рекомендации должен использоваться лес решающих деревьев, который будет уникальным для каждого пользователя.
Он будет модифицироваться при каждом запросе рекомендации от пользователя.

\subsubsection{Информационное обеспечение}
В качестве базы данных должна использоваться СУБД PostgresSQL\@.
Формат обмена сообщений между клиентом Android и API должен быть описан в Swagger документе.


\subsubsection{Лингвистическое обеспечение}
Мобильное приложение должно быть написано на языке Kotlin для платформы Android.
api, система рекомендаций и парсинг должен быть написан на языке Python.
Интерфейс должен быть выполнен на русском языке.
Android api должно быть 35 версии.

\subsubsection{Техническое обеспечение}
Северная часть системы должно разворчится на сервере с 4 ядрами и 16 гигабайтами оперативной памяти на ОС Linux внутри
Docker.


Мобильное приложение должно запускть на устройстве с ОС Android не старше 9.0. С оперативной памятью не менее
4 гигабайт.


\subsection{Общие технические требования к АС}
\subsubsection{Требования к численности и квалификации персонала и пользователей АС}
Системой должен управлять 2 системных администратора уровня Middle.
Так же 1 программист уровня Senior.


\subsubsection{Требования к показателям назначения}
Система должна поддерживать 200 одновременно работающих в системе пользователей.
Система должна поддерживать 1000 одновременно выполняемых запросов к серверу.
Система должна поддерживать 1000 одновременно выполняемых запросов к серверу.

\subsubsection{Требования к надежности}
Система должна работать 90\% времени в день.

\subsubsection{Требования к безопасности}
Пароли пользователей должны храниться в зашифрованном виде.
