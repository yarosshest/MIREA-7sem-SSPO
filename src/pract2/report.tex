\section*{\LARGE Цель практической работы}
\addcontentsline{toc}{section}{Цель практической работы}

\textbf{Цель практической работы:}
Приобретение навыков в области документирования характеристик
программного обеспечения посредством разработки
паспорта программного продукта.

\textbf{Задание на практическую работу:}

\begin{itemize}
	\item Выбрать программный продукт. В качестве программного продукта
		выступает тема дипломного проекта.
	\item Разработать технический паспорт программного продукта,
		включающий ключевые разделы.
\end{itemize}

\clearpage

\section*{\LARGE Выполнение практической работы}
\addcontentsline{toc}{section}{Выполнение практической работы}

\section{Общие сведения о программном продукте}

\subsection{Наименование программного продукта}

\textbf{Полное наименование программного продукта}:
Программа-конвертер DRC правил для использования
в открытых инструментах проектирования цифровых микросхем\par
\textbf{Краткое наименование программного продукта}:
Программа-конвертер DRC правил

\subsection{Назначение программного продукта,
	его основные функции и области применения}

Программный продукт, представляющий собой конвертер DRC правил,
предназначен для преобразования правил проектирования
из проприетарных форматов (SVRF Calibre) в форматы, совместимые
с открытыми инструментами проектирования цифровых микросхем (KLayout).
Основная цель --- обеспечить удобство
и доступность работы с DRC правилами для разработчиков и инженеров,
работающих в области проектирования интегральных схем.\par
Основная функция заключается в преобразования DRC правил
из проприетарных форматах в форматы,
поддерживаемые открытыми инструментами.

Области применения:

\begin{itemize}
	\item Использование в процессе проектирования интегральных схем,
		где необходимо соблюдать DRC правила
		для корректного функционирования и производства.
	\item Интеграция в существующие инструменты проектирования
		для автоматизации работы с DRC правилами и улучшения совместимости.
	\item Применение в учебных заведениях
		и исследовательских лабораториях для обучения студентов
		и специалистов работе с правилами проектирования.
	\item Помощь компаниям, желающим перейти
		на открытые инструменты проектирования,
		в конвертации существующих данных для обеспечения совместимости.
\end{itemize}

\section{Перечень модулей и слоев, входящих в состав программного продукта}

\subsection{Модуль пользовательского интерфейса}
Обработывает параметры командной строки.
Обрабатывает конфигурационный файлы,
определяющий параметры работы программы,
пути к исходным данным, настройкам преобразования и т.д.
И возвращает результат преобразования в виде файла или вывода в консоль,
в зависимости от настроек.

\subsection{Модуль парсинга исходного DRC файла}
Отвечает за чтение исходного файла DRC Calibre
и преобразование его с помощью библиотеки для парсинга Lark
в предоставляемый им структуру данных для дальнейшей конвертации.

\subsection{Модуль предобработки}
Подготавливает данные к конвертации.
Он объединяет многострочные команды для корректного парсинга,
разворачивает макросы, содержашиеся в коде, а также
создает корректную последовательность команд в коде.

\subsection{Модуль конвертации}
Основной модуль, который выполняет преобразование команд DRC Calibre
в формат KLayout.

\subsection{Модуль генерации отчета}
Модуль, генерирующий отчет о конвертации DRC Calibre на основе информации,
полученной на стадии конвертации.

\section{Архитектура программного продукта}

Основной принцип архитектуры программы-конвертера DRC правил
--- \textbf{модульность}.
Программа состоит из отдельных компонентов (модулей),
которые отвечают за конкретные функции:
парсинг, конвертация, предобработка и взаимодействие с пользователем.
Это позволяет легко изменять или добавлять новые функции
без необходимости переписывать всю систему.\par
Также в программном продукте используется принцип
\textbf{разделения ответственности} 
(Separation of Concerns или SoC), где каждому модулю назначена
своя ответственность, что снижает взаимозависимость между компонентами.
Это делает систему более поддерживаемой и упрощает тестирование и отладку.

\section{Функциональные возможности модулей и слоев}

\subsection{Модуль пользовательского интерфейса}
\begin{itemize}
	\item Чтение DRC файл;
	\item Чтение грамматики;
	\item Чтение конфигурационного файл;
	\item Формирование правил и настроек конвертора.
\end{itemize}

\subsection{Модуль парсинга исходного DRC файла}
\begin{itemize}
	\item Парсинг DRC файла;
	\item Преобразование кода во внутреннее представление.
\end{itemize}

\subsection{Модуль предобработки}
\begin{itemize}
	\item Конкатенация многострочных команд;
	\item Развертывание макросов;
	\item Преобразование внутреннего представления кода в ориентированный граф;
	\item Топологическая сортировка;
	\item Преобразование графа во внутреннее представление кода.
\end{itemize}

\subsection{Модуль конвертации}
\begin{itemize}
	\item Конвертация правил;
	\item Конвертация декларативных команд;
	\item Конвертация команд геометрических проверок;
	\item Объединение кода.
\end{itemize}

\subsection{Модуль генерации отчета}
\begin{itemize}
	\item Обработка информации о трансляции;
	\item Создание отчета в Markdown;
	\item Создание отчета в Latex.
\end{itemize}


\section{Перечень входной и выходной информации}

\subsection{Описание форматов входных и выходных данных}

Входные данные:

\begin{itemize}
	\item Путь до файла правил DRC,
		созданный для коммерческих инструментов (формат SVRF Calibre).
	\item Путь до файла конфигурации.
\end{itemize}

Выходные данные:

\begin{itemize}
	\item Правила DRC, преобразованный для использования
		в открытых инструментах (в формате Klayout).
	\item Отчет о результатах трансляции, в формате LaTex, Markdown или txt.
\end{itemize}


\subsection{Протоколы и интерфейсы взаимодействия с другими системами}

Основной метод передачи данных между программным продуктом
и внешними инструментами проектирования осуществляется через обмен файлами.
Поддержка файловых систем и форматов критична для работы с данными.

\section{Требования к аппаратно-программному обеспечению}

\subsection{Требования к аппаратной части}

Минимальные характеристики оборудования:

\begin{itemize}
	\item Процессор (CPU):
		Двухъядерный процессор с тактовой частотой не менее 2 ГГц
	\item Оперативная память (RAM):
		4 ГБ оперативной памяти для работы с небольшими файлами
		и основными функциями программы.
	\item Дисковое пространство:
		500 МБ свободного дискового пространства для установки программы
		и временных файлов.
		2 ГБ для хранения логов и выходных файлов в процессе работы.
\end{itemize}

Рекомендуемые характеристики оборудования:

\begin{itemize}
	\item Процессор (CPU):
		Четырехъядерный процессор с тактовой частотой 3 ГГц и выше.
	\item Оперативная память (RAM):
		8 ГБ или более для работы с большими объемами данных
		и увеличенной производительностью.
	\item Дисковое пространство:
		5 ГБ свободного пространства для установки, полученных скриптов
		и отчетов о трансляции.
		10 ГБ или более для работы с крупными проектами.
\end{itemize}

\subsection{Требования к программной части}

\subsubsection{Необходимое программное обеспечение}

Программа совместима со всеми \textbf{операционными системами},
поддерживающие Python 3.9 выше.

Программа написана на Python
и требует установленного \textbf{интерпретатора} Python 3.9
или выше для выполнения.

Для работы программы необходимо установить
несколько \textbf{Python-библиотек}:
\begin{itemize}
	\item argparse:
		Для обработки командной строки (встроено в Python).
	\item configparse:
		Для обработки конфигурационного фала (встроено в Python).
	\item os и sys:
		Для взаимодействия с операционной системой (встроено в Python).
	\item Lark (версии 1.2.2 и выше):
		Для парсинга DRC Calibre.
	\item NetworkX (версии 3.3 и выше):
		Для предобработки исходного файла правил,
		создания корректной последовательности команд.
\end{itemize}

\textbf{Дополнительные инструменты:}

\begin{itemize}
	\item Git: Если программа интегрируется в рабочие процессы
		с использованием системы контроля версий,
		Git может быть полезен для управления проектами
		и конфигурационными файлами.
	\item KLayout (опционально) версии 0.29.5 выше:
		Инструменты верификации для работы
		с выходными DRC правилами в рамках проектирования схем.
\end{itemize}

\subsubsection{Требования к сетевой инфраструктуре и пропускной способности}

Для работы программы не требуется подключение к сети,
так как вся обработка данных и конвертация происходят на локальной машине.

\clearpage

\section*{\LARGE Вывод}
\addcontentsline{toc}{section}{Вывод}

В ходе выполнения практической работы были приобретены
и закреплены навыки документирования характеристик программного обеспечения,
а также создан паспорт программного продукта.
Этот документ позволяет структурировано и
полно описать программное обеспечение, его назначение, основные функции,
архитектуру, интерфейсы, требования к аппаратным и программным средствам,
а также особенности эксплуатации и сопровождения.

Создание паспорта программного продукта способствует
более глубокому пониманию структуры ПО, облегчает его дальнейшую разработку,
сопровождение и поддержку,
а также упрощает взаимодействие между разработчиками,
пользователями и техническими специалистами.

