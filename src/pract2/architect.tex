\section{Архитектура программного продукта}

Для реализации серверной части проекта была использована контейнеризация, что обеспечивает удобство развертывания
и масштабирования приложения.
Серверная часть состоит из двух Docker контейнеров:

Контейнер для API:
\begin{itemize}
    \item FastAPI: используется для создания высокопроизводительного веб-API. FastAPI позволяет быстро и удобно реализовывать эндпоинты для обработки запросов и предоставления ответов.
    \item SQLAlchemy: используется для взаимодействия с базой данных PostgreSQL. Эта библиотека обеспечивает ORM (Object-Relational Mapping), что упрощает работу с базой данных на уровне кода.
\end{itemize}

Контейнер для базы данных: PostgreSQL: реляционная база данных, используемая для хранения всей необходимой
информации о пользователях, фильмах и их оценках.
PostgreSQL известна своей надежностью и производительностью, что делает её отличным выбором для данного проекта.


\begin{image}
    \includegrph{deploy}
    \caption{Диограмма развертки}
    \label{fig:deploy}
\end{image}
При запуске системы Docker управляет обоими контейнерами, обеспечивая их взаимодействие.
Контейнер с FastAPI обрабатывает HTTP-запросы, поступающие от клиентов, и использует SQLAlchemy для выполнения
запросов к базе данных, которая находится в контейнере с PostgreSQL.
