\section*{\LARGE Цель практической работы}
\addcontentsline{toc}{section}{Цель практической работы}

\textbf{Цель практической работы:}
Анализ и классификация выбранных стандартов информационных
технологий для оценки их актуальности, применимости и соответствия
современным требованиям стандартизации.

\textbf{Задачей} практической работы является приобретение навыков:
\begin{itemize}
	\item идентификации органа стандартизации для каждого выбранного
		стандарта и анализ их принадлежности;
	\item классификации стандартов по кодам, категориям и областям
		стандартизации;
	\item оценки актуальности стандартов и выявления элементов, требующих
		обновления;
	\item определения сфер применения каждого стандарта и их соответствие
		современным требованиям отрасли.
\end{itemize}

\clearpage

\section*{\LARGE Выполнение практической работы}
\addcontentsline{toc}{section}{Выполнение практической работы}

В практической работе необходимо выбрать
минимум по 2 стандарта из разных категорий, таких как:

\begin{enumerate}
	\item международные (Таблица \ref{table:international});
	\item национальные (Таблица \ref{table:national});
	\item межгосударственные (Таблица \ref{table:interstate});
	\item национальные,
		идентичные международным (Таблица \ref{table:national:international});
	\item национальные,
		модифицированные по отношению к международным
		(Таблица \ref{table:national:international:mod}).
\end{enumerate}

Это могут быть стандарты, касающиеся различных этапов жизненного
цикла разработки ПО, стандарты управления качеством, стандарты
информационной безопасности, стандарты по облачным вычислениям,
стандарты баз данных и т.д.

\begin{table}[h!tp]
	\centering
	\caption{\leftline{Международные}}
	\label{table:international}
	\begin{tabular}{|p{10em}|p{12em}|p{12em}|}
		\hline
		\textbf{Обозначение}
			& \textbf{ISO/IEC 23894:2023} & \textbf{ISO/IEC 27017:2015} \\ \hline
		\textbf{Название стандарта}
				& Информационная технология. Искусственный интеллект. Руководство по менеджменту риска
				& Информационные технологии. Методы и средства обеспечения безопасности. Правила применения мер обеспечения информационной безопасности на основе ISO/IEC 27002 при использовании облачных служб \\ \hline
		\textbf{Индекс стандарта}
			& ISO/IEC & ISO/IEC \\ \hline
		\textbf{Регистрационный номер}
			& 23894 & 27017 \\ \hline
		\textbf{Номер комплексной системы стандартов}
			& 2 & 2 \\ \hline
		\textbf{Код ОКС}
			& 35.020 & 35.040 \\ \hline
		\textbf{Категория документа}
			& Международный & Международный \\ \hline
		\textbf{Организация по стандартизации}
			& Joint Technical Committee ISO/IEC JTC 1, Information technology, Subcommittee SC 42, Artificial intelligence. & Joint Technical Committee ISO/IEC JTC1,Information technology \\ \hline
		\textbf{Область применения в ИТ}
			& Искусственный интеллект & Методы и средства обеспечения безопасности \\ \hline
		\textbf{Объект стандартизации}
			& Руководство по менеджменту риска
			&  Правила применения мер обеспечения информационной безопасности на основе ISO/IEC 27002 при использовании облачных служб \\ \hline
		\textbf{Аспект стандартизации}
			& Принцы управления рисками связанных с  ИИ
			& Политики информационной безопасности \\ \hline
		\textbf{Последнее изменение} & - & 30.11.2021 \\ \hline
		\textbf{Связанные стандарты} & ISO 31000:2018 & ISO/IEC 27002 \\ \hline
	\end{tabular}
\end{table}

\begin{table}[h!tp]
	\centering
	\caption{\leftline{Национальные}}
	\label{table:national}
	\begin{tabular}{|p{10em}|p{12em}|p{12em}|}
		\hline
		\textbf{Обозначение}
			& \textbf{ГОСТ Р 34.11-2012} & \textbf{ГОСТ Р 34.30-93} \\ \hline
		\textbf{Название стандарта}
			& Информационная технология. Криптографическая защита информации. Функция хэширования
			& Информационная технология. Передача данных. Интерфейс между оконечным оборудованием и аппаратурой окончания канала данных и распределение номеров контактов соединителей. Общие требования \\ \hline
		\textbf{Индекс стандарта}
			& ГОСТ Р & ГОСТ Р \\ \hline
		\textbf{Регистрационный номер}
			& 34.11 & 34.30 \\ \hline
		\textbf{Номер комплексной системы стандартов}
			& 34 & 34 \\ \hline
		\textbf{Код ОКС}
			& 35.040 & 35.200 \\ \hline
		\textbf{Категория документа}
			& Национальный & Национальный \\ \hline
		\textbf{Организация по стандартизации}
			& Федеральное агентство по техническому регулированию и метрологии
			& ТК 22 Информационные технологии \\ \hline
		\textbf{Область применения в ИТ}
			& Криптографическая защита информации
			& Передача данных \\ \hline
		\textbf{Объект стандартизации}
			& Функция хэширования
			& Интерфейс между оконечным оборудованием и аппаратурой окончания канала данных и распределение номеров контактов соединителей \\ \hline
		\textbf{Аспект стандартизации}
			& Требования к содержанию документов
			& Формирование и проверка электронной цифровой подписи \\ \hline
		\textbf{Последнее изменение}
			& 01.01.2021 & 06.04.2015 \\ \hline
		\textbf{Связанные стандарты}
			& 34.11—94
			& ИСО 4902 ИСО 4903 ИСО 2110 \\ \hline
	\end{tabular}
\end{table}

\begin{table}[h!tp]
	\centering
	\caption{\leftline{Межгосударственные}}
	\label{table:interstate}
	\begin{tabular}{|p{10em}|p{12em}|p{12em}|}
		\hline
		\textbf{Обозначение}
			& \textbf{ГОСТ 34.201-2020} & \textbf{ГОСТ 34.321-96} \\ \hline
		\textbf{Название стандарта}
			& Информационные технологии. Комплекс стандартов на автоматизированные системы. Виды, комплектность и обозначение документов при создании автоматизированных систем
			& Информационные технологии. Система стандартов по базам данных. Эталонная модель управления данными \\ \hline
		\textbf{Индекс стандарта} & ГОСТ & ГОСТ \\ \hline
		\textbf{Регистрационный номер} & 34.201 & 34.321 \\ \hline
		\textbf{Номер комплексной системы стандартов} & 34 & 34 \\ \hline
		\textbf{Код ОКС} & 35.240 01.040.35 & 35.100.70 35.240 \\ \hline
		\textbf{Категория документа}
			& Межгосударственный & Межгосударственный \\ \hline
		\textbf{Организация по стандартизации}
			& Межгосударственный совет по стандартизации метрологии и сертификации
			& Межгосударственный Совет по стандартизации метрологии и сертификации \\ \hline
		\textbf{Область применения в ИТ}
			& Комплекс стандартов на автоматизированные системы
			& Система стандартов по базам данных \\ \hline
		\textbf{Объект стандартизации}
			& Виды, комплектность и обозначение документов при создании автоматизированных систем
			& Эталонная модель управления данными \\ \hline
		\textbf{Аспект стандартизации}
			& Виды и наименование документов, Комплектность документации, Обозначения документов
			& Термины и определения, Графические представления, Требования к управлению данными, Архитектурная модель \\ \hline
		\textbf{Последнее изменение} & 01.01.2024 & 19.06.2001 \\ \hline
		\textbf{Связанные стандарты}
			& ГОСТ 1.0  ГОСТ 1.2  ГОСТ 34.602
			& ISO/IEC 10032:1995 \\ \hline
	\end{tabular}
\end{table}

\begin{table}[h!tp]
	\centering
	\caption{\leftline{Национальные, идентичные международным}}
	\label{table:national:international}
	\begin{tabular}{|p{10em}|p{12em}|p{12em}|}
		\hline
		\textbf{Обозначение}
			& \textbf{ГОСТ Р 34.913.3-91 (ИСО 8802-3-89)}
			& \textbf{ГОСТ Р 34.701.1-92 (ИСО 8632-1-87)} \\ \hline
		\textbf{Название стандарта}
			& Информационная технология. Локальные вычислительные сети. Метод случайного доступа к шине и спецификация физического уровня
			& Информационная технология. Машинная графика. Метафайл для хранения и передачи информации об описании изображения \\ \hline
		\textbf{Индекс стандарта}
			& ГОСТ Р ИСО & ГОСТ Р ИСО \\ \hline
		\textbf{Регистрационный номер}
			& 34.913.3 & 34.701.1 \\ \hline
		\textbf{Номер комплексной системы стандартов}
			& 34 & 34 \\ \hline
		\textbf{Код ОКС}
			& 35.100.05 35.180 & 35.140 \\ \hline
		\textbf{Категория документа}
			& Нац = межд & Нац = межд \\ \hline
		\textbf{Организация по стандартизации}
			& Комитет стандартизации и метрологии cccp
			& Отдел стандартизации и сертификации информационных технологий, продукции электротехники и приборостроения \\ \hline
		\textbf{Область применения в ИТ}
			& Локальные вычислительные сети
			& Машинная графика \\ \hline
		\textbf{Объект стандартизации}
			& Метод случайного доступа к шине и спецификация физического уровня
			& Метафайл для хранения и передачи информации об описании изображения \\ \hline
		\textbf{Аспект стандартизации}
			& Метод доступа к физической среде
			& Особенности хранения и передачи метафайла \\ \hline
		\textbf{Последнее изменение}
			& 06.04.2015 & 11.01.1993 \\ \hline
		\textbf{Связанные стандарты}
			& ГОСТ 24402 ИСО 2382/25
            & ГОСТ 27463-87 ГОСТ 27466-87 ГОСТ 27817-88 \\ \hline
	\end{tabular}
\end{table}

\begin{table}[h!tp]
	\centering
	\caption{\leftline{Национальные, модифицированные по отношению к международным}}
	\label{table:national:international:mod}
	\begin{tabular}{|p{10em}|p{12em}|p{12em}|}
		\hline
		\textbf{Обозначение}
			& \textbf{ГОСТ Р 34.91-94}
			& \textbf{ГОСТ Р 34.1702.3-92} \\ \hline
		\textbf{Название стандарта}
			& Информационная технология. Взаимосвязь открытых систем. Методология и основы аттестационного тестирования. Часть 6. Спецификация тестов протокольного профиля
			& Информационная технология. Машинная графика. Связь ядра графической системы с языком программирования Ада \\ \hline
		\textbf{Индекс стандарта}
			& ГОСТ Р & ГОСТ Р \\ \hline
		\textbf{Регистрационный номер}
			& 34.91 & 34.1702.3 \\ \hline
		\textbf{Номер комплексной системы стандартов}
			& 34 & 34 \\ \hline
		\textbf{Код ОКС}
			& 35.100 & 35.060 35.140 \\ \hline
		\textbf{Категория документа}
			& Нац, мод межд & Нац, мод межд \\ \hline
		\textbf{Организация по стандартизации}
			& МНИЦ Комитета при Президенте РФ по политике информатизации
			& ТК 22 Информационные технологии \\ \hline
		\textbf{Область применения в ИТ}
			& Взаимосвязь открытых систем
			& Машинная графика \\ \hline
		\textbf{Объект стандартизации}
			& Методология и основы аттестационного тестирования
			& Связь ядра графической системы с языком программирования АДА \\ \hline
		\textbf{Аспект стандартизации}
			& Спецификация тестов протокольного профиля
			& Имена и списки параметров процедур на языке Ада, соответствующие функциям ЯГС; имена типов данных ЯГС в языке Ада; структуры данных ЯГС в языке Ада; имена функций обработки ошибок \\ \hline
		\textbf{Последнее изменение}
			& 16.05.1995
			& 01.01.2021 \\ \hline
		\textbf{Связанные стандарты}
			& ISO/IEC 9646-6:1991
			& ГОСТ 27817-88 ИСО 8651-3-88 \\ \hline
	\end{tabular}
\end{table}

\clearpage

\section*{\LARGE Вывод}
\addcontentsline{toc}{section}{Вывод}

В ходе выполнения практической работы были рассмотрены
и проанализированы стандарты из различных категорий.
Эти стандарты охватывают различные аспекты информационных технологий,
управления качеством и информационной безопасности,
что позволило глубже понять особенности и различия между международными,
национальными и межгосударственными стандартами,
а также их роль в разработке программного обеспечения.

Таким образом, анализ вышеупомянутых стандартов
из различных категорий позволил систематизировать знания о стандартизации
в области ИТ и разработки ПО. Выбранные стандарты наглядно демонстрируют,
как различные типы стандартов взаимодействуют
и дополняют друг друга для обеспечения высокого уровня качества,
безопасности и совместимости программных решений.
